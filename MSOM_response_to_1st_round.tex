
\documentclass[12pt]{article}
\usepackage{geometry}
\geometry{margin=1in}
\usepackage{setspace}
\usepackage{titlesec}
\usepackage{enumitem}
\usepackage{hyperref}
\usepackage[dvipsnames]{xcolor}
\usepackage{natbib}

% Highlight Setting
\usepackage{xcolor, soul}
\sethlcolor{yellow}

% Private macros here (check that there is no clash with the style)
\usepackage{latexsym, amsthm, amsmath, amssymb, color, rotating, multirow, graphicx, subfigure, subcaption, array, tablefootnote}
\usepackage{threeparttable,booktabs}
\usepackage{enumitem}
%\usepackage[center]{caption}
\usepackage{caption}
\usepackage{tikz}
\usepackage[cjk]{kotex} %Korean TeX
\usepackage[margin=1in]{geometry}
\usepackage{amsfonts}
\usepackage{mathrsfs}
%\newtheorem{definition}{Definition}%[section]
%\newtheorem{theorem}{Theorem}%[section]
\newtheorem{prop}{Proposition}%[section]
\newtheorem{Lemma}{Lemma}%[section]
%\newtheorem{corollary}{Corollary}%[section]
%\newtheorem{assumption}{Assumption}%[section]
%\newtheorem{result}[theorem]{Result}
%\newtheorem{remark}{Remark}%[section]
%\newtheorem{example}{Example}%[section]
\newtheorem{cond}{Condition}

% URL control
\usepackage{url}
\usepackage{xurl}

%%%%%%%%%%%%%%%%%%%%%%%%%%%%%%%%%%%%%%%%%
\begin{document}
\begin{center}
{{\large Response to Review Reports on MSOM-2024-1493}\\[6mm]
{\Large \bf Sharing Economy in the Cloud: }\\[2mm]
{\Large \bf Pricing Schemes for Peer-to-Peer Storage Platforms}\\[10mm]}
\end{center}

\baselineskip 18pt
% ===============================================================

\textbf{Status 정의} (Color Rules): (Default: 검정색) \textbf{1. AI 답변}\\ 

\textbf{2. JE 수정} 
\begin{itemize}
    \item (\textcolor{blue}{파란색}) 2-1. 진행 중
    \item (\textcolor{red}{빨간색}) 2-2. 작성 완료 - JE의 가설로 작성(영재 검토 필요)
    \item (\textcolor{red}{빨간색}) 2-3. 작성 완료 - JE 본문 반영 완료(영재 검토 필요)
    \item (검정색 + \textcolor{violet}{코멘트 보라색}) 2-4. 작성 완료 - 리뷰 불필요
\end{itemize}

\textbf{3. YJ 리뷰 (\textcolor{red}{2-2} / \textcolor{red}{2-3}에 대하여)}
\begin{itemize}
    \item (\textcolor{olive}{올리브}) 3-1. 완료 - 본문 반영 필요
    \item (검정색 + \textcolor{violet}{코멘트 보라색}) 3-2. 완료 - 본문 반영 완료 or 미필요
\end{itemize}

\vspace{1cm}
\textbf{수정된 본문}(``MSOM\_manuscript\_since\_2025-09.tex'') (Color Rules):
\begin{itemize}
    \item \textcolor{red}{빨간색}: 추가/수정된 내용
    \item \textcolor{brown}{브라운}: 날렸으면 하는 내용 - 코멘트로 날렸으면 하는 이유 같이 남기기
\end{itemize}

%1. numerical \\
%2. insight 정리 \\ 
%3. 증명 수정 (or 추가)\\

%response 전체 방향성 align\\
%-> 본문에서 주요하게 바꿀 부분들 align\\

\vspace{1cm}
\textbf{작업 필요 사항}
\begin{itemize}
    \item 라이팅 정교화 (notation, logic 등등)
    \item 모든 참고문헌 / 시장조사 내용 최신화 
    \item numerical study -- consistency (분포별/paremeter별)
    \item theorem 인사이트 구체화 -- 결과만 써져있는 친구들 인사이트 추가 (volume관련 or surplus decompose)
    \item utility에 K 반영 (증명으로 할 지 numerical로 할 지는 시간보고 결정)
    \item design parameter로 명확히 읽히도록 전반적인 라이팅 재점검
    \item algorithm 관련해서 내용 재정비 
    \item Contribution (vs. two-sided platforms)
\end{itemize}

\vspace{1cm}
\textbf{Timeline}
\begin{itemize}
    \item 2025-11-09: 재웅 response 다 읽고 + 영재 본문 쭉 읽고 -- 서로 생각 방향 공유 
    \item 2025-11-16: 전체 미팅 및 방향성 얼라인(토/일 밤 10시 이후)
    \item 2025-11-23:
    \item 2025-11-30: 재웅 response 초안 전부 작성 / 영재 numerical 공유
    \item 2025-12-07: 영재 response 검토 / 재웅 본문 반영 시작
    \item 2025-12-14: 
    \item 2025-12-21:
    \item 2025-12-28: 본문/response 초안 완성 + 박건수 교수님 / 조대곤 교수님 리뷰
    \item 2025-01-04: 본문/response 리뷰 반영
    \item 2025-01-11: 본문/response 완성 + Cover letter 완성
    \item 2025-01-18 (제출주): 
\end{itemize}

% ===============================================================
\section*{Overall Summary and Action Plan for Revision}

\noindent \textbf{1. 전체 리뷰 총평 (Meta Summary)} \\[0.3em]
본 논문은 P2P 스토리지 플랫폼의 가격 정책을 분석한다는 점에서 주제의 참신성 및 실무 관련성은 AE와 DE 모두에게 인정받았습니다. Reviewer 1은 비교적 우호적이며 “minor-level” 개선을 요구하였고, Reviewer 2는 가정과 모델 구조에 대해 보다 근본적인 문제를 제기하며 “very major revision” 수준의 수정을 요청했습니다. Associate Editor는 두 리뷰어의 의견을 통합하여, \textbf{모델 가정의 정당성, 내생/외생 변수 설정, equilibrium의 엄밀성, 노테이션 및 설명력 부족}을 핵심 이슈로 지적하며 \textbf{major revision}을 권고하였습니다. Department Editor 또한 AE의 의견에 동의하며, 공식적으로 \textbf{major revision} 결정을 내렸습니다.

\vspace{1em}
\noindent \textbf{2. 주요 요구사항 요약}
\begin{itemize}
    \item \textbf{(가정 정당화)} — 왜 렌터가 redundancy-adjusted 용량으로 요금을 내는가? 공급자는 예약(capacity) 기준 보상을 받는가, 실제 사용량 기준인가? Provider와 Renter의 실체(개인/기업 등) 명확화 필요.
    \item \textbf{(모델의 엄밀성)} — 공급자 참여가 rational expectations equilibrium임을 본문에서 수식과 함께 명확히 설명해야 함. Market-clearing의 정의 및 비-유일성(non-uniqueness)도 명시 필요.
    \item \textbf{(유틸리티/수익 구조)} — 렌터 효용에 저장 자체의 가치($K$) 또는 접근 실패 리스크를 포함할 필요. Simulation을 통해 다양한 $K$ 값을 실험하거나, 최소한 일정 상수로 modeling.
    \item \textbf{(q parameter 해석 변경)} — $q$를 연속적인 결정변수라기보다는, “정책/요금 구조의 구조적 차이”를 나타내는 설계변수(design parameter)로 해석하고 표현을 일관되게 수정.
    \item \textbf{(표현 및 용어 정교화)} — “first-best pricing” 용어는 사회후생 최대를 의미하므로 부정확. “unconstrained profit-maximizing pricing” 등으로 교체 필요. Market-clearing, pricing scheme 등 모든 용어를 명확히 정의해야 함.
    \item \textbf{(문헌 리뷰 보완)} — P2P storage vs. P2P computing vs. file sharing 비교, cloud pricing 및 subscription 문헌 보강 요청.
\end{itemize}

\vspace{1em}
\noindent \textbf{3. 수정 작업 계획 (Action Plan / Work Breakdown)}
\begin{enumerate}
    \item \textbf{Introduction \& Assumption Motivation 강화}
    \begin{itemize}
        \item Redundancy 기반 요금 구조와 실제 산업 사례(Stroj, Filecoin 등) 추가.
        \item Provider 보수 방식(예약된 공간 vs. 실제 사용량)에 대한 설명 또는 현행 플랫폼 관행 인용.
        \item Provider/Renter 유형과 uptime 보장·위반 시 메커니즘 보완.
    \end{itemize}

    \item \textbf{Model Section 재정비}
    \begin{itemize}
        \item Endogenous vs. exogenous variables 명확화: 가격 변수($p_s,p_b,q$)는 내생, redundancy/uptime은 고정 또는 기술적 제약으로 설명.
        \item Provider 참여를 rational expectations equilibrium으로 명시하고, $n_p$와 $n_r$ 간 고정점 조건을 본문에 정식화.
        \item Market-clearing 가격 정의를 명확히 제시하고, 복수해가 존재 가능함을 언급.
    \end{itemize}

    \item \textbf{Utility/Profit Function 보완}
    \begin{itemize}
        \item Renter 효용에 저장 자체 가치 $K$ 포함: $U_i = K + \lambda_i u_i$ 형태 또는 simulation으로 다양한 $K$ 확인.
        \item Fail probability에 따른 기대효용 감소를 간단히 언급하거나 robustness test로 처리.
    \end{itemize}

    \item \textbf{Pricing Scheme 및 $q$ 해석 정리}
    \begin{itemize}
        \item $q$를 연속변수로 보기보다 정책 유형으로 설명: low-$q$ (traffic-sensitive hybrid), high-$q$ (flat-rate-like hybrid) 구분.
        \item Two-part tariff vs. high-$q$ hybrid를 동일선상에서 비교하는 simulation 포함.
    \end{itemize}

    \item \textbf{Simulation 및 Robustness Analysis}
    \begin{itemize}
        \item 다양한 $K$, $q$, consistency test, redundancy 수준 변화 시 결과 유지 여부 수치로 확인.
        \item Commission rate endogenous일 때의 결과도 numerical하게 포함.
    \end{itemize}

    \item \textbf{Literature Review 확장}
    \begin{itemize}
        \item P2P storage vs. P2P cloud computing vs. file sharing 문헌 비교 강화.
        \item Cloud-based pricing (Li \& Kumar 2022; Chen et al. 2023 등) 추가.
        \item Subscription pricing (Balasubramanian 2015 등) 반영.
    \end{itemize}

    \item \textbf{Notation \& Exposition 개선}
    \begin{itemize}
        \item 모든 기호 정의를 첫 사용 시 명시, table로 정리.
        \item Appendix에만 있는 중요한 수식 일부는 본문으로 이동 (proof sketch 수준).
        \item 오타 수정 (e.g., Lemma ???, $v_s^T$ missing $n_r$ factor 등).
    \end{itemize}
\end{enumerate}

\vspace{1em}
\noindent \textbf{4. 향후 리스크 및 유의점}
\begin{itemize}
    \item 모델 전면 수정은 요구되지 않으나, \textbf{가정의 논리적 설득력 + equilibrium의 명확성}이 확보되지 않으면 R2 또는 AE에서 재차 문제 제기 가능.
    \item 특히 redundancy 기반 가격 부과, provider 참여 논리, $q$의 해석은 \textbf{가장 중요한 3대 코어 포인트}이므로 서술의 일관성과 근거가 필수.
    \item 모든 대응은 수학적 rigor뿐 아니라 \textbf{실제 플랫폼 운영 맥락}과 연결되어야 설득력이 올라감.
\end{itemize}

% ===============================================================


% ===============================================================
\section*{Sections Requiring Revision in the Main Manuscript}

\noindent 아래는 Reviewer 1, Reviewer 2, AE, 그리고 DE의 코멘트를 바탕으로, \textbf{논문 본문에서 실제로 수정이 필요한 위치와 내용}을 Section별로 정리한 것이다. 각 항목은 본문 수정 시 참조할 수 있도록 요약되어 있다.

\vspace{1em}
\noindent \textbf{1. Introduction}
\begin{itemize}
    \item P2P 스토리지 시장 규모 및 최신 데이터 업데이트 (2023–2024년 기준 Storj, Filecoin 등).
    \item 왜 \textit{redundancy-adjusted storage} 기준으로 가격 부과가 이루어지는지 실증적 사례를 근거로 추가.
    \item 기존 클라우드(subscription 기반) 대비 본 논문의 차별점(특히 multi-sided incentive 구조)을 명확히 설명.
    \item ``First-best pricing''이라는 용어는 삭제 또는 ``unconstrained profit-maximizing pricing'' 등으로 변경.
\end{itemize}

\vspace{1em}
\noindent \textbf{2. Literature Review}
\begin{itemize}
    \item P2P storage vs. P2P computing vs. P2P file-sharing의 구조적 차이를 명확히 기술.
    \item Cloud pricing 관련 최신 문헌 추가 (예: Li \& Kumar 2022; Chen et al. 2023).
    \item Subscription pricing 관련 마케팅/정보시스템 문헌 (Balasubramanian et al. 2015 등) 보완.
\end{itemize}

\vspace{1em}
\noindent \textbf{3. Model Setup and Assumptions}
\begin{itemize}
    \item Provider 및 Renter가 실제로 누구인지(개인/기업) 명확히 정의하고, 저장/대역폭 비용이 각각 어떤 의미인지 설명.
    \item Provider 보상 방식: ``reserved capacity'' 기준인지, ``actual usage'' 기준인지 명확히 설명하고 필요 시 실제 플랫폼 사례 인용.
    \item Uptime, redundancy 불이행 시 페널티 또는 통제 방식 설명 (기술적으로 간략히라도 필요).
    \item 모든 기호($p_s, p_b, q, \alpha, m/k, \xi, \lambda_i$, etc.)를 등장 시점에 명확히 정의하고, 별도 notation table 추가.
\end{itemize}

\vspace{1em}
\noindent \textbf{4. Utility and Profit Functions}
\begin{itemize}
    \item Renter 효용식을 $U_i = \lambda_i u_i$에서 $U_i = K + \lambda_i u_i$ 또는 $K$ sensitivity simulation 형태로 확장.
    \item 데이터 접근 실패 시 효용 손실 포함 여부에 대해 주석 또는 보완 논리 제공.
    \item Provider profit은 ``기대 참여자 수'' 기반의 rational expectations equilibrium임을 명시적으로 본문 설명.
    \item $v_s^T, v_b^T$ 등 일부 식에서 $n_r$가 누락된 부분 수정.
\end{itemize}

\vspace{1em}
\noindent \textbf{5. Equilibrium and Market-Clearing Prices}
\begin{itemize}
    \item Market-clearing price 정의를 수학적으로 명시: 공급자 참여 인센티브와 총 공급–수요 균형 조건.
    \item Market-clearing 가격은 $(p_s,p_b)$ 조합이 \textbf{유일하지 않을 수 있음}을 명확하게 언급.
    \item 주요 결과는 공급자 수 $n_p$가 아니라, \textbf{공급/수요 비율 $n_p / n_r$}을 기준으로 정리.
    \item 기존 ``first-best pricing'' 용어는 ``benchmark pricing'' 또는 ``unconstrained optimal pricing''으로 교체.
\end{itemize}

\vspace{1em}
\noindent \textbf{6. Pricing Schemes and Interpretation of \textit{q}}
\begin{itemize}
    \item $q$를 단순 연속 변수로 보지 않고, ``설계 파라미터(design parameter)'' 또는 ``pricing scheme category''로 재정의.
    \item $q<1$ (traffic-sensitive), $q=1$ (two-part), $q>1$ (subscription-like) 구간을 명확한 구조로 구분.
    \item High-$q$ hybrid vs. pure two-part tariff 비교는 analytical theorem이 없으므로 numerical simulation으로 보완.
\end{itemize}

\vspace{1em}
\noindent \textbf{7. Theorems and Analytical Results}
\begin{itemize}
    \item Theorem 1: failure probability constraint 포함 여부 명확히 하고, 필요한 경우 조건을 추가하거나 가정 정당화.
    \item Lemma 3에서 ``belong to the market-clearing prices'' 표현을 보다 정확한 수학 언어로 수정.
    \item Theorem 3, 4 등 결과는 $n_p/n_r$ 기반으로 재표현하고, pricing 비교에서 누락된 two-part tariff 항목 반영.
\end{itemize}

\vspace{1em}
\noindent \textbf{8. Numerical Simulation and Robustness}
\begin{itemize}
    \item 다음의 sensitivity simulation을 추가:
    \begin{enumerate}
        \item Renter utility $K$ 변화에 따른 profit/surplus 결과
        \item $q$ 변화 (low, medium, high) 구간별 비교
        \item redundancy 수준 $(m/k)$ 변화 시 결과
        \item commission rate을 endogenous로 두었을 때 결과 유지 여부
    \end{enumerate}
    \item 모든 결과는 table/figure로 구성하고, 본문에서 최소한의 직관과 해석 제공.
\end{itemize}

\vspace{1em}
\noindent \textbf{9. Appendix and Notation Table}
\begin{itemize}
    \item Appendix에만 있던 equilibrium 유도 및 기대값 계산 과정 중 핵심 부분은 본문에 ``proof sketch'' 형태로 이동.
    \item Notation summary table을 Appendix 또는 본문 초반에 추가.
    \item Lemma ??, reference index 오류 등 typo 전체 점검 및 수정.
\end{itemize}

% ===============================================================

\newpage
%%%%%%%%%%%%%%%%%%%%%%%%%%%%%%%%%%%%%%%%%
\noindent \underline{\large \bf General Response to the Review Team}
%%%%%%%%%%%%%%%%%%%%%%%%%%%%%%%%%%%%%%%%%

\vspace{4mm}
\noindent The authors are grateful for constructive feedback to the Department Editor, Associate Editor, and anonymous reviewers. \hl{[...]}

\textcolor{blue}{[To be written after completing each response]}

%%%%%%%%%%%%%%%%%%%%%%%%%%%%%%%%%%%%%%%%%
\newpage
\noindent \textbf{\underline{Authors' Response to Department Editor's Comments}}\\
%%%%%%%%%%%%%%%%%%%%%%%%%%%%%%%%%%%%%%%%%
\section*{Summary of Department Editor’s Comments}

\noindent \textbf{전반적인 평가}
\begin{itemize}
    \item 논문 주제는 MSOM의 \textit{services, platforms, and revenue management} 분야와 잘 부합하며 적절한 투고라고 판단됨.
    \item AE와 두 명의 리뷰어가 평가를 완료하였으며, 전체적인 반응은 \textbf{긍정적}이라고 언급.
    \item AE는 \textbf{major revision}을 공식적으로 추천했고, DE도 이에 동의하며 최종 결정은 \textbf{major revision}.
\end{itemize}

\vspace{0.5em}
\noindent \textbf{요구되는 주요 수정 방향}
\begin{itemize}
    \item 가정(assumptions)의 동기 및 정당성을 보다 명확하게 설명할 것.
    \item 일부 모델 가정을 완화하거나, 최소한 \textbf{왜 필요했는지 명확히 정당화}할 것.
    \item 분석의 논리적 엄밀성(analytical rigor)을 강화할 것.
    \item 논문의 전반적인 서술력 및 가독성(exposition)을 개선할 것.
\end{itemize}

\vspace{0.5em}
\noindent \textbf{DE의 판단}
\begin{itemize}
    \item AE 및 리뷰어의 코멘트는 매우 성실하고 논리적이며, DE는 이 판단을 존중함.
    \item 따라서 \textbf{“major revision”}이 공식적인 에디토리얼 결정임.
\end{itemize}

\newpage
\begin{quotation}
{\em
\noindent \textbf{Department Editor:} Thank you for submitting your paper to MSOM.  The topic of your paper is of clear interest to the services, platforms, and revenue management department, and as such I sent your paper to an expert Associate Editor (AE) for evaluation.  The AE in turn recruited two reviewers.  All three members of the review team have now submitted their reports, and I want to thank each of them for their service to the journal.
}
\end{quotation}

\noindent \textbf{Authors' Response:} 
We sincerely appreciate your careful handling of our manuscript and the coordination of the review process. We are grateful to the Associate Editor and the two reviewers for their time, expertise, and constructive comments, which have been extremely helpful in strengthening our work.

\begin{quotation}
{\em
\noindent \textbf{Department Editor:} As you will see in the reports, the overall reaction to the paper is positive, and the AE has recommended a major revision. However, the AE (in an exceptionally thorough report) and the reviewers have provided quite a long list of comments and areas for improvement.  I am not going to repeat the issues raised by the review team here, but suffice it to say that these comments center around stronger motivation of some assumptions, potentially relaxing other assumptions, ensuring the analytical rigor is high, and improving the exposition.
}
\end{quotation}

\noindent \textbf{Authors' Response:} 
We are grateful for the overall positive assessment of our work and sincerely appreciate the thorough and detailed reports provided by the Associate Editor and the reviewers. We recognize that the review team has identified several important areas for improvement from the comments raised by the review team.

%To fully address each concern raised, we have carefully studied all comments and have made substantial revisions. We believe that these revisions have significantly improved the manuscript.

\begin{quotation}
{\em
\noindent \textbf{Department Editor:} Given the care and thought put into the reports, I see no reason to disagree with the AE's recommendation, and my decision is a major revision.  Thank you once again for submitting your work to MSOM, and best of luck in revising the paper.
}
\end{quotation}

\noindent \textbf{Authors' Response:} 
We sincerely appreciate your careful assessment and respect your decision for a major revision. We have carefully addressed the comments of the Associate Editor and the reviewers and have made substantial revisions throughout the paper. 
We are grateful for the thoughtful feedback from the review team, which improved this manuscript significantly. 

%%%%%%%%%%%%%%%%%%%%%%%%%%%%%%%%%%%%%%%%%
\newpage
\noindent \textbf{\underline{Authors' Response to Associate Editor's Comments}}\\
%%%%%%%%%%%%%%%%%%%%%%%%%%%%%%%%%%%%%%%%%
% ===============================================================
\section*{AE 코멘트 요약}

\noindent \textbf{■ AE의 전체 평가 및 톤}
\begin{itemize}
    \item 논문의 주제는 흥미롭고 시의성이 있으며, MSOM에 기여할 잠재력이 있다고 평가함.
    \item 다만 \textbf{“주요 수정(major revision)”}을 요구하며, 제기된 코멘트들이 단순 문장 교정 수준이 아닌 \textbf{핵심 모델 구조·가정·서술 방식의 개선}을 필요로 한다고 강조함.
    \item 전체적인 톤은 \textbf{건설적이며 호의적이지만, 매우 신중하고 엄격한 수정}을 요청하고 있는 형태.
\end{itemize}

\vspace{0.7em}
\noindent \textbf{■ AE가 인정한 장점}
\begin{itemize}
    \item P2P 스토리지 플랫폼의 pricing 문제를 다룬 점은 참신하고 현재 산업에서 중요한 주제임.
    \item 수리 모델과 분석은 의미 있고, 구조는 의도는 명확함.
    \item Reviewer 1과 2 모두 높은 수준의 코멘트를 제공해주었으며, 그것을 충실히 반영하면 좋은 논문이 될 수 있다고 판단함.
\end{itemize}

\vspace{0.8em}
\noindent \textbf{■ AE가 제기한 핵심 우려사항 (주제별 정리)}

\begin{enumerate}
    \item \textbf{(모델 가정에 대한 정당성 부족)}
    \begin{itemize}
        \item 왜 렌터가 실제 저장 용량이 아니라 \textbf{중복 보관 반영 후 저장량}으로 요금을 내는가? 이에 대한 실무적 설명 필요.
        \item 공급자는 실제 사용된 용량 기준으로 보상을 받는가, 아니면 예약된(할당된) 용량 기준인가?
        \item Provider와 Renter는 어떤 유형의 경제 주체인지(기업? 개인?)를 더 명확히 해야 하며, uptime, redundancy 비준수 시 어떻게 처리되는지 설명 필요.
    \end{itemize}

    \item \textbf{(모델의 엄밀성 및 내생/외생 변수 구분)}
    \begin{itemize}
        \item 가격 관련 변수(예: 커미션율, free bandwidth allowance $q$)는 플랫폼이 결정하는 값인데 왜 외생처럼 두었는가?  
        \item 반면 redundancy $(m/k)$나 uptime $\alpha$는 기술적 요소인데 왜 내생적으로 다루는가?
        \item 공급자 참여는 \textbf{기대 참여자 수에 기반한 rational expectations equilibrium}인데, 이를 본문에서 명확히 정의해야 함.
        \item 렌터 효용은 $U=\lambda u$만이 아니라, 저장 자체의 가치($K$) 또는 접근 실패 시 손실도 고려 가능해야 한다는 지적.
    \end{itemize}

    \item \textbf{(서술 및 용어 정의 개선 필요)}
    \begin{itemize}
        \item “First-best pricing” 용어 사용이 부정확함 → 경제학적으로는 사회후생 최대화를 의미하므로, “공급 여유 시 플랫폼 이익 최대화 가격”이라는 의미라면 다른 용어 사용 필요.
        \item Market-clearing 결과는 공급자 수 $n_p$가 아니라 \textbf{공급/수요 비율 $n_p/n_r$}로 표현하는 것이 더 적절함.
        \item “Market-clearing pricing”의 정의 자체가 본문에 없음 → 반드시 수학적으로 정의 필요.
        \item Market-clearing 해는 $(p_s,p_b)$의 조합이 여러 개 존재할 수 있음 → 유일하지 않을 수 있다는 점 명시 필요.
        \item 중요한 직관·도출 과정을 Appendix에만 두지 말고, 본문에 “proof sketch” 형태로 일부 포함해야 함.
        \item 모든 기호와 notation은 등장 시점에 명확히 정의되어야 하며, Notation Table 제공이 필요함.
    \end{itemize}

    \item \textbf{(문헌 리뷰 보완 필요)}
    \begin{itemize}
        \item P2P 컴퓨팅 자원 공유 논문과의 차이점을 더 명확히 구분할 것.
        \item P2P 파일 공유(file-sharing)와 P2P storage는 근본적으로 목적과 인센티브 구조가 다르므로 명확한 대비 필요.
        \item 클라우드 서비스 및 storage pricing 관련 최신 연구(Li \& Kumar 2022; Chen et al. 2023 등) 추가할 것.
        \item Subscription pricing은 마케팅/정보시스템 분야에서도 정교한 이론들이 있으므로, 해당 문헌(Balasubramanian et al. 2015 등)도 충분히 반영할 것.
    \end{itemize}
\end{enumerate}

\vspace{0.8em}
\noindent \textbf{■ 전반적으로 주의할 점}
\begin{itemize}
    \item 용어(First-best, Market-clearing 등)를 정확히 정의하고 논문 전체에 일관되게 사용할 것.
    \item 모형의 내생·외생 변수 구분을 명확히 하고, equilibrium 의사결정 구조를 수식과 문장으로 분명히 표현.
    \item Referee 1, 2가 제기한 모든 세부 코멘트와 충돌 없이, AE의 요구를 통합적으로 반영해야 함.
\end{itemize}


% ===============================================================
\newpage

\begin{quotation}
{\em
\noindent \textbf{Associate Editor (Summary of the paper):} The paper develops a game-theoretic model to examine how various pricing schemes employed by peer-to-peer (P2P) storage platforms impact platform profits and overall system surplus. The study considers three pricing schemes: the widely-used two-part tariff employed by P2P storage platforms and two schemes commonly adopted by conventional public cloud services – subscription-based and hybrid pricing. The authors further integrate the endogenous selection of redundancy algorithms to determine profit-maximizing service fees under each pricing scheme. The findings reveal that when “first-best” prices – those that maximize profit under ample provider availability – are feasible, the two-part tariff maximizes platform profit, whereas subscription-based pricing yields the highest total surplus. Conversely, when providers are scarce, the so-called “market-clearing” prices must be employed to balance supply and demand. In this case, the two-part tariff and hybrid pricing outperform subscription-based pricing in terms of both profit and total surplus. To support these conclusions, the authors also investigate endogenous commission rates and test key assumptions, confirming the robustness of their primary insights. 
}
\end{quotation}

% ===============================================================
\noindent \textbf{Comment Analysis (코멘트 분석)}  
- 이 부분은 질의나 비판이 아니라, AE가 논문의 핵심 내용을 정확히 요약한 것이다.  
- AE는 논문의 스코프(three pricing schemes + redundancy), 주요 결과(first-best vs market-clearing), profit vs surplus 대비, 그리고 robustness 검증까지 잘 이해하고 있다는 신호이다.  
- 즉, AE의 기본 태도는 “이 논문이 무엇을 하고 있는지는 명확히 이해했다”는 긍정적 출발점에 가깝다.

\vspace{0.8em}

\noindent \textbf{Response Strategy (답변 방향)}  
- 이 파트는 비판이 아니므로 직접 반박하거나 설명할 내용은 없다.  
- 다만 AE가 논문을 정확히 이해해주었음을 인정하고 감사의 뜻을 간단히 표현하면 된다.  
- 이어서 등장할 실제 코멘트/우려에 대비해 “논문의 목적, 분석 구조, 기여를 명확히 했다”는 토대 위에서 대응을 시작하면 된다.

\vspace{0.8em}

\noindent \textbf{Response (영문)}  
Thank you for the accurate summary of our paper. We appreciate that you have clearly articulated the goal, modeling structure, key results, and robustness checks of our study. We are grateful that the main contributions of the paper – particularly the comparison of three implementable pricing schemes under both first-best and market-clearing environments, the role of redundancy decisions, and the profit–welfare trade-offs – were correctly captured. We now address your specific comments below.

\vspace{0.8em}

\noindent \textbf{Manuscript Changes (본문 수정 필요 여부)}  
- 해당 부분은 요약이므로 본문 수정을 요구하지 않음.  
- 단, 이후 AE가 제기할 구체적 코멘트에 따라 Abstract, Introduction, Results section에서 일부 표현을 강화/수정할 계획임.

\vspace{1.0em}
% ===============================================================

\noindent \textbf{Authors' Response:} \textcolor{blue}{Thank you for the accurate summary of our paper. We appreciate that you have clearly articulated the goal, modeling structure, key results, and robustness checks of our study.} \hl{[...]}

\vspace{1.0em}
% ===============================================================

\begin{quotation}
{\em
\noindent \textbf{Associate Editor (Review outcomes):} The paper was reviewed by two experts in analytical modeling. Reviewer 1 (R1) started with, “this paper develops and analyzes a stylized model . . .” and praised the paper’s focus on an “interesting and current business model.” R1 recommended a minor revision but provided a comprehensive list of points for improvement. Reviewer 2 (R2), however, expressed reservations about the industrial significance of the P2P storage platforms examined in the study. R2 also raised more substantial concerns regarding the paper’s assumptions, formulation, and analysis, ultimately recommending a very major revision. 

Having read the paper thoroughly, I must commend both reviewers for their detailed, thoughtful, and constructive feedback. I would like to mention that their reports are among the best I have encountered recently. I sincerely thank them for their time and effort in reviewing this submission. 
}
\end{quotation}


%-----------------------------
\noindent \textbf{Comment Analysis (코멘트 분석)}
%-----------------------------
\begin{itemize}
    \item AE는 R1과 R2의 평가가 서로 다른 방향임을 명확히 짚고 있다:  
    \begin{itemize}
        \item R1 → “흥미롭고 시의적절한 주제 + 모델도 좋다 → minor revision”  
        \item R2 → “산업적 중요성 의문 + 가정과 모델 설명 부족 → very major revision”
    \end{itemize}
    \item 이는 AE가 단순히 minor 혹은 major로 결론 내린 것이 아니라, \textbf{“두 리뷰의 균형을 반영하여, 본격적인 수정이 필요하다”}는 판단을 하고 있음을 시사한다.
    \item AE의 이 문장은 비판 자체라기보다는 “두 리뷰 모두 중요하며, R2가 더 강하게 문제 제기했기 때문에 그 부분을 중심으로 해결할 필요가 있다”는 메시지이다.
\end{itemize}

\vspace{0.8em}

%-----------------------------
\noindent \textbf{Response Strategy (답변 방향)}
%-----------------------------
\begin{itemize}
    \item AE에 대한 답변은 방어적이거나 변명식이 아니라, \textbf{감사의 표명 + 두 리뷰어 의견을 모두 진지하게 반영하겠다는 의지}를 먼저 보여주는 것이 중요하다.
    \item 특히 R2가 요구한 부분(assumption의 강도, equilibrium 구조 설명, q의 해석 및 비교, 산업적 타당성 등)을 얼마나 구체적으로 보완할지를 명확히 약속해야 AE를 설득할 수 있다.
    \item 따라서 이 섹션의 답변에서는:  
    \begin{enumerate}
        \item 두 리뷰어의 노력과 피드백에 대해 감사  
        \item R1의 minor-level 코멘트 + R2의 structural-level 코멘트 모두를 반영하겠다 명시  
        \item 특히 R2가 제기한 assumption, formulation clarity, industrial relevance를 강화할 것을 AE에게 분명히 약속  
    \end{enumerate}
\end{itemize}

\vspace{0.8em}

%-----------------------------
\noindent \textbf{Response (영문)}
%-----------------------------
We sincerely thank you for summarizing the two reviews and for overseeing the evaluation process. We are grateful to both reviewers for their thoughtful and constructive feedback. We appreciate that Reviewer~1 finds the topic relevant and the modeling framework promising, while providing helpful suggestions for clarity and exposition. At the same time, we fully acknowledge Reviewer~2’s concerns regarding the assumptions, formulation, and industrial significance of P2P storage platforms.

In the revised manuscript, we carefully address all comments raised by both reviewers. In particular, we incorporated the followings:
\begin{itemize}
    \item clarify and relax the modeling assumptions where appropriate and provide robustness checks through analytical and numerical analysis;
    \item improve the presentation of the provider participation equilibrium and market-clearing mechanism;
    \item better articulate the economic relevance and practical implications of P2P storage platforms, distinguishing them from traditional file-sharing or computing-sharing systems;
    \item clearly explain the role of the design parameter $q$ in hybrid pricing and provide additional numerical analysis for high-$q$ regimes.
\end{itemize}
We appreciate the opportunity to revise the paper and hope that the review team finds the revised version of the manuscript substantially clearer and more robust such that it now aligns well with the high standards of \textit{MSOM}.

\vspace{0.8em}

%-----------------------------
\noindent \textbf{Manuscript Changes (본문 수정 계획)}
%-----------------------------
\begin{itemize}
    \item 별도 문단 수정은 필요 없으나, AE와 R2가 제기한 문제 (assumption clarity, equilibrium 설명, 산업적 의미, $q$ 해석)를 반영하여 Introduction, Section~3–5, Discussion을 강화.
    \item AE 서문에 대한 직접 수정은 하지 않지만, 요약 및 Abstract에서 exaggerated statements(예: two-part always dominates)는 conditional language로 조정.
\end{itemize}

\vspace{1.0em}

% ===============================================================

\noindent \textbf{Authors' Response:} \textcolor{blue}{We sincerely thank you for summarizing the two reviews and for overseeing the evaluation process. We are grateful to both reviewers for their thoughtful and constructive feedback. We appreciate that Reviewer~1 finds the topic relevant and the modeling framework promising, while providing helpful suggestions for clarity and exposition. At the same time, we fully acknowledge Reviewer~2’s concerns regarding the assumptions, formulation, and industrial significance of P2P storage platforms.} \hl{[...]}

\vspace{1.0em}
% ===============================================================


\begin{quotation}
{\em
\noindent \textbf{Associate Editor (Main Comments):} I strongly recommend that the authors carefully address all the points raised by both reviewers to advance the paper to the next stage. Since the reviewer reports are detailed and comprehensive, I will refrain from repeating all their points here. Instead, I summarize below the key issues based on the reviewers' feedback, as well as my own observations and suggestions. 
}
\end{quotation}

%-----------------------------
\noindent \textbf{Comment Analysis (코멘트 분석)}
%-----------------------------
\begin{itemize}
    \item AE는 Reviewer 1과 Reviewer 2의 모든 코멘트를 진지하게 반영해야만 논문이 다음 단계(acceptable revision)로 갈 수 있다고 명확히 요구하고 있다.
    \item 동시에 상세 코멘트는 반복하지 않고, “핵심 쟁점만 요약해주겠다”고 말했으므로, 이후 등장할 AE Point \#1, \#2, ... 가 사실상 최종 판단을 좌우할 것이다.
    \item 해석하자면, \textbf{“단순히 Reviewer 1/2 의견에 답장을 쓰는 수준이 아니라, 논문 전체의 구조와 논지를 강화해달라”}는 요청이다.
\end{itemize}

\vspace{0.8em}

%-----------------------------
\noindent \textbf{Response Strategy (답변 방향)}
%-----------------------------
\begin{itemize}
    \item 이 코멘트 자체에는 구체적 반박이나 수정 요구가 없으므로, 우리의 답변은 \textbf{전반적인 동의 + 모든 reviewer 의견을 성실히 반영하겠다는 약속} 형태로 concise하게 가야 한다.
    \item 그리고 이어지는 AE의 각 핵심 포인트에서, Referee 1/2의 코멘트와 어떻게 연결되는지까지 고려하여 정교하게 대응할 예정이다.
\end{itemize}

\vspace{0.8em}

%-----------------------------
\noindent \textbf{Response (영문)}
%-----------------------------
We appreciate your guidance and completely agree that all points raised by both reviewers must be carefully addressed in order to move the paper forward. We are grateful that you have consolidated the reviewers’ detailed reports into key issues, and we will ensure that our revision tackles each of these concerns rigorously rather than merely providing superficial replies. 

In the following sections, we respond point-by-point to the major issues you identified, and we also detail how we revise the manuscript accordingly. We believe this approach will significantly enhance both the clarity and the contribution of the paper.

\vspace{0.8em}

%-----------------------------
\noindent \textbf{Manuscript Changes (본문 수정 계획)}
%-----------------------------
\begin{itemize}
    \item 이 코멘트는 직접적인 문장 수정 요구는 아니므로, 본문에 수정하지 않음.
    \item 다만 이후 AE가 제시하는 핵심 논점(Point \#1, \#2 등)에 따라 Introduction, Model, Equilibrium Section, Pricing Comparison, Numerical Analysis 전반을 수정할 예정임.
\end{itemize}

\vspace{1.0em}
% ===============================================================

\noindent \textbf{Authors' Response:} We appreciate your guidance and completely agree that all points raised by both reviewers must be carefully addressed in order to move the paper forward. We are grateful that you have consolidated the reviewers' detailed reports into key issues, and we will ensure that our revision tackles each of these concerns rigorously rather than merely providing superficial replies.

\textcolor{blue}{[Brief overview of our revision]}

In the following sections, we respond point-by-point to the major issues you identified, and we also detail how we revise the manuscript accordingly. We believe this approach will significantly enhance both the clarity and the contribution of the paper.

\vspace{1.0em}
% ===============================================================


\begin{quotation}
{\em
\noindent \textbf{Associate Editor (1. Justification of Context-Specific Assumptions):} Some of the paper’s context-specific assumptions require further justification to strengthen the study’s credibility. Below, I outline several areas of concern and related points for consideration: 
}
\end{quotation}

%-----------------------------
\noindent \textbf{Comment Analysis (코멘트 분석)}
%-----------------------------
\begin{itemize}
    \item AE는 본 연구가 설정한 몇 가지 “문맥 특화(Context-Specific)” 가정들—예를 들어 수요/공급의 단위 구조, Pareto 분포 사용, 비용 구조, redundancy 설정 등—에 대해 \textbf{“왜 그런 가정을 두었는지 충분히 설명되지 않았다”}고 우려하고 있다.
    \item 이 코멘트는 단일 항목이 아니라, AE가 이어서 나열할 여러 세부 질문(sub-points: 1-a, 1-b, 1-c …)의 상위 주제에 해당한다.
    \item 본 논문이 Stylized Model임을 감안하더라도, \textbf{“실제 산업적 맥락에서 왜 이 수치를 택했는지, 다른 분포나 파라미터에서도 결과가 유지되는지”}를 명확히 하라는 의미다.
\end{itemize}

\vspace{0.8em}

%-----------------------------
\noindent \textbf{Response Strategy (답변 방향)}
%-----------------------------
\begin{itemize}
    \item 전체 전략은 다음과 같다:
    \begin{enumerate}
        \item 각 가정의 \textbf{산업적/경제적 의미}를 논리적으로 설명한다 (why this assumption is reasonable or standard).
        \item 해당 가정이 핵심 결과에 영향을 주는지 여부를 \textbf{분리하여 설명}:  
            - “결과의 구조적 본질을 유지하는 가정” vs.  
            - “단순화(simplifying) 또는 calibration 목적의 가정”
        \item 필요한 경우 \textbf{numerical robustness test} 또는 \textbf{supplementary simulation}을 포함시키겠다고 명시한다.
        \item 모든 justification은 \textbf{본문(Introduction/Model Section)에 간결히 추가}, Appendix에는 더 긴 기술적 근거 또는 데이터 기반 사례를 포함.
    \end{enumerate}
    \item 이후, AE가 제시할 세부 이슈들(예: distributional assumption, demand=1 unit, cost condition $\xi > \cdot$, redundancy mechanism 등)에 대해 Referee 2 대응과 연결하여 구체적으로 point-by-point 답변을 작성할 것이다.
\end{itemize}

\vspace{0.8em}

%-----------------------------
\noindent \textbf{Response (영문)}
%-----------------------------
We appreciate your observation that several of our context-specific modeling assumptions require clearer justification. We fully agree that explicating the economic rationale behind these assumptions is essential for ensuring the study’s credibility and generalizability. 

In the revised manuscript, we will:
\begin{itemize}
    \item explicitly justify each key assumption (e.g., unit storage demand, Pareto-type bandwidth usage, cost structure, redundancy modeling) based on industry practices or standard modeling conventions in the literature;
    \item clarify which assumptions are essential for the structural insights versus those adopted for analytical tractability;
    \item where appropriate, provide robustness checks or numerical evidence to demonstrate that the main insights remain valid under alternative specifications.
\end{itemize}

We address each of your sub-points in detail below.

\vspace{1.0em}
% ===============================================================

\noindent \textbf{Authors' Response:} \textcolor{blue}{We thank the AE for raising this important point. In our model, renters are charged based on the redundancy-adjusted storage volume, i.e., the total amount of storage required to guarantee reliable data availability under the chosen redundancy scheme. While renters only upload 1 unit of data, the platform must store $m/k$ units in aggregate to provide the desired reliability.} \hl{[...]}

\vspace{1.0em}
% ===============================================================

\begin{quotation}
{\em
\noindent \textbf{Associate Editor (Point \#1a): Charging Renters Based on Redundancy-Adjusted Storage Volume vs. Actual Demand.} A key assumption in the model to justify is that renters are charged based on the redundancy-adjusted storage volume, rather than their actual demand for storage volume. This raises an important question: Will renters resist being charged for the redundancy factor (which is determined by the platform)? For instance, a renter seeking to store 1GB of data might question why they should pay for 2GB, even if redundancy enhances reliability. The counterargument could be that ensuring reliability is the platform’s responsibility, not the renter’s financial burden. This issue directly impacts the renters’ utility functions and, consequently, the platform’s revenue model. 
}
\end{quotation}

\noindent \textbf{Comment Analysis (코멘트 분석)}  
\begin{itemize}
    \item AE는 본 논문이 **renter는 실제 데이터 크기(예: 1GB)가 아니라, “redundancy가 반영된 저장량(m/k)”에 따라 비용을 지불한다**고 가정한 점을 문제 삼았다.
    \item 즉, “redundancy는 플랫폼이 정하는 기술적 결정인데, 왜 비용 부담이 renter에게 전가되는가?”라는 경제적 정당성을 요구하는 것이다.
    \item 이 가정은 Utility 설정, Platform profit 구조, Adoption condition 등 모델 전체에 영향을 주기 때문에 **정당화되지 않으면 논문 신뢰도가 무너질 수 있는 핵심 논점**이다.
\end{itemize}

\vspace{1em}

\noindent \textbf{Response Strategy (답변 방향)}  
\begin{itemize}
    \item 우리는 “redundancy 비용을 renter가 부담하는 구조”가 실제 산업 관행에서도 존재함을 설명할 것이다.
    \item 특히 **AWS S3, Google Cloud Storage, Backblaze B2, Storj, Filecoin 등** 대부분의 클라우드/분산 저장 서비스는:
    \begin{itemize}
        \item 가격을 “사용자가 업로드한 원본 파일 크기” 기준으로 보이게 하지만,
        \item 실질 billing은 내부적으로 **replication/erasure coding 후의 실제 저장량 기준**으로 계산됨.
    \end{itemize}
    \item 즉, 사용자는 “1GB를 저장”한다고 생각하지만 실제로는 2GB, 3GB 등 다양한 redundancy overhead가 가격 구조에 반영되고 있으며, 이는 standard industry practice이다.
    \item 또한 플랫폼이 redundancy를 제공하는 대신, **renter는 higher reliability\slash uptime이라는 편익을 얻기 때문에**, 경제적으로는 “premium reliability를 포함한 단가를 지불”한다고 해석할 수 있다.
    \item 본문에 이 논리를 명확히 추가하고, renter utility를 “effective cost per 1GB of reliable storage” 관점으로 재해석하겠다고 약속한다.
\end{itemize}

\vspace{1em}

\noindent \textbf{Response (영문)}  
We thank the AE for raising this important point. In our model, renters are charged based on the redundancy-adjusted storage volume, i.e., the total amount of storage required to guarantee reliable data availability under the chosen redundancy scheme. While renters only upload 1 unit of data, the platform must store $m/k$ units in aggregate to provide the desired reliability.

This pricing structure is consistent with common industry practice. For example, commercial cloud providers such as Amazon S3, Google Cloud Storage, and decentralized P2P platforms such as Storj and Filecoin implicitly charge renters for the redundant copies or coded fragments necessary to ensure durability (e.g., 99.999999999\% availability). Although the user interface displays the cost per unit of original data, the actual billing reflects the replicated or erasure-coded storage volume used internally by the provider.

Economically, we interpret the payment as the price for ``1 unit of reliable storage,'' not merely ``1 unit of raw disk space.'' The renter pays a higher effective price because the platform guarantees higher reliability, and the redundancy factor is the technological mechanism through which this reliability is achieved.

To avoid confusion, we will revise the manuscript to:
\begin{itemize}
    \item explicitly state that the price is for reliable storage service, which inherently requires redundant capacity;
    \item provide examples of industry pricing schemes where redundancy is reflected in the user-facing cost;
    \item clarify that the renter’s utility is derived from reliable storage, not from the raw storage volume alone.
\end{itemize}

\vspace{1em}

\noindent \textbf{Manuscript Changes (본문 수정 계획)}  
\begin{itemize}
    \item Section 2 (Model Assumptions)에 ``renter는 raw data 저장량이 아니라 reliable storage 단위를 구매하며, redundancy 비용은 reliability를 위한 내재된 가격 요소”라는 설명 추가.
    \item 실제 산업 사례를 각주 또는 문장으로 삽입:  
          (예: AWS S3, Storj, Filecoin, Microsoft Azure LRS/ZRS replication cost 구조)
    \item Utility function에서 $p_s + p_b \cdot (\text{expected redundant storage})$ 해석을 명확히 수정.
\end{itemize}

\vspace{1em}
% ===============================================================

\noindent \textbf{Authors' Response:} \textcolor{blue}{We appreciate your observation that several of our context-specific modeling assumptions require clearer justification. We fully agree that explicating the economic rationale behind these assumptions is essential for ensuring the study’s credibility and generalizability.} \hl{[...]}

\vspace{1.0em}
% ===============================================================

\begin{quotation}
{\em
\noindent \textcolor{red}{\textbf{Associate Editor (Point \#1b):} Payment to Providers: Reserved Capacity vs. Allocated Capacity.} Similar to the above question regarding how renters are charged by the platform in practice. R2 highlighted a concern about whether providers are compensated based on the storage capacity they set aside or the actual storage allocation determined by the platform. This distinction is critical because providers might perceive unfairness if their allocated storage demand falls short of their reserved capacity, resulting in underpayment. Addressing this issue would enhance the model’s alignment with practical business dynamics and provider incentives. 
}
\end{quotation}

\noindent \textbf{Authors' Response:} \textcolor{red}{We appreciate the review team's comment on possible compensation schemes for shared capacity. As you and Reviewer 2 (Point \#3) noted, our model assumes that providers are compensated based on the amount of data actually stored on their devices (i.e., the allocated capacity), rather than the total raw capacity they pledge to the platform. We address your concern about the plausibility of this assumption by (i) offering details of the real-world practices among P2P storage platforms and (ii) discussing how reserved capacity might interact with our model and results.}

\textcolor{red}{First, by investigating current practices, we find that our assumption is aligned with a common payment structure observed in decentralized P2P storage platforms. To be specific, Storj pays node operators for actual data stored or transferred, not reserved capacity (source: \url{https://storj.dev/node/payouts}). Sia also uses a usage‐based scheme or pays for what is stored with collateral and penalties tied to uptime and proofs, rather than paying simply for reserved capacity (source: \url{https://docs.sia.tech/provide-storage/about-hosting-on-sia}). Lastly, Filecoin combines usage fees, consensus block rewards, and collateral-based penalties. Providers do not simply get paid for reserved space, and they must actively prove storage over time and their rewards are tied to actual network performance and consensus incentives (source: \url{https://filecoin.io/blog/posts/the-economics-of-storage-providers/}).}

\textcolor{red}{Our profit function follows this practice. Providers decide whether to reserve 1 unit of capacity, but their revenue share is proportional to the expected stored data volume (adjusted for redundancy). We clarify this in the revised manuscript as follows:}

\begin{quotation}
{\em
\noindent \textcolor{red}{``Following industry practice, we assume that providers are compensated for actual storage and bandwidth usage rather than merely for reserving capacity.''}

\vspace{0.4cm}
\noindent \textcolor{red}{``[Footnote] Storj pays node operators for the actual amount of data stored or transferred, not for reserved capacity (\url{https://storj.dev/node/payouts}). Sia also employs a usage-based scheme in which users pay for data that is actually stored, with collateral and penalties tied to uptime and proofs, rather than payments for reserved capacity (\url{https://docs.sia.tech/provide-storage/about-hosting-on-sia}). Lastly, Filecoin combines usage fees, consensus block rewards, and collateral-based penalties: providers are not simply paid for reserved space, but must actively prove storage over time, and their rewards are tied to realized network performance and consensus incentives (\url{https://filecoin.io/blog/posts/the-economics-of-storage-providers/}).''} \hl{(page XX in Section 3.2.2)}
}
\end{quotation}

\textcolor{red}{Second, we have updated the manuscript to discuss how reserved capacity can affect the provider's utility function and our results. When we suppose a provider who is compensated for reserved capacity, the platform should pay for all reserved capacity regardless of its utilization, while renters pay for their usage only. Then, the functional form of $\pi_j$ can be rewritten as:}
\begin{equation*}
    \pi_j = R +\alpha (\hat{\omega}_s p_s + \hat{\omega}_{bp} p_b) - \rho_j \hat{\omega}_b \xi
\end{equation*}
\textcolor{red}{where $R$ indicates a fixed compensation for a provider's reserved capacity, which is independent of actual usage $\hat{\omega}_s$ and $\hat{\omega}_b$; the other notations are identically defined as the profit function in \textbf{Section 3.2.2}.}

\textcolor{red}{In this setting, some providers earn positive profits only at low utilization because their operating costs rise sharply with actual usage. As a result, they may be willing to join when $R$ is large and $\hat{\omega}_b$ is small, but once capacity becomes valuable they are motivated to exit immediately. Put differently, these providers gain only when there is a sufficiently large pool of participating providers while the network's capacity remains largely unused. Under such conditions, the platform’s business model is unlikely to be sustainable, which may help explain why existing firms do not pursue this approach. We include this discussion in \textbf{Section 6.2} as follows:}

\begin{quotation}
{\em
\noindent \textcolor{red}{``Third, one might be curious if it is a plausible option to consider compensating providers for reserved capacity rather than for realized usage. Suppose that a provider receives a fixed payment $R$ for the capacity it reserves, independently of the realized usage levels $\hat{\omega}_s$ and $\hat{\omega}_b$. Then the provider's profit can be written as $\pi_j = R + \alpha (\hat{\omega}_s p_s + \hat{\omega}_{bp} p_b) - \rho_j \hat{\omega}_b \xi$, where other notations are as in the profit function in Section~3.2.2. Under this scheme, some providers become profitable only when usage is low, because their operating costs associated with actual usage are high; that is, such providers may find it attractive to join the platform when $R$ is large and $\hat{\omega}_b$ is small, but they are then incentivized to leave once capacity becomes valuable and utilization increases. This mismatch between incentives and actual usage makes the platform's business model unlikely to be sustainable, which is consistent with the observation that leading decentralized storage platforms do not compensate providers solely for reserved capacity.''} \hl{(page XX in Section 6.2)}
}
\end{quotation}

\vspace{1.0em}
% ===============================================================



\begin{quotation}
{\em
\noindent \textbf{Associate Editor (Point \#1c): Uptime and Redundancy Parameters.} The paper should also clarify how frequently platforms adjust key redundancy parameters such as 
uptime and redundancy factor and whether these adjustments are synchronized or handled 
independently. Specific considerations include: 
\begin{itemize}
    \item Could uptime requirements be dynamically adjusted based on shifts in demand and supply? 
    \item What mechanisms are in place to address non-compliance with uptime requirements by 
providers (as noted by R1)? 
    \item Who typically constitute the providers and renters in P2P storage platforms (as noted by R1)? Are their primary concerns centered around storage costs, bandwidth costs, or a combination of both? Additionally, do they primarily value the storage itself, or are they more interested in the flexibility and ease of accessing stored data?  
\end{itemize}
}
\end{quotation}

%-----------------------------
\noindent \textbf{Comment Analysis (코멘트 분석)}
%-----------------------------
\begin{itemize}
    \item AE는 본 논문에서 redundancy $m/k$, uptime requirement $\alpha$, provider 타입 정의 등이 **수학적으로 주어져 있지만 현실적 동작 방식이 충분히 설명되지 않았다**고 지적한다.
    \item 구체적으로 요구하는 것은:
    \begin{enumerate}
        \item 플랫폼이 redundancy 또는 uptime threshold를 수시로 조정하는지?  
        \item uptime 미준수 시 어떤 penalty/보상 시스템이 존재하는지?  
        \item 공급자/수요자는 누구인지? 그들의 주요 비용/편익 고려 요소는 무엇인지?  
    \end{enumerate}
    \item 결국, 이는 모델의 \textbf{현실 기반 설득력(realism)}을 보강하라는 요구이며, Referee 1의 유사 질문(typical provider, renter, maintenance, incompetence risk)과 연결된다.
\end{itemize}

\vspace{0.8em}

%-----------------------------
\noindent \textbf{Response Strategy (대응 방향)}
%-----------------------------
\begin{itemize}
    \item redundancy factor $m/k$와 uptime $\alpha$는 본 논문에서 \textbf{“platform design parameters (long-term policy)”}로 취급하며, 단기적으로 자주 변하지 않는다고 정리할 계획.
    \item 단, 수요/공급 조건 변화 및 기술 발전에 따라 \textbf{주기적으로 재설정 가능함}을 인정하고, 현실 사례(AWS S3 Standard vs. Glacier, Filecoin의 storage deal expiry 등)를 근거로 설명.
    \item uptime 불이행 시:
    \begin{itemize}
        \item Filecoin: collateral slashing, penalty, storage contract 해지  
        \item Storj: reputation system 및 payout 감소  
        \item → 이러한 제재 구조가 $\alpha$ 만족을 강제하며, 우리 모델에서는 이를 “availability constraint + profit function penalty”로 내재화했다고 설명한다.
    \end{itemize}
    \item 공급자/사용자 정체성:
    \begin{itemize}
        \item Provider: 개인 고성능 NAS 사용자, 중소 데이터 센터, Idle data centers  
        \item Renter: 개인 사용자(backup), 스타트업/SME, DApp 개발자 등  
        \item 비용 고려: storage cost + bandwidth + uptime risk, renters는 “raw storage”보다 “reliable & accessible storage service”를 구매.
    \end{itemize}
\end{itemize}

\vspace{0.8em}

%-----------------------------
\noindent \textbf{Response (영문)}
%-----------------------------
Thank you for these thoughtful and practical questions. In our model, the redundancy level ($m/k$) and uptime requirement ($\alpha$) are treated as platform design parameters that are chosen at a strategic (medium- to long-term) horizon. In practice, major P2P and cloud storage platforms (e.g., Storj, Filecoin, Amazon S3) do not update these parameters continuously; instead, they set service tiers or storage contracts (e.g., S3 Standard, S3 Glacier, Filecoin storage deals) whose redundancy and uptime guarantees remain fixed for the contractual duration. These parameters can be revised over time when supply conditions, demand patterns, or technology evolve.

Regarding enforcement, if a provider fails to meet the uptime requirement, real-world platforms impose penalties: Filecoin slashes the provider’s collateral or terminates the contract; Storj reduces payout and reputation. Our model incorporates this as an expected cost (implicitly embedded in the provider’s payoff), and we will clarify this in the paper.

As for who participates in such markets: providers often include individuals or small data centers with idle disk capacity, while renters range from individual users seeking backup to small firms or decentralized applications. Providers incur storage and bandwidth costs, whereas renters are primarily paying for \emph{reliable accessibility of their data}, not just raw storage. We will revise the manuscript to explain these roles and motivations more clearly.

\vspace{0.8em}

%-----------------------------
\noindent \textbf{Manuscript Changes (본문 수정 사항)}
%-----------------------------
\begin{itemize}
    \item Section 2 또는 Introduction에 “redundancy and uptime are platform-level design choices, typically fixed over service contracts” 문장 추가.
    \item Provider payoff 설명에 “uptime enforcement via penalty or lost revenue” 명시.
    \item Typical provider/renter examples, 주요 cost 요소(storage vs bandwidth vs reliability), 그들의 주된 관심사를 Remark 또는 Table로 정리.
    \item Referee 1이 지적한 “technically incompetent provider” 문제도 함께 보완.
\end{itemize}

\vspace{1em}
% ===============================================================

\noindent \textbf{Authors' Response:} \textcolor{blue}{[...]} \hl{[...]}

\vspace{1.0em}
% ===============================================================



\begin{quotation}
{\em
\noindent \textbf{Associate Editor:} These assumptions should be explicitly addressed and properly justified. Providing clarity on these issues will not only resolve reviewers’ concerns but also improve the paper’s practical relevance and robustness. 
}
\end{quotation}

\noindent \textbf{Comment Analysis (코멘트 분석)}  
- AE는 앞서 제기된 (1a)–(1c) 모든 세부 사항—즉,  
  (i) 렌터가 redundancy-adjusted 용량에 대해 과금되는 논리,  
  (ii) 공급자가 reserved vs allocated capacity 기준으로 보상받는 문제,  
  (iii) uptime 및 redundancy 정책의 실제 운영 방식과 참여자 특성—  
  이 세 가지를 논문 내에서 명시적으로 언급하고 정당화해야 한다고 강조한다.  
- 이는 단순한 기술적 수정이 아니라, \textbf{논문의 현실성과 신뢰성을 높이는 핵심 보완 요구}라는 의미이다.

\vspace{0.8em}

\noindent \textbf{Response Strategy (대응 방향)}  
- 우리의 대응 방향은 다음으로 요약된다:
  \begin{itemize}
      \item 각 가정(assumption)의 산업적 근거, 경제적 의미, 기존 문헌과의 연결을 명확히 설명.
      \item 어떤 가정이 결과의 구조에 본질적인지 vs. 단순화를 위해 사용된 것인지를 구분.
      \item 필요한 경우 robustness check나 numerical test를 추가하여, 주된 결과가 특정 가정에만 의존하지 않음을 보여줌.
  \end{itemize}

\vspace{0.8em}

\noindent \textbf{Response (영문)}  
We fully agree with the AE’s concluding observation. In the revised manuscript, we will explicitly state and justify all key context-specific assumptions discussed in Points~1(a)–1(c). In particular, we will clarify:
\begin{itemize}
    \item why renters are effectively charged for reliable (redundancy-backed) storage rather than for raw capacity alone,  
    \item how providers are compensated in practice based on utilized capacity and reliability performance rather than mere reserved capacity, and  
    \item how uptime and redundancy policies are chosen and enforced in real platforms, including the roles and incentive structures of typical providers and renters.
\end{itemize}
Providing this clarification will strengthen the practical relevance and robustness of our model, and directly address both reviewers’ concerns. We appreciate the AE’s guidance on this matter.

\vspace{0.8em}

\noindent \textbf{Manuscript Changes (본문 수정 계획)}
\begin{itemize}
    \item Section 2–3에 “Assumptions and Industry Practices”라는 짧은 subsection 또는 Remark 추가.  
    \item 위 1(a)–1(c)에서 언급된 가정의 경제적 정당성, 현실적 사례, 참고 문헌 등을 명시.  
    \item Footnote 또는 Appendix에 참고 사례(AWS, Google Cloud, Storj, Filecoin 등) 및 페널티 구조 요약.
\end{itemize}

\vspace{1.0em}
% ===============================================================

\noindent \textbf{Authors' Response:} \textcolor{blue}{[...]} \hl{[...]}

\vspace{1.0em}
% ===============================================================

\begin{quotation}
{\em
\noindent \textbf{Associate Editor (2. Rigor of the Model):} The rigor of the model is an essential aspect that requires further scrutiny and enhancement to address reviewer concerns and ensure the framework’s robustness. Below, I summarize the key points and provide suggestions for improvement: 
}
\end{quotation}

%-----------------------------
\noindent \textbf{Comment Analysis (코멘트 분석)}
%-----------------------------
\begin{itemize}
    \item AE는 본 논문의 모델에서 “수학적 완전성(missing definitions, lemma conditions, equilibrium structure)”과 “이론적 정확성(가정의 일관성, 결과 도출 방식)”이 일부 부족하다고 판단하고 있다.
    \item 이는 특히 Referee 2가 강조한:
    \begin{itemize}
        \item 공급자 참여 모델의 균형 구조 설명 부족  
        \item Market-clearing price 존재성/유일성 정당화 미흡  
        \item Lemma 1–3, Theorem 1 등의 전제 조건, notation 정의 누락  
        \item $q$ 값 (hybrid pricing)이 연속적으로 연결되는지, 혹은 discrete design인지를 명확히 구분해야 한다는 지적  
    \end{itemize}
    \item AE는 단순히 “Proof를 Appendix에 넣어라”가 아니라, \textbf{모델 자체가 logically closed-form + self-contained}될 수 있도록 구조 강화가 필요하다고 본다.
\end{itemize}

\vspace{0.8em}

%-----------------------------
\noindent \textbf{Response Strategy (대응 방향)}
%-----------------------------
\begin{itemize}
    \item 전체 model rigor를 강화하기 위해 아래와 같은 방향을 따른다:
    \begin{enumerate}
        \item \textbf{Equilibrium 정의 명확화}:  
        Provider 참여 임계값 → Fixed-point mapping → Market-clearing 해 설계 → 이를 Lemma 1 직전에 명시 (notation 포함).
        \item \textbf{Lemma/Theorem 재정의 및 가정 명시}:  
        각 Lemma/Theorem 앞에 “Given $(p_s,p_b,q)$ and $(v_s,v_b,...)$”처럼 전제조건을 명확하게 기술.  
        특히 Reviewer 1이 언급한 Lemma 1, Lemma 2의 ambiguity 수정.
        \item \textbf{Market-clearing 다차원성(price pair)} 문제 보완:  
        “Two-part or hybrid has 2-dimensional price space → multiple clearing pairs 가능” (Referee 2) → 이 부분은 추가 정리 및 profit-maximizing pair 설명.
        \item \textbf{$q$의 연속 variable 설명 vs. design variable}:  
        단순 continuous 변화가 아니라, $q \le 1$ (low regime) vs $q>1$ (high regime)로 나누는 design logic 명확히 정리.
        \item \textbf{불필요한 복잡 증명 본문화는 피하되, 핵심 논리 흐름을 본문에 concise하게 포함}하고, Appendix에 full proof 제공.
    \end{enumerate}
\end{itemize}

\vspace{0.8em}

%-----------------------------
\noindent \textbf{Response (영문)}
%-----------------------------
Thank you for emphasizing the importance of rigor in our modeling framework. We agree that strengthening the internal logic and clarity of the equilibrium formulation, as well as the presentation of Lemmas and Theorems, will greatly improve the paper’s robustness.

In the revised manuscript, we will:
\begin{itemize}
    \item clearly define the participation equilibrium of providers and the market-clearing condition as a fixed-point relation between supply and demand;
    \item restate each Lemma and Theorem with explicit assumptions and parameter conditions (e.g., which variables are taken as given, and which are endogenous);
    \item explain how the profit-maximizing prices are selected among the potentially many market-clearing price pairs in two-dimensional pricing schemes;
    \item clarify that the hybrid pricing parameter $q$ is treated as a design variable and that our comparison focuses on low-$q$ (equivalent to two-part) and high-$q$ (subscription-like) regimes, rather than a fully continuous price path;
    \item move key parts of the equilibrium intuition and redundancy–profit linkage from the Appendix to the main text in a concise and readable form.
\end{itemize}

These revisions will make the analytical structure more transparent, help address both reviewers’ concerns, and reinforce the theoretical soundness of the model.

\vspace{0.8em}

%-----------------------------
\noindent \textbf{Manuscript Changes (본문 수정 계획)}
%-----------------------------
\begin{itemize}
    \item Section 3–4 도입부에 Equilibrium Definition + Market-Clearing Condition 명시 (notation 포함).  
    \item Lemma 1–3, Theorem 1 전제조건/notation 재정리 및 문장 명확화.  
    \item 2D pricing $(p_s,p_b)$에서 multiple clearing pair 가능성을 간단히 언급하고, platform이 선택하는 기준(surplus or profit maximization)을 본문에 포함.  
    \item $q$는 continuous adjustment parameter가 아니라 design policy parameter임을 명시하고, low-$q$/high-$q$로 구분.  
    \item Appendix는 유지하되, 핵심 직관 및 증명 요약을 본문에 포함.
\end{itemize}

\vspace{1.2em}
% ===============================================================

\noindent \textbf{Authors' Response:} \textcolor{blue}{[...]} \hl{[...]}

\vspace{1.0em}
% ===============================================================

\begin{quotation}
{\em
\noindent \textcolor{red}{\textbf{Associate Editor (Point \#2a: Part 1 of 2): Exogenous vs. Endogenized Variables.}} The justification for which parameters are treated as endogenous versus exogenous remains 
insufficient. Specifically: 
\begin{itemize}
    \item Since the focus of this paper is on pricing schemes employed in P2P platforms, it would make more sense to endogenize not only the commission rate (as R1 noted) but also the threshold for free bandwidth allowance (as R2 noted). It is unclear why the paper focuses on cases where these pricing parameters are exogenous, especially since the threshold for free bandwidth allowance is usually determined by platforms.
\end{itemize}
}
\end{quotation}

\noindent \textbf{Authors' Response:} \textcolor{red}{We appreciate the review team's constructive comments on endogenous vs. exogenous parameters. Concerning commission rates $\alpha$, R1 asked us to ``discuss why it is assumed to be exogenous'' (R1's \textbf{Point \#6}). Also, both reviewers (R1's \textbf{Point \#11} and R2's \textbf{Point \#8}) raised a concern about exogenous free bandwidth allowance. We have taken these points seriously and addressed them very carefully as described below.}

\textcolor{red}{First, R1's \textbf{Point \#6} asked us to add the justification of exogenous commission rates when they are introduced. In our earlier version, such explanations were only included in \textbf{Section 6}, where we extend our main model by endogenizing commission rates. We admit that our previous manuscript did not provide sufficient explanations in \textbf{Section 3.1}, where we introduce assumptions and their justifications, which can make readers concerned about the external validity of our main model.}

\textcolor{red}{Following R1's suggestion, we added our justification about using exogenous commission rates in the main model as follows:}

\begin{quotation}
{\em
\noindent \textcolor{red}{``Given that public controversies over commission rates often raise fairness and public-sentiment concerns beyond profitability (Gartenberg, 2021; Scheiber, 2015), platforms may be unable to implement profit-maximizing commission rates.''}
} \hl{(page XX in Section 3.1)}
\end{quotation}

\textcolor{red}{Second, we admit that our previous manuscript confused the review team. As you mentioned, firms can determine their own free bandwidth allowance in practice but in a sense that they can adopt either subscription pricing or two-part tariffs on their own in real-world business. Namely, it needs to be seen as a method to represent hypothetically determined pricing schemes, and our paper aims to compare outcomes across these hypothetical pricing schemes.} 

\textcolor{red}{To be specific, free bandwidth allowance is a \textit{design parameter} introduced to quantitatively express hybrid pricing in addition to the two extremes---i.e., two-part tariff and subscription pricing---instead of using it as a continuous policy variable that the firm can manipulate in our model. Specifically, a small allowance ($0 < q \le 1$) represents a capped or two-part–like design, while a large allowance ($q > 1$) corresponds to a subscription-like design. Our goal is to compare the performance of these distinct pricing regimes rather than to study infinitesimal changes in $q$.}  

\textcolor{red}{To avoid the potential confusion related to the role of $q$, we have revised the sentence preceding Lemma~2 to explicitly state the comparison across pricing schemes rather than the monotonicity within hybrid pricing, although this monotonicity is technically correct. By doing so, we clarify that two-part and pure subscription can be viewed as special, discrete designs ($q=0$ and sufficiently large $q$, respectively), distinguished from the points of a continuous axis within hybrid pricing.} 

\textcolor{red}{In the new manuscript, we have updated the text to clarify this aspect as follows. First, we have included the overview of pricing schemes and design parameter $q$ in \textbf{Section 3.1} as:} 
\begin{quotation}
{\em
\noindent \textcolor{red}{``Cloud service providers commonly rely on two canonical nonlinear pricing schemes: two-part tariffs and flat-rate subscriptions \hl{[citations]}. Under a two-part tariff, the provider charges a fixed access fee in combination with a per-unit usage price, so that users internalize their marginal consumption through the usage fee while sharing common infrastructure costs through the fixed fee. In contrast, a flat-rate subscription specifies a single periodic payment that grants access to a prescribed level of service, effectively setting the marginal price of incremental usage to zero within the contracted scope. Prior analytical studies in centralized cloud markets have compared these two schemes in terms of provider profit, user surplus, and capacity utilization, highlighting that two-part tariffs can better align usage with costs when demand is heterogeneous, whereas subscriptions can be attractive for demand stimulation and risk reduction from the user's perspective \hl{[citations]}.}

\textcolor{red}{Motivated by both practice and this prior theoretical work, we consider a hybrid pricing scheme that is observed in centralized cloud offerings and is increasingly adopted in decentralized cloud environments \hl{[citations]}. In hybrid pricing, the provider specifies a fixed fee together with a free bandwidth allowance and applies a positive usage price only to consumption that exceeds this allowance. To capture and systematically analyze this design flexibility, we introduce a design parameter $q$ that determines the firm's pricing scheme by quantifying the free bandwidth allowance embedded in the contract. By doing so, we could compare the three pricing schemes in a unified analytical framework.''}
} \hl{(page XX)}
\end{quotation}

\textcolor{red}{Second, we have revised the last sentence of \textbf{Lemma 2} to avoid the confusion as:} 
\begin{quotation}
{\em
\noindent \textcolor{red}{``Also, pricing schemes with higher bandwidth allowance have higher optimal fees; that is, $p_s^T < p_s^{Hl} < p_s^{Hh} < p_s^S$ and $p_b^T \le p_b^{Hl} \le p_b^{Hh} \le p_b^S$.''}
} \hl{(page XX)}
\end{quotation}

\noindent \textbf{References:}

\vspace{0.2cm}
\noindent Gartenberg C (2021) Google will reduce play store cut to 15 percent for a developer’s first \$1m in annual revenue. \textit{The Verge}. \url{https://www.theverge.com/2021/3/16/22333777/
google-play-store-fee-reduction-developers-1-million-dollars}.

\vspace{0.2cm}
\noindent Scheiber N (2015) Growth in the ‘gig economy’ fuels work force anxieties. \textit{The New York Times}. \url{https://www.nytimes.com/2015/07/13/business/
rising-economic-insecurity-tied-to-decades-long-trend-in-employment-practices.
html}.

\vspace{1.0em}

% ===============================================================

\begin{quotation}
{\em
\noindent \textcolor{red}{\textbf{Associate Editor (Point \#2a: Part 2 of 2): Exogenous vs. Endogenized Variables.}} 
\begin{itemize}
    \item On the other hand, if the authors wanted to simplify their model and analysis, it might be more appropriate to treat the redundancy parameters as exogenous, rather than the above pricing-related parameters. In practice, redundancy parameters are determined based on both financial considerations and engineering constraints, the latter of which fall outside the scope of this paper and are less relevant to its focus on pricing schemes.  
\end{itemize}
}
\end{quotation}

\noindent \textbf{Authors' Response:} \textcolor{red}{Thank you for the thoughtful comment. Following your suggestion, our revised model focuses on pricing-related parameters rather than redundancy parameters. To be specific, we postulate that algorithmic decisions are exogenous in the main model, and endogenous algorithms are separated from the main analysis and provided as a part of extensions in \textbf{Section 6.1}. As a result, the theorem order has changed, and to focus more on pricing and meet the page limit, we reduced technical details of erasure coding from the manuscript. We believe this revision helped us better structure our paper to discuss direct effects of pricing more concisely.}

\vspace{1.0em}
% ===============================================================

\begin{quotation}
{\em
\noindent \textcolor{red}{\textbf{[영재 작업 후 진행 가능]}} \textbf{Associate Editor (Point \#2b): Renter’s Utility Function.} The assumption that renter utility is solely proportional to bandwidth usage, $U_i=\lambda_i u_i$, is restrictive. A more general utility function should include a term that captures the renter’s utility derived from storage itself, rather than bandwidth usage to access the storage. For example, $U_i=K_i + \lambda_i u_i$. This generalization can better capture the renters’ diverse preferences. For instance, some renters derive high utility from secure and reliable storage, while they derive low utility from the flexibility in accessing to the stored data (in this case, $K_i$ is high while $u_i$ is low). Conversely, other renters may prioritize accessibility and derive high utility from bandwidth usage while placing less emphasis on the storage itself (in this case, $K_i$ is low while $u_i$ is high). 

Additionally, R1 also raises a valid point regarding renters’ potential concerns about unavailability. Even though the probability of unavailability might be very small, its cost implications can be substantial for the renters. The question is: Should a loss of utility that captures the consequences of denied access to stored data be included in the renters’ utility function? Furthermore, since this probability depends on the setting of the two parameters of the redundancy algorithm, it would be prudent to incorporate availability constraints into the optimization of key parameters, as R1 noted. 
}
\end{quotation}

% ===============================================================
\noindent \textbf{AE (Point \#2b) — Addendum from Authors: Common-$K$ Generalization}

\vspace{0.6em}
\noindent \textbf{Comment Analysis (코멘트 분석)}  
개별 $K_i$까지 식별·추정하는 것은 작업 범위상 부담이 크므로, \emph{모든 렌터에게 동일한 저장 가치} $K$를 두는 최소 확장으로 대응하고자 한다.  
이는 “저장 그 자체에서 얻는 효용”을 한 항으로 포착하면서도, 분석 복잡도를 크게 늘리지 않고 정리 가능한 실용적 타협이다.  
또한 다양한 $K$ 시나리오(작음/보통/큼)에 대한 강건성(robustness)을 수치적으로 확인해, 주요 비교 결과가 유지됨을 보여주려 한다.

\vspace{0.6em}
\noindent \textbf{Response Strategy (답변 방향)}  
\begin{itemize}
  \item \textbf{효용 형태(공통 $K$):} 렌터 효용을 $U = K + \lambda u$로 확장(여기서 $K$는 모든 렌터에게 공통).  
  \item \textbf{임계값의 단순 이동:} $K$는 모든 렌터에 동일하게 더해지므로, 채택 임계값이 선형적으로 이동한다.  
  예를 들어, 하이브리드/투파트에서 사용량 $u>0$일 때, 채택 조건은
  \[
  K + \lambda u - \big(p_s + p_b u\big) \ge 0
  \ \Rightarrow\ 
  \lambda \ \ge\ \frac{p_s + p_b u - K}{u}.
  \]
  즉, $K$는 \emph{유효 진입비용을 $K$만큼 낮추는} 효과가 있다.  
  \item \textbf{비교정리 불변성:} $K$가 공통 상수이므로, 동일 $K$ 하에서 정책 간 비교(투파트 vs 하이브리드 vs 구독)의 순서와 market-clearing 최적화 구조는 변하지 않는다.  
  \item \textbf{강건성 보고:} $K\in\{K_{\text{low}},K_{\text{mid}},K_{\text{high}}\}$에 대해 수치 비교(이익/후생 순위, 채택률, 균형가격)를 표·그림으로 제시.
\end{itemize}

\vspace{0.6em}
\noindent \textbf{Response (영문)}  
We appreciate the AE’s suggestion to allow renters to derive value from storage itself. To keep the analysis tractable while capturing this dimension, we adopt a common storage value $K$ for all renters and write the utility as
\[
U \;=\; K + \lambda\,u.
\]
With $K$ common across renters, the adoption condition under any scheme becomes
\[
K + \lambda u \;\ge\; p_s + p_b u,
\qquad\text{so}\qquad
\lambda \;\ge\; \frac{p_s + p_b u - K}{u}\quad (u>0).
\]
Hence $K$ simply shifts the participation threshold by $-K/u$, which is equivalent to reducing the effective entry cost by $K$. Because this shift applies uniformly across pricing schemes, our qualitative comparisons (profit and welfare rankings under market clearing) remain unchanged. 

In the revision, we will report numerical robustness across three representative values $K\in\{K_{\text{low}},K_{\text{mid}},K_{\text{high}}\}$ and show that the main insights are preserved. This common-$K$ extension addresses the AE’s concern about storage-driven value while avoiding unnecessary analytical complexity from fully heterogeneous $K_i$.

\vspace{0.6em}
\noindent \textbf{Manuscript Changes (본문 수정 사항)}  
\begin{itemize}
  \item \textbf{Section 3 (Renter Utility) Remark:} “We allow a common storage value $K$ so that $U=K+\lambda u$. This shifts adoption thresholds by $-K/u$ but leaves the comparative results intact.” 간결 수식 포함.
  \item \textbf{Section 5 (Numerical):} $K\in\{K_{\text{low}},K_{\text{mid}},K_{\text{high}}\}$에 대한 균형·이익·후생 강건성 표/그림 추가.
  \item \textbf{Discussion:} 공통 $K$가 \emph{유효 진입비용 감소}로 해석됨을 두세 문장으로 설명.
\end{itemize}

% ===============================================================

\noindent \textbf{Authors' Response:} \textcolor{blue}{[...]} \hl{[...]}

\vspace{1.0em}
% ===============================================================

\begin{quotation}
{\em
\noindent \textbf{Associate Editor (Point \#2c): Provider’s Participation and Rational Expectations.} R2 expresses concerns about whether the formulation of the providers’ profit functions have considered providers’ rational expectations. Specifically, each individual provider’s participation decision is based on its expectations regarding the number of participating renters and providers. However, the number of participating providers is itself derived from individual provider’s participation decisions, forming a rational expectation. While the proof provided in the appendix seems to have incorporated this consideration, the paper should clearly explain how the provider profit is calculated based on the expectation of the number of participating providers and how this expectation is aligned with the outcome derived from individual providers’ decisions. This would strengthen the model’s clarity and rigor. 
}
\end{quotation}

%-----------------------------
\noindent \textbf{Comment Analysis (코멘트 분석)}
%-----------------------------
\begin{itemize}
    \item 본 코멘트는 “공급자가 어떻게 시장 참여자 수를 예상(rational expectation)하고, 이 예상이 실제 균형 결과와 어떻게 일치하는지를 설명하라”는 요구.
    \item 실제로 Appendix에서는 fixed-point 형태로 공급자 참여비율을 유도했지만, 본문에는 \textbf{“Equilibrium = 기대-participation과 실제 결과가 일치하는 상태”}라는 설명이 부족했다는 지적.
    \item AE의 목적: 
    \begin{itemize}
        \item 공급자의 기대 수입 계산 시, “예상 참여 공급자 수 $\hat{n}_s$와 참여 렌터 수 $\hat{n}_r$에 기반한다”는 점을 서술하라.
        \item 이후 실제 참여비율이 이 예상과 동시에 일치하는 fixed-point임을 명확하게 적시하라.
    \end{itemize}
\end{itemize}

\vspace{0.8em}

%-----------------------------
\noindent \textbf{Response (영문)}
%-----------------------------
Thank you for this important comment. We agree that the equilibrium logic of provider participation should be more explicitly described in the main text.

In our model, an individual provider of type $\theta$ decides whether to participate based on the \emph{expected} number of participating providers $\hat{n}_s$ and renters $\hat{n}_r$. Given these expectations, the provider computes its expected profit as
\[
\pi(\theta; \hat{n}_s,\hat{n}_r)= 
\text{(expected storage revenue + bandwidth revenue)} 
- \text{(operating cost)}.
\]
A provider participates if $\pi(\theta;\hat{n}_s,\hat{n}_r)\ge 0$. Since provider types are distributed over $[0,1]$, this yields a cutoff type $\theta^*(\hat{n}_s,\hat{n}_r)$ such that all providers with $\theta \le \theta^*$ enter. Hence the \emph{actual} number of participating providers is
\[
n_s = N_s \cdot F(\theta^*(\hat{n}_s,\hat{n}_r)).
\]

An equilibrium is a fixed point where expectations are consistent with outcomes:
\[
(\hat{n}_s,\hat{n}_r) = (n_s,n_r).
\]
This is the standard rational-expectations or self-confirming equilibrium condition. The proof in Appendix A (Lemma 1) characterizes $\theta^*$ and shows that such a fixed point exists and is unique under the monotonicity assumptions on providers’ profit.

In the revision, we will explicitly introduce this fixed-point interpretation in the main text before Lemma 1 and clarify how provider profits are computed based on expectations that are consistent in equilibrium.

\vspace{0.8em}

%-----------------------------
\noindent \textbf{Manuscript Changes (본문 수정 계획)}
%-----------------------------
\begin{itemize}
    \item Section 3.2 “Provider Participation” 도입부에 아래 내용을 추가:
    \begin{itemize}
        \item 공급자가 $(\hat{n}_s,\hat{n}_r)$를 예상하고 기대수익 $\pi(\theta;\hat{n}_s,\hat{n}_r)$를 계산한다.
        \item 기대 cutoff $\theta^*$에 따라 실제 참여 공급자 수는 $n_s = N_s \cdot F(\theta^*)$가 된다.
        \item 균형 정의: $(\hat{n}_s,\hat{n}_r) = (n_s,n_r)$를 만족하는 fixed point.
    \end{itemize}
    \item Lemma 1 앞에 “This lemma characterizes the fixed point of providers’ rational expectations”라는 문장 추가.
    \item Appendix A에 있는 증명이 위와 같은 fixed-point 기반임을 간단히 앞서 언급.
\end{itemize}

\vspace{1em}
% ===============================================================

\noindent \textbf{Authors' Response:} \textcolor{blue}{[...]} \hl{[...]}

\vspace{1.0em}
% ===============================================================

\begin{quotation}
{\em
\noindent \textbf{Associate Editor (3. Expositional Suggestions):} Both reviewers have raised many good suggestions for improving the exposition of the paper. Additionally, I would like to emphasize several relatively major points: 
}
\end{quotation}

%-----------------------------
\noindent \textbf{Comment Analysis (코멘트 분석)}
%-----------------------------
\begin{itemize}
    \item AE는 모델이나 결과의 내용 자체보다는 \textbf{독자가 논문을 읽고 이해하기 쉽도록, 글의 구조와 설명 방식을 개선해 달라}는 요청을 하고 있다.
    \item Referee 1과 2 모두 이미 많은 “서술 명확성 + notation 정의 + lemma 서술 방식”에 대해 지적했고, AE 역시 그 중요성을 강조.
    \item 이 “Point 3”는 세부 항목(3a, 3b, …)으로 이어질 예정이며, \textbf{각 항목마다 Introduction, Notation, Lemma wording, Figure explanation 등 특정 수정 포인트가 제시}될 가능성이 있다.
\end{itemize}

\vspace{0.8em}

%-----------------------------
\noindent \textbf{Response Strategy (대응 방향)}
%-----------------------------
\begin{itemize}
    \item 우리는 exposition 강화를 위해 아래와 같은 \textbf{일관된 수정 원칙}을 적용할 계획:
    \begin{enumerate}
        \item \textbf{논문의 Roadmap 명확화:}  
        Introduction 말미와 Section 2 도입부에 각 Section의 역할과 핵심 질문을 요약해 서술.
        \item \textbf{Notation Table 추가:}  
        주요 변수(가격, 수요/공급, redundancy, utility 파라미터 등)에 대한 요약 표를 Appendix 또는 Section 2에 삽입.
        \item \textbf{Lemma/Theorem 명확화:}  
        - 가정(“Given…”)과 결과(“Then…”)를 분리  
        - 포함된 변수들이 endo/exo 인지 명확히 구분  
        - reviewer 1이 혼란스러워한 표현(“belong to market-clearing prices” 등) 교체
        \item \textbf{중요 도표(Figure/Table) 보강:}  
        Figure 앞에 “economic intuition and takeaway”를 2–3문장 추가.
        \item \textbf{Appendix와 본문 연결 강화:}  
        본문에서 “Proof intuition is as follows…” 식으로 핵심 직관을 제공하고, 자세한 수학은 Appendix 참조.
    \end{enumerate}
\end{itemize}

\vspace{0.8em}

%-----------------------------
\noindent \textbf{Response (영문)}
%-----------------------------
We appreciate the AE’s emphasis on improving the clarity and readability of the manuscript. In addition to addressing all detailed comments from both referees, we will implement several global revisions to enhance exposition:

\begin{itemize}
    \item We will provide a clearer roadmap in the Introduction and at the start of Section~2, summarizing each section’s role and the main questions it addresses.
    \item A concise notation table will be added to help readers keep track of key variables and parameters.
    \item Each Lemma and Theorem will be reworded to explicitly separate assumptions from conclusions and to clearly indicate which variables are treated as exogenous or endogenous.
    \item Before major figures and numerical results, we will add brief intuitive explanations that highlight the economic meaning of each comparison.
    \item While full technical proofs will remain in the Appendix, we will include short proof sketches or key intuition in the main text to guide the reader.
\end{itemize}

These changes are aimed at making the paper more accessible to MSOM’s broad audience while maintaining analytical rigor.

\vspace{0.8em}

%-----------------------------
\noindent \textbf{Manuscript Changes (본문 수정 계획)}
%-----------------------------
\begin{itemize}
    \item Introduction 마지막 문단: “Roadmap of the paper” 추가.
    \item Section 2 or Appendix A: 주요 notation 정리 표 추가.
    \item Lemma 1–3 및 Theorem 1–3: “Given … Then …” 형태로 재작성.
    \item Figure 3–5 등 핵심 그래프 앞에 2–3문장의 경제적 해석 추가.
    \item Appendix와 본문 연결: “Proof sketch” 또는 “intuition” 문단 추가.
\end{itemize}

\vspace{1em}
% ===============================================================

\noindent \textbf{Authors' Response:} \textcolor{blue}{[...]} \hl{[...]}

\vspace{1.0em}
% ===============================================================

\begin{quotation}
{\em
\noindent \textbf{Associate Editor (Point \#3a):} I don’t think “first-best” pricing is used appropriately in the context of this paper. Typically, “first-best” pricing is the set of pricing decisions that maximize the total profit or minimize the total cost of the entire system (in this context, the total surplus of the providers, the renters, and the platform). But this is different from the definition of the so-called “first-best” pricing in the paper. I suggest the authors consider alternative name for this pricing outcome. 
}
\end{quotation}

%-----------------------------
\noindent \textbf{Comment Analysis (코멘트 분석)}
%-----------------------------
\begin{itemize}
    \item AE는 “first-best”라는 용어가 전통적인 의미(시스템 전체 후생 극대화)와 다르게 사용되었다고 지적.
    \item 우리 논문에서는 “공급이 충분할 때 플랫폼 이익을 극대화하는 pricing”을 first-best라 불렀는데, 이는 학문적 용어 정의와 충돌.
    \item 따라서 용어를 바꾸고, 논문 전반에서 통일된 terminology를 사용해야 함.
\end{itemize}

\vspace{0.8em}

%-----------------------------
\noindent \textbf{Response Strategy (대응 방향)}
%-----------------------------
\begin{itemize}
    \item first-best 용어는 삭제하고, 다음 중 하나로 대체:
    \begin{itemize}
        \item \textbf{“Unconstrained profit-maximizing pricing”}  
        \item \textbf{“Abundant-supply benchmark pricing”}  
        \item \textbf{“Profit-maximizing pricing under no supply constraint”}
    \end{itemize}
    \item 이를 통해 “시장균형 필요 없이 플랫폼이 단독 최적화할 수 있는 이익 극대화 상태”라는 의미를 표현.
    \item total surplus를 극대화하는 pricing은 별도로 “social welfare-optimal pricing”으로 구분.
\end{itemize}

\vspace{0.8em}

%-----------------------------
\noindent \textbf{Response (영문)}
%-----------------------------
Thank you for pointing this out. We agree that our use of the term “first-best” may be misleading from a welfare-economics perspective. In our paper, the so-called “first-best” pricing referred to the profit-maximizing prices chosen by the platform when the provider supply is sufficiently abundant so that supply does not constrain adoption. This is not equivalent to the classical definition of first-best pricing, which maximizes total surplus (platform + providers + renters).

To avoid confusion, we will replace “first-best pricing” with a more precise term, such as:
\begin{quote}
\emph{“unconstrained profit-maximizing pricing”} or \emph{“benchmark pricing under abundant supply.”}
\end{quote}
We will also use “welfare-optimal pricing” or “social-surplus-maximizing pricing” when referring to outcomes that maximize total system surplus.

We will revise the manuscript accordingly in all relevant sections (abstract, introduction, Section~4, and Theorem statements).

\vspace{0.8em}

%-----------------------------
\noindent \textbf{Manuscript Changes (본문 수정 사항)}
%-----------------------------
\begin{itemize}
    \item Abstract 및 Section 1–2: “first-best pricing” 용어 삭제 및 “unconstrained profit-maximizing pricing”으로 변경.
    \item Figure/표 제목과 Theorem 1 관련 문구 모두 통일된 terminology로 수정.
    \item welfare 비교 시는 “socially optimal pricing” 등으로 명확히 구분.
\end{itemize}

\vspace{1em}
% ===============================================================

\noindent \textbf{Authors' Response:} \textcolor{blue}{[...] renamed these pricing contexts as follows:}

\textbf{\textit{Unconstrained Pricing:}} \hl{[...]}

\textbf{\textit{Constrained Pricing:}} \hl{[...]}

\hl{[...]}

\vspace{1.0em}
% ===============================================================


\begin{quotation}
{\em
\noindent \textbf{Associate Editor (Point \#3b):} Regarding results of the “market clearing” pricing, it would make more sense to state the results based on the ratio of the potential providers to the potential renters, $n_p/n_r$, rather than based on solely $n_p$ (e.g., Lemma 3, Figure 5, Theorem 3, and Theorem 4).  
}
\end{quotation}

%-----------------------------
\noindent \textbf{Comment Analysis (코멘트 분석)}
%-----------------------------
\begin{itemize}
    \item AE는 시장 균형 결과를 해석할 때, 공급자 수($n_p$) 단독 기준보다 \textbf{공급 대비 수요 비율 $n_p/n_r$}가 더 적절하다고 지적.
    \item 이는 시장의 “상대적 공급 풍부도”를 더 정확히 나타내며, Lemma 3, Figure 5, Theorem 3–4의 설명 방식과 매개변수 정의도 이에 맞춰 조정될 필요가 있음.
    \item 특히 시장-clearing 조건 자체가 “공급자 참여 수 = 수요자 참여 수”라는 형태이므로, ratio를 활용하면 해석이 직관적이고 일반화 가능.
\end{itemize}

\vspace{0.8em}

%-----------------------------
\noindent \textbf{Response (영문)}
%-----------------------------
We appreciate this helpful suggestion. Indeed, what matters for market clearing is not the absolute number of potential providers $n_p$ alone, but the \emph{relative availability of supply} compared to demand. In equilibrium, the platform needs the number of active providers to match the number of renters who adopt the platform. Hence, expressing results in terms of the ratio 
\[
\frac{n_p}{n_r}
\]
offers a more natural and scalable way to characterize the market-clearing regime.

In the revised manuscript, we will:
\begin{itemize}
    \item restate Lemma 3, Theorem 3, and Theorem 4 using the ratio $n_p/n_r$ as the key parameter for distinguishing abundant- vs. scarce-supply regimes;
    \item modify Figure~5 and the associated discussion to show platform profit and surplus as functions of $n_p/n_r$ rather than $n_p$ alone;
    \item clarify in the text that market clearing occurs when the equilibrium number of participating providers equals the number of renters, and that the feasibility of this condition depends on $n_p/n_r$.
\end{itemize}

This change improves interpretability and makes the results less sensitive to the absolute scale of the market.

\vspace{0.8em}

%-----------------------------
\noindent \textbf{Manuscript Changes (본문 수정 계획)}
%-----------------------------
\begin{itemize}
    \item Lemma 3: “The platform’s optimal service fees belong to … when $n_p$ is large” → revised to “when the supply-to-demand ratio $n_p/n_r$ exceeds a threshold $\tau(\cdot)$”.
    \item Theorem 3–4: 모든 조건 및 결과를 $n_p/n_r$ 기반으로 재표현.
    \item Figure 5: x축을 $n_p / n_r$로 변경하고 설명 문구 보완.
    \item Section 5 본문: “provider-abundant regime”을 “high $n_p/n_r$ regime”으로 표현 변경.
\end{itemize}

\vspace{1em}
% ===============================================================

\noindent \textbf{Authors' Response:} \textcolor{blue}{[...]} \hl{[...]}

\vspace{1.0em}
% ===============================================================

\begin{quotation}
{\em
\noindent \textbf{Associate Editor (Point \#3c):} The term “market-clearing” pricing needs to be clearly defined (as noted by R1). Notably, both R1 and R2 argues that the market-clearing prices are not unique. Broadly speaking, all notations and terms need to be defined when they first appear in the paper (as noted by R1). 
}
\end{quotation}

%-----------------------------
\noindent \textbf{Comment Analysis (코멘트 분석)}
%-----------------------------
\begin{itemize}
    \item AE, R1, R2 모두 본 논문의 핵심 용어인 “market-clearing pricing”의 의미가 명확하지 않다고 지적.
    \item 문제의 핵심:
    \begin{itemize}
        \item 시장에서 “공급자 수 = 수요자 수” 조건을 만족하도록 가격을 조정하는 것이 market clearing이라는 의미인데, 본문에서 공식 정의가 없다.
        \item $(p_s, p_b)$와 같은 2차원 가격체계에서는 market-clearing price가 일반적으로 **단일 해가 아니라 해 집합**일 수 있음.
        \item notation 정의 시점이 일관되지 않아 독자들이 혼란을 겪을 수 있다는 지적.
    \end{itemize}
\end{itemize}

\vspace{0.8em}

%-----------------------------
\noindent \textbf{Response (영문)}
%-----------------------------
Thank you for emphasizing this important point. We agree that the definition of “market-clearing pricing” should be explicitly and formally introduced. In the revised manuscript, we will include the following definition early in Section~4:

\begin{quote}
\textbf{Definition (Market-Clearing Pricing).}  
Given a pricing scheme with storage fee $p_s$ and bandwidth fee $p_b$, a price pair $(p_s,p_b)$ is said to be \emph{market-clearing} if the equilibrium number of participating providers equals the equilibrium number of participating renters. That is,
\[
n_s(p_s,p_b) = n_r(p_s,p_b),
\]
where $n_s(\cdot)$ and $n_r(\cdot)$ are determined by provider and renter participation conditions, respectively.
\end{quote}

We will also clarify that, because pricing is two-dimensional (especially in the two-part and hybrid schemes), the set of price pairs satisfying $n_s=n_r$ may not be unique. Therefore, in our analysis, the platform selects from this feasible set the price pair that maximizes its profit (or total surplus, depending on the objective).

Additionally, we will ensure that:
\begin{itemize}
    \item every notation and economic term (e.g., $n_s$, $n_r$, $\rho_m$, “market clearing”, “abundant supply regime”) is defined at its first appearance;
    \item a summary table of key symbols is provided in the Appendix or at the end of Section~2.
\end{itemize}

\vspace{0.8em}

%-----------------------------
\noindent \textbf{Manuscript Changes (본문 수정 계획)}
%-----------------------------
\begin{itemize}
    \item Section 4 초입에 “Definition: Market-Clearing Pricing” 박스로 추가.
    \item Market-clearing 가격이 generally non-unique함을 명시하고, platform은 이 집합 중 profit-maximizing price를 선택한다고 기술.
    \item Lemma 3, Theorem 3–4에서 해당 정의를 참고하도록 문장 정정.
    \item Section 2 또는 Appendix에 Notation Summary Table 추가.
    \item 모든 수식 변수와 용어가 첫 등장 시 정의되도록 전체 문서 점검.
\end{itemize}

\vspace{1em}
% ===============================================================

\noindent \textbf{Authors' Response:} \textcolor{blue}{[...]} 

\textcolor{blue}{As defined in \textbf{Point \#3a}, [...]}

\hl{[...]}

\vspace{1.0em}
% ===============================================================

\begin{quotation}
{\em
\noindent \textbf{Associate Editor (Point \#3d):} It would enhance clarity of the formulations and key derivations if the authors could selectively move some necessary technical details into the main text (as noted by R2). 
}
\end{quotation}

%-----------------------------
\noindent \textbf{Comment Analysis (코멘트 분석)}
%-----------------------------
\begin{itemize}
    \item AE는 “핵심 논리(메커니즘) 대부분이 Appendix에만 있으며, 본문에는 직관 없이 결과만 제시되어 독자가 따라가기 어렵다”고 지적.
    \item R2 또한 “too much of the technical content is hidden in the appendix, making it hard to understand what drives the results”라며 동일 문제를 제기.
    \item 따라서, 모든 증명을 본문으로 옮기라는 뜻은 아니고, \textbf{핵심 아이디어와 직관적인 derivation flow를 본문에 요약소개하고, 세부 수학은 Appendix에 유지}하라는 요구.
\end{itemize}

\vspace{0.8em}

%-----------------------------
\noindent \textbf{Response (영문)}
%-----------------------------
Thank you for this valuable suggestion. We agree that providing key derivation steps and economic intuition in the main text will improve accessibility and help readers better understand the mechanisms behind the results.

In the revision, we will:

\begin{itemize}
    \item Include brief “proof sketches” or derivation outlines for the most critical results (e.g., Lemma~1 on participation thresholds, Lemma~3 on market-clearing prices, Theorem~1 on profit-maximizing pricing).
    \item Highlight the main economic logic — for example, how provider revenue depends on expected bandwidth usage and how this leads to a unique cutoff type.
    \item Keep the detailed and lengthy algebraic derivations in the Appendix but clearly reference them in the main text (e.g., “see Appendix A.2 for full proof”).
    \item Add intuitive explanations before presenting dense mathematical expressions in Sections 4 and 5.
\end{itemize}

\vspace{0.8em}

%-----------------------------
\noindent \textbf{Manuscript Changes (본문 수정 계획)}
%-----------------------------
\begin{itemize}
    \item Section 4 (Participation \& Market Clearing):  
    Add a short explanation of how the cutoff type $\theta^*$ is derived from expected profit and how it forms a fixed-point.
    \item Lemma/Theorem statements:  
    Immediately follow with 2–3 lines of intuition or proof sketch.
    \item Numerical Section:  
    Clarify how the analytical structure connects to simulation results.
    \item Appendix 유지, 단 본문에서 명확히 cross-reference (“detailed proof in Appendix A.1”).
\end{itemize}

\vspace{1em}
% ===============================================================

\noindent \textbf{Authors' Response:} \textcolor{blue}{[...]} 

\hl{[...]}

\vspace{1.0em}
% ===============================================================

\begin{quotation}
{\em
\noindent \textcolor{red}{\textbf{Associate Editor (Point \#3e):}} It is important to expand the literature review to emphasize the key distinctions, particularly compared to the literature on P2P sharing of computing resources (as noted by R2) and the literature on P2P online files sharing. Additionally, since the paper focuses on the pricing schemes of P2P storage, which is a format of cloud services, it would make sense to expand the discussions on recent studies of cloud-based pricing schemes (see, e.g., Li and Kumar 2022, Chen et al. 2023, and the references therein). Last but not least, the subscription pricing plans have been well studied in the broader literature on marketing and information systems, so the authors should also expand the discussion on this body of literature (see, e.g., Balasubramanian et al. 2015, and the references therein). 
}
\end{quotation}

\noindent \textbf{Authors' Response:} \textcolor{red}{Thank you for the thoughtful and constructive comment on our literature review. In this revision, we strengthen our paper's positioning in the literature following the review team's guidance as follows.} 

\textcolor{red}{First, we clarified how P2P cloud sharing is differentiated from traditional P2P file sharing and other computing-resource sharing in the manuscript as shown below:} 

\begin{quotation}
{\em
\noindent \textcolor{red}{``Moreover, P2P storage sharing differs from both traditional P2P file-sharing and other computing models in several important aspects. In typical P2P file-sharing systems (e.g., BitTorrent-type networks), users temporarily exchange copies of existing files, and no peer is required to reserve capacity, retain data, or guarantee uptime after a transfer. Likewise, P2P computing platforms that share CPU or GPU cycles allocate short-lived processing tasks without any long-term obligation to store data or ensure its future availability. By contrast, P2P storage sharing requires providers to commit disk capacity over time, maintain availability, and bear redundancy and retrieval costs so that renters' data remain intact and accessible. These persistent capacity and reliability requirements make storage sharing different from a one-shot exchange of content or computation, and they critically shape the platform's pricing, redundancy, and service design problems studied in this paper.''} \hl{(page XX in Section 2.2)}

\vspace{0.4cm}
\noindent \textcolor{red}{``In addition, in contrast to CPU/GPU sharing, which involves only short-lived computation tasks with no ongoing data responsibility, P2P storage sharing imposes persistent capacity, availability, and redundancy obligations on providers. Our modeling framework enables us to offer unique insights into these services by incorporating these unique aspects of P2P storage sharing, such as providers' subsequent commitments of capacity, joint pricing of storage and bandwidth under heterogeneous costs, and considering redundancy algorithms as an extension.''} \hl{(page XX in Section 2.4)}
}
\end{quotation}

\textcolor{red}{Second, we have expanded the discussions on recent studies of centralized cloud platforms guided by Li and Kumar (2022) and Chen et al. (2023). Specifically, considering that our paper examines the role of pricing decisions in managing individual capacity supply, we introduced the OM literature in centralized cloud as follows:}

\begin{quotation}
{\em
\noindent \textcolor{red}{``Compared with private clouds, which are deployed for the exclusive use of a single organization, public clouds that serve the general public pose greater challenges for capacity management (Li and Kumar, 2022). Reflecting this, a growing operations management (OM) literature examines how to design pricing schemes for public cloud services.''} \hl{(page XX in Section 2.1)}

\vspace{0.4cm}
\noindent \textcolor{red}{``Recent OM research has emphasized that capacity and supply decisions are central in centralized cloud services. Chen et al. (2023) highlighted that cloud providers must plan data center capacity and hardware purchases under long lead times and uncertain demand, so decisions about when and how much to expand are critical. Along these lines, Arbabian et al. (2021) studied how a cloud provider expands capacity when servers come in fixed configurations that bundle CPU and memory, which can create imbalances between what is supplied and what users demand.''} \hl{(page XX in Section 2.1)}
}
\end{quotation}

\textcolor{red}{and then, we have rephrased this study's position regarding the centralized cloud literature as:}

\begin{quotation}
{\em
\noindent \textcolor{red}{``In our setting, the platform does not directly choose data center investments or server configurations; instead, it must use pricing to attract and coordinate capacity from many decentralized providers, making pricing and capacity management inherently intertwined.''} \hl{(page XX in Section 2.1)}
}
\end{quotation}

\textcolor{red}{Third, we have complemented the paragraph describing our paper's relationship with general subscription pricing papers by citing more relevant papers. In doing so, we have clarified how our setup is associated with these papers as below:} 

\begin{quotation}
{\em
\noindent \textcolor{red}{``Balasubramanian et al. (2015) compared selling and pay-per-use contracts for information goods, modeling usage as consumption of an information service with negligible marginal cost, and showed that pay-per-use can yield higher profits than selling for a monopolist. Our setting is also related to Essegaier et al. (2002), who compared flat-rate, usage-based, and two-part-tariff pricing for centralized access services under capacity constraints and heterogeneous usage rates.''} \hl{(page XX in Section 2.1)}
}
\end{quotation}

\noindent \textbf{References:}

\vspace{0.2cm}
\noindent Arbabian, ME, Chen, S, \& Moinzadeh, K (2021) Capacity expansions with bundled supplies of attributes: An application to server procurement in cloud computing. \textit{Manufacturing \& Service Operations Management} 23(1):191-209.

\vspace{0.2cm}
\noindent Balasubramanian S, Bhattacharya S, Krishnan VV (2015) Pricing information goods: A strategic analysis of the selling and pay-per-use mechanisms. \textit{Marketing Science} 34(2):218–234. 

\vspace{0.2cm}
\noindent Chen S, Moinzadeh K, Song JS, Zhong Y (2023). Cloud computing value chains: Research from the operations management perspective. \textit{Manufacturing \& Service Operations Management} 25(4):1338-56. 

\vspace{0.2cm}
\noindent Essegaier, S, Gupta, S, \& Zhang, ZJ (2002) Pricing access services. \textit{Marketing Science} 21(2):139-159.

\vspace{0.2cm}
\noindent Li B, Kumar S (2022). Managing software‐as‐a‐service: Pricing and operations. \textit{Production and Operations Management} 31(6):2588-608. 

\vspace{1.0em}
% ===============================================================

\begin{quotation}
{\em
\noindent \textbf{Associate Editor (Summary):} Based on the detailed feedback and recommendations provided by the two reviewers, as well as my own reading of the paper, I recommend a major revision. I urge the authors to approach the reviewers' comments with the utmost care and rigor. The concerns raised are substantial and warrant thoughtful consideration to enhance the quality and impact of the paper. I hope the authors find our reviews helpful in refining their work. I wish them the best of luck with their revisions and look forward to seeing how the paper evolves.\\ 

\noindent References:

\vspace{0.2cm}
\noindent Balasubramanian S, Bhattacharya S, Krishnan VV (2015) Pricing information goods: A strategic analysis of the selling and pay-per-use mechanisms. \textit{Marketing Science} 34(2):218–234. 

\vspace{0.2cm}
\noindent Chen S, Moinzadeh K, Song JS, Zhong Y (2023). Cloud computing value chains: Research from the operations management perspective. \textit{Manufacturing \& Service Operations Management} 25(4):1338-56. 

\vspace{0.2cm}
\noindent Li B, Kumar S (2022). Managing software‐as‐a‐service: Pricing and operations. \textit{Production and Operations Management} 31(6):2588-608. 
}
\end{quotation}

% ===============================================================
\noindent \textbf{Authors' Response:} \textcolor{blue}{[...]} 

\hl{[...]}

\vspace{1.0em}
% ===============================================================

%%%%%%%%%%%%%%%%%%%%%%%%%%%%%%%%%%%%%%%%%
\newpage
\noindent \textbf{\underline{Authors' Response to Referee 1}} 
%%%%%%%%%%%%%%%%%%%%%%%%%%%%%%%%%%%%%%%%%
\section*{Summary of Referee 1 Responses and Strategic Review}

%===========================
\subsection*{1. Consistency Check and Suggested Revisions}
%===========================

\noindent 아래는 지금까지 작성된 Referee 1에 대한 Response 중, 
\textbf{일관성 또는 논리 구조상 점검이 필요한 부분}을 정리한 것이다.

\begin{itemize}
    \item \textbf{(A) Pricing parameter $q$에 대한 해석 변화 반영 필요} \\
    초기 Response (특히 R1 \#11 이전)에서는 $q \in [0,\infty)$을 연속적인 정책 스펙트럼으로 간주한 표현이 있었다.  
    하지만 이후 논의에 따라 우리는 $q$를 \textbf{“연속 축(continuous spectrum)”이 아닌, 설계 가능한 discrete pricing design parameter}로 명확히 재해석하였다.  
    따라서, 다음 항목들은 최종 통합 시 문장 정합성을 반드시 재확인해야 한다:
    \begin{itemize}
        \item Lemma 2에서 ``monotonic in $q$''라는 표현이 남아 있다면, 
        이를 ``for a given design with allowance $q$'' 또는 ``comparison across different pricing designs'' 형태로 통일.
        \item Figure 설명에서 $q \rightarrow \infty$와 같은 표현은 footnote에서 
        ``used only as a conceptual limit, our analysis does not rely on continuity''라는 식으로 보완 필요.
    \end{itemize}

    \item \textbf{(B) Provider--Renter 균형을 설명하는 구조에서 용어 일관성 필요} \\
    R1 \#10, \#12, \#13, \#14에서 ``market-clearing'', ``belong to'', ``optimal pricing path'' 등의 표현이 수정되었기 때문에, 
    본문에서는 일관되게 다음 구조를 유지해야 한다:
    \[
    \max \Pi(\cdot) \quad \text{s.t.} \quad \text{Demand} = \text{Supply}, 
    \quad P_{\text{fail}} \le \varepsilon.
    \]
    즉, ``capacity-binding region에서 platform의 최적 가격은 market-clearing locus 위에서 선택된다''는 설명으로 통일 필요.

    \item \textbf{(C) Theorem 1 -- Reliability 조건 반영 여부의 전역 확인 필요} \\
    R1 \#7과 \#15에서 availability = 100\% 삭제 및 failure probability 제약을 포함하기로 하였음.  
    Theorem 1뿐 아니라, Abstract / Introduction / Numerical 실험(section 6)에서 
    ``target reliability, $P_{\text{fail}} \le \varepsilon$''이 자연스럽게 이어지도록 문맥 정리 필요.

    \item \textbf{(D) Notation consistency} \\
    이미 R1 \#8에서 notation table을 추가하기로 했으므로, $v_{bp}^S, \rho_m, \theta, t, \alpha, \varepsilon$ 등의 기호가  
    모든 lemma, theorem, appendix 및 figure caption에서 동일한 의미로 쓰였는지 cross-check 필요.
\end{itemize}

\vspace{1em}


%===========================
\subsection*{2. Tone, Difficulty, and Key Strategic Points}
%===========================

\noindent Referee 1의 전체적인 tone과 대응 전략에서 중요한 관찰은 다음과 같다.

\begin{itemize}
    \item \textbf{(A) Tone \& 심리적 맥락} \\
    Referee 1은 매우 ``constructive''하고 ``friendly''한 tone이다.  
    \begin{itemize}
        \item 논문 주제는 흥미롭고 시의적절하다고 평가.
        \item 분석은 충분히 rigorous하며, revision은 ``major overhaul''가 아니라 ``clear exposition + minor model clarification''이라고 명시.
        \item 즉, ``수정하면 publishable potential''이 있다고 보는 긍정적 리뷰.
    \end{itemize}

    \item \textbf{(B) 수정 난이도 평가} \\
    전체적으로는 ``manageable but requires precision'' 수준이다.  
    수식/결과를 바꾸기보다는, \textbf{모델의 언어적 정의(clear assumptions), 용어 정리(notation), 경제적 해석(interpretation)}이 핵심이다.  
    단순 오타가 아니라, ``모형을 어떻게 설명해야 독자가 정확히 이해하는가''에 초점을 두고 있음.

    \item \textbf{(C) 가장 중요한 또는 주의해야 할 코멘트}  
    \begin{enumerate}
        \item \textbf{R1 \#4: “Who are providers/renters? 비용과 유지 구조 설명 필요”} \\
        시장 구조를 현실적으로 설명하라는 요청으로, 모형의 external validity와 연결된다.
        \item \textbf{R1 \#7 \& \#15: Reliability, failure probability, availability = 100\% 문제} \\
        이는 기술적으로 모델의 기본 가정에 해당하므로 Theorem/Appendix/Notation에 모두 반영되어야 함.
        \item \textbf{R1 \#11: $q$ interpretation (pricing design vs continuous spectrum)} \\
        본 논문의 pricing framework를 어떻게 positioning 할지를 결정하는 큰 방향 전환이기 때문에, 
        이후 numerical 및 discussion 전반에 반영 필요.
        \item \textbf{R1 \#14: “increase both $p_s$ and $p_b$”를 최적화 구조로 명확히 바꿀 것} \\
        단순히 가격을 올린다가 아니라, constrained optimization 형태로 재정의해야 하므로 필수 수정 사항.
    \end{enumerate}

    \item \textbf{(D) 향후 수정 시 주의할 점}  
    \begin{itemize}
        \item ``논리적 연결성 유지''가 가장 중요하다. 특히 Lemma 1 → Lemma 2 → Lemma 3 → Theorem 구조에서  
        어떤 값이 exogenous인지, 어떤 값이 endogenous인지 항상 명확히 해야 함.
        \item $q$에 대한 새로운 해석(정책 비교/설계 관점)을 Numerical Section에서 일관되게 반영해야 한다.
        \item Appendix의 수식 및 Main text의 언어가 서로 충돌하지 않도록 마지막에 전체 proofread 필요.
    \end{itemize}
\end{itemize}

\vspace{1em}

\noindent 전체적으로 Referee 1의 요구는 “모델을 바꾸라”는 것이 아니라,  
\textbf{“모델이 무엇을 가정하고 있고, 그 가정이 현실적으로/수학적으로 명확히 전달되도록 설명하라”}는 방향에 가깝다.  
정확하고 조직적인 수정만 하면 충분히 수용 가능성이 높은 tone이다.


\newpage
\begin{quotation}
{\em
\noindent \textbf{Referee 1:} This paper develops and analyzes a stylized model for a peer-to-peer data-storage and retrieval platform on the cloud. In particular, it considers three pricing policy classes, namely, two-part tariffs, subscription, and hybrid. Within each class, the authors study the platform’s profit maximizing parameters (usually prices, and also a free download allowance in the hybrid class) taking into account the decisions of providers and customers (i.e., renters). The arrangement between the platform and the providers is that of revenue sharing. The authors also use the results of their analysis to provide insights on the surplus enjoyed by both sides of the platform under different policies.
}
\end{quotation}

\noindent \textbf{Authors' Response:} [...]
% ===============================================================
\noindent \textbf{Comment Analysis (코멘트 분석)}  

이 코멘트는 비판이나 문제 제기가 아니라, 논문 내용을 정확하게 요약하며 시작하는 일반적 도입부이다.  
즉 “당신의 논문은 이런 내용을 다루고 있다”는 사실 확인이자, 리뷰어가 논문 구조와 주제를 제대로 이해하고 있다는 긍정적 신호이다.  
이 경우에는 논리적 방어보다는 공손한 동의, 감사, 논문의 목적을 다시 간결히 명확히 해주는 형태가 이상적이다.

\vspace{0.8em}

\noindent \textbf{Response Strategy (답변/대응 방향)}  

- 입장은 “감사 + 맞다(We agree)”이며, 불필요한 추가 설명이나 변명은 하지 않는다.  
- 논문의 목적과 구조가 정확히 이해된 것에 감사하며, 이후 코멘트에서 요청된 사항에 대해 명확히 보완할 것임을 암시한다.  
- 본 코멘트에 대해서는 manuscript 변경은 필요하지 않다고 명확하게 밝힌다.

\vspace{0.8em}

\noindent \textbf{Response (영어 원문)}  

We thank the reviewer for this clear and accurate summary of our work.  
Indeed, our paper develops a stylized model of a peer-to-peer storage and retrieval platform and compares three pricing schemes—two-part tariffs, subscription, and hybrid—under a revenue-sharing arrangement with providers.  
Within each scheme, we characterize the platform’s profit-maximizing pricing decisions (and, in the hybrid case, the optimal free-download allowance) while accounting for the participation decisions of both renters and providers.  
We also analyze how these pricing choices affect the surplus earned by each side of the market.  
We appreciate the reviewer’s accurate understanding of our modeling framework and objectives.

\vspace{0.8em}

\noindent \textbf{Manuscript Changes (본문 수정 사항)}  

- 본 코멘트는 단순 요약 및 긍정적 서술이므로, 이에 따라 직접적인 본문 수정은 필요하지 않다.  
- 다만 이후 코멘트에서 제안된 사항을 반영하여, 정의(예: first-best, market-clearing), 모형 전개 순서, 그리고 surplus 설명을 더 명확히 정리할 예정이다.

\vspace{1.2em}
% ===============================================================

\begin{quotation}
{\em
\noindent \textbf{Referee 1:} In terms of methodology, this paper is a good example of stylized modeling and tractable analysis. The strength of the paper lies in the interesting and current business model it studies. Overall, the writing is of good quality too; there is certainly scope for improvement too – I hope some of my comments or doubts (to be presented, below) will show the authors some of the expositional gaps that they should fill and make their writing throughout the paper even tighter. I should admit that I have not checked the proofs. My sense is that MSOM values papers that provide rigorous analysis and useful insights for important and current business models – from that perspective, I think this paper certainly has potential for publication in MSOM. I provide more specific suggestions/comments below – I believe the revision that my comments will entail is minor.
}
\end{quotation}

\noindent \textbf{Authors' Response:} [...]
% ===============================================================
\noindent \textbf{Comment Analysis (코멘트 분석)}  

- 리뷰어는 이 논문이 “stylized modeling + tractable analysis”의 좋은 사례라고 평가하며,  
  주제 또한 시의성이 있고 흥미롭다고 긍정적으로 반응하고 있다.  
- 필력도 전반적으로 좋다고 평가했으며, exposition(서술 방식, 설명 구조)에서의 개선 가능성을 언급했다.  
- 중요한 부분은 “MSOM에 적합한 주제와 방법”이라고 명확히 언급했고,  
  “revision은 minor 수준”이라고 기대를 낮춰준 점이다.  
- 즉, 이 코멘트는 논문의 방향성에 동의하고 있으며, “설명만 조금 더 명확히 하면 accept 가능”이라는 신호이다.

\vspace{0.8em}

\noindent \textbf{Response Strategy (답변/대응 방향)}  

- 리뷰어의 긍정적인 평가에 대해 감사의 뜻을 명확히 전달한다.  
- “Stylized yet insightful model”이라는 평가를 재확인하며, exposition 개선을 적극적으로 수용할 의지를 표현해야 한다.  
- 또한 “minor revision 정도면 충분하다”고 한 기대를 존중하면서,  
  이후 코멘트에서 그 구체적 요구를 충분히 반영할 것이며, 전체 논문 설명을 더 명확히 정리하겠다고 약속하면 된다.  
- Proof를 확인하지 않았다고 했으므로, proofs의 정확성을 재차 언급할 필요는 없으며, 대신 “필요시 supplementary check 가능” 정도로만 암시하거나 생략한다.

\vspace{0.8em}

\noindent \textbf{Response (영어 원문)}  

We sincerely appreciate the reviewer’s positive evaluation of our modeling approach, the relevance of the business setting, and the overall quality of the writing.  
We are also grateful for the reviewer’s observation that the paper has potential for publication in \textit{MSOM}, and we take seriously the suggestion that certain parts of the exposition can be tightened further.  
In the revision, we have carefully addressed the reviewer’s comments to improve clarity, strengthen the presentation of assumptions and economic intuition, and make the writing more concise and coherent throughout the paper.  
We thank the reviewer for indicating that the required revisions are minor, and we hope that the changes we have made meet that expectation.

\vspace{0.8em}

\noindent \textbf{Manuscript Changes (본문 수정 사항)}  

- 이 코멘트 자체는 구체적인 수정 요구 사항이 아니므로, 본문에 직접 반영되는 수정은 없음.  
- 하지만 이후 개별 코멘트에서 지적된 exposition 관련 문제를 반영하여 전체 텍스트 설명 흐름, notation 정리, 정의와 직관 설명 등을 더 명확히 수정 예정임.  
- General Response 및 각 섹션(모형 설명, 균형 조건, 그래프 해석 등)에 설명 문장 보강 계획 포함.

\vspace{1.2em}
% ===============================================================

\begin{quotation}
{\em
\noindent \textbf{Referee 1 (Point \#1):} Introduction: The data cited is not current. Please update the data you cite.
}
\end{quotation}

\noindent \textbf{Authors' Response:} [...]
% ===============================================================
\noindent \textbf{Comment Analysis (코멘트 분석)}  

리뷰어는 Introduction에서 인용한 시장 데이터 및 통계가 오래되었으며,  
최근의 산업 상황을 반영하도록 데이터를 업데이트하라고 요청하고 있다.  
이는 논문의 시의성과 현실성을 높이기 위한 요구이며, 난이도는 낮고 수정 가능성이 높은 코멘트다 (minor but necessary).

\vspace{0.8em}

\noindent \textbf{Response Strategy (답변/대응 방향)}  

- 리뷰어 의견에 전적으로 동의하고, 논문에 사용된 시장 규모, 사용자 수, P2P 스토리지 플랫폼의 채택 데이터 등을 2023–2024년 기준 최신 통계로 교체하겠다는 방향으로 대응한다.  
- 단순히 숫자만 대체하는 것이 아니라, 해당 숫자가 활용된 문장의 문맥까지 자연스럽게 수정하겠다고 약속한다.  
- 또한 모든 데이터는 공신력 있는 출처(Statista, DataIntelo, Mordor Intelligence, IDC 등)를 명시하고, 정확한 링크를 제공한다.  
- 만약 특정 수치가 보고서에서 명확히 인용되지 않은 경우에는, Response 또는 본문 각주에 “?”를 붙여 표시하여 투명성을 유지한다.

\vspace{0.8em}

\noindent \textbf{Response (영어 원문)}  

Thank you for this helpful comment. We agree that the market statistics reported in the Introduction should reflect the most recent data. In the revised manuscript, we have updated all numerical facts—such as the global size of the peer-to-peer storage market, the adoption of decentralized storage platforms, and overall cloud storage demand—using 2023–2024 industry reports. We also revised the surrounding text to ensure consistency with the updated figures. All new statistics are now cited with their original sources in the manuscript.

\vspace{0.8em}

\noindent \textbf{Manuscript Changes (본문 수정 사항 및 최신 데이터 예시)}  

다음과 같이 본문을 수정할 예정이며, 실제 논문에 반영 가능한 최신 수치를 제공함:

- **(수정 위치)** Introduction 1–2단락에서 기존의 2019~2021년 데이터(예: “cloud storage market was valued at…”)를 모두 삭제 또는 대체  
- **(업데이트될 실제 수치 예시)**  
  - Global peer-to-peer storage market size reached **USD 1.67 billion in 2024** (DataIntelo, 2024).  
%    - 출처: https://dataintelo.com/report/peer-to-peer-storage-space-market?utm_source=chatgpt.com  
  - Global cloud storage market revenue grew to **USD 103.7 billion in 2024** and is projected to reach **USD 220 billion by 2028** (Mordor Intelligence, 2024).  
%    - 출처: https://www.mordorintelligence.com/industry-reports/cloud-storage-market  
  - Decentralized storage network Storj reported over **24,000 active nodes and 30+ million stored objects** in 2024 (Storj Network Statistics, 2024?)  
%    - 출처: https://www.storj.io/blog/storj-network-statistics-2024  (정확한 수치 여부는 논문 인용 전에 검증 필요 → “?” 표시)

- **(본문 문장 예시 수정)**  
  - 기존: “In 2020, the global cloud storage market was valued at X…”  
  - 수정: “As of 2024, the global cloud storage market has reached approximately USD 103.7 billion, reflecting sustained growth driven by decentralized architectures and peer-to-peer storage adoption.”  

\vspace{1.2em}
% ===============================================================

\begin{quotation}
{\em
\noindent \textbf{Referee 1 (Point \#2):} Page 3 of 48: Line 1: “widely used the”: I think “the” should be deleted.
}
\end{quotation}

\noindent \textbf{Authors' Response:} [...]
% ===============================================================
\noindent \textbf{Comment Analysis (코멘트 분석)}  

리뷰어는 논문 본문 Page 3, Line 1에 있는 표현 “widely used the”가 문법적으로 잘못되었으며, “the”를 삭제해야 자연스럽다고 지적하였다.  
이는 단순 문법 오류 수정 요청이며, 논리나 모델에는 영향을 주지 않지만, MSOM의 언어 정확성 기준에서는 반드시 반영해야 하는 사항이다.

\vspace{0.8em}

\noindent \textbf{Response Strategy (답변/대응 방향)}  

- 리뷰어의 지적에 전적으로 동의하며, 해당 문장에서 “the”를 삭제하여 “widely used”로 수정하겠다고 명확히 답변한다.  
- 추가적으로, 유사한 문법 오류가 없는지 전반적으로 점검하겠다는 의지를 표현하면 신뢰도가 높아진다.

\vspace{0.8em}

\noindent \textbf{Response (영어 원문)}  

Thank you for pointing this out. We agree that the phrase “widely used the” on Page 3, Line 1 is grammatically incorrect. In the revised manuscript, we have removed “the” so that the sentence now reads “widely used.” We have also carefully proofread the manuscript to correct similar minor grammatical issues.

\vspace{0.8em}

\noindent \textbf{Manuscript Changes (본문 수정 사항)}  

- Page 3, Line 1: “widely used the” → “widely used” 로 수정 완료.  
- 추가로, 서론 및 본문 전반에서 유사한 문법 오류가 있는 문장을 함께 점검 및 수정할 예정.

\vspace{1.2em}
% ===============================================================


\begin{quotation}
{\em
\noindent \textbf{Referee 1 (Point \#3):} Next paragraph: Redundancy algorithms are mentioned without telling the reader what that means.
}
\end{quotation}

\noindent \textbf{Authors' Response:} [...]
% ===============================================================
\noindent \textbf{Comment Analysis (코멘트 분석)}  

- 리뷰어는 Introduction 다음 문단에서 “redundancy algorithms”라는 용어가 등장하는데,  
  이것이 정확히 무엇을 의미하는지 설명 없이 사용되고 있다고 지적한다.  
- 즉, m-of-n erasure coding, replication, Reed–Solomon 등 “redundancy 알고리즘의 개념 및 역할(데이터 조각 분산 저장을 통해 신뢰도 유지)”이 독자에게 설명되지 않은 상태라는 점을 문제 삼는 것.  
- 이는 모델 구조를 이해하는 데 핵심 용어이므로 단순 문장 수정이 아니라 **정의/설명 추가**가 필요하다.

\vspace{0.8em}

\noindent \textbf{Response Strategy (답변/대응 방향)}  

- 리뷰어 의견에 동의한다는 태도로 접근해야 한다.  
- “redundancy algorithms”을 처음 사용하는 부분(Introduction 말미 또는 Model Section 초반)에  
  간단한 정의 + 직관적 예시(m-of-n coding)를 추가해야 한다.  
- θ (replication factor 또는 redundancy ratio)가 어떻게 결정되는지와도 연결할 필요가 있다.  
- 추가 설명은 3~4문장 정도로 충분하되, 본문 흐름을 깨지 않도록 간결하게 작성한다.

\vspace{0.8em}

\noindent \textbf{Response (영어 원문)}  

Thank you for this helpful comment. We agree that the term “redundancy algorithms” appears without sufficient explanation in the current text. In the revised manuscript, we now provide a brief clarification when this term first appears. Specifically, we explain that a redundancy algorithm refers to the mechanism by which the platform divides each file into multiple encoded pieces and stores them across different providers (e.g., an $(m,k)$ erasure code or replication scheme) so that the file can still be recovered even if some providers become unavailable. We also clarify that the parameter $\theta$ in our model captures the storage expansion factor (i.e., the total amount of data stored relative to the original file size) induced by the chosen redundancy algorithm.

\vspace{0.8em}

\noindent \textbf{Manuscript Changes (본문 수정 사항)}  

- Introduction 또는 Model Section의 첫 부분에 다음과 같은 정의 문장 추가 예정:  
  “We use the term ‘redundancy algorithm’ to refer to the method used by the platform to split and encode data across multiple providers (e.g., replication or $(m,k)$ erasure coding) to ensure reliability despite provider failures. This choice determines the storage expansion factor $\theta$.”  
- θ, t and reliability\footnote{e.g., failure probability decreasing in t and increasing in θ}의 관계가 명확히 드러나도록 해당 문단을 재구성.  
- 이후 본문에서 redundancy 관련 표현이 등장할 때, 독자가 앞선 정의를 떠올릴 수 있도록 notation을 일관되게 사용.

\vspace{1.2em}
% ===============================================================


\begin{quotation}
{\em
\noindent \textbf{Referee 1 (Point \#4):} Discussion on this page: Who is a typical provider and who is a typical renter? Large businesses, small businesses, individuals? More generally, please explain more about who the participants in this market typically are and what is the meaning of storage and bandwidth costs that these participants incur. How significant are these costs today? Similarly, how does the platform guarantee a minimum amount of “uptime” at the providers – what if the providers do not follow the prescribed rules? More broadly, I am requesting the authors to make their discussion more transparent so that we can understand who the players are, how these systems are “maintained well” (the analysis takes care of the financial incentives; I am concerned about other issues such as a technically incompetent provider, etc.).
}
\end{quotation}

\noindent \textbf{Authors' Response:} [...]
% ===============================================================
\noindent \textbf{Comment Analysis (코멘트 분석)}  

리뷰어는 다음과 같은 현실적 질문을 제기하고 있다:  
1) 플랫폼의 typical provider와 renter는 누구인가? (개인? 중소기업? 데이터센터?)  
2) 이들이 부담하는 storage cost / bandwidth cost는 실제 무엇이며, 현재 얼마 정도인가?  
3) 공급자가 uptime을 지키지 않거나 rule을 따르지 않으면 플랫폼은 어떻게 보장하거나 제재하는가?  
4) 기술적으로 미숙한 provider나 불안정한 환경에서는 시스템이 어떻게 유지되는가?  
→ 즉, “이 모델이 실제 세계의 플레이어, 비용 구조, 운영 메커니즘과 잘 연결되어 있는가?”를 요구하는 코멘트다.  

\vspace{0.8em}

\noindent \textbf{Response Strategy (답변/대응 방향)}  

- 이 코멘트는 매우 중요하므로 **적극적으로 동의 + 명확한 설명 + 실제 시장 사례 인용** 방식으로 대응한다.  
- 본문에 “participant description subsection” 혹은 Model 앞부분에 짧은 설명을 추가한다.  
- Storj, Sia, Filecoin 등의 실제 사례를 근거로:  
  • Providers = idle storage 가진 개인/소형 서버 운영자/일부 스몰 비즈니스  
  • Renters = 백업 목적의 개인 사용자, Web3 서비스, 소규모 기업 등  
  • Storage cost ≈ per TB/Month 기준 (e.g., Storj에서는 $4/TB storing + $7/TB egress 비용?)  
  • Uptime = Sia, Filecoin 등은 minimum uptime (95~99%) 요구 + collateral or slashing 제도 도입  
- 논문에서는 비용을 단순화했지만, 현실과 연결될 수 있는 방식으로 storage cost, bandwidth cost를 해석하는 문장을 추가한다.  

\vspace{0.8em}

\noindent \textbf{Response (영어 원문)}  

We appreciate the reviewer’s insightful question regarding who the actual participants in peer-to-peer storage platforms are and how operational aspects such as storage costs, bandwidth expenses, and uptime requirements are reflected in our model. In the revised manuscript, we have added a short subsection in the Introduction (and briefly in the model setup) to clarify the following:

(1) \textit{Who are the providers and renters?}  
In existing decentralized storage platforms such as Storj, Sia, and Filecoin, most providers are individuals or small-scale businesses who contribute excess disk capacity and bandwidth from personal computers, NAS devices, or small servers; in some cases, professional node operators and data centers also participate. Renters are usually individual users or small firms seeking low-cost or censorship-resistant backup/storage.

(2) \textit{What do storage and bandwidth costs represent?}  
In practice, providers incur (i) the opportunity cost of dedicating disk space and (ii) bandwidth/energy costs for uploading and retrieving data. For example, Storj reports provider payouts of around \$1.50–\$2.00 per TB-month of storage provided and \$7 per TB downloaded (2024 data).\footnote{Storj Network Statistics, https://www.storj.io/blog (accessed 2024).} These values motivate our modeling of storage cost and bandwidth cost as per-unit operational burdens.

(3) \textit{How is uptime enforced?}  
Real-world platforms require providers to maintain minimum availability (typically 95–99\%). Enforcement mechanisms include periodic audits, cryptographic proofs-of-storage, and financial penalties such as forfeited collateral or slashing if a provider goes offline (as in Filecoin or Sia).\footnote{Filecoin Slashing Mechanism, https://spec.filecoin.io/; Sia Uptime Policy, https://sia.tech.} This aligns with the reliability parameter in our model, where providers must maintain uptime to earn revenue.

We have integrated these explanations into the revised manuscript so that readers can more clearly understand who the players are, which costs our parameters represent, and how real systems enforce service quality beyond financial incentives.

\vspace{0.8em}

\noindent \textbf{Manuscript Changes (본문 수정 사항)}  

- **Introduction (or Section 2 before model):** “Participants and cost interpretation” 단락 신설  
  - Providers = 개인 + 소규모 서버 운영자 + 일부 전문 노드  
  - Renters = 개인, 스타트업, Web3 서비스 등  
  - Storage cost = 디스크 용량 점유 비용 + 하드웨어 감가상각  
  - Bandwidth cost = 업/다운로드 시 데이터 전송 비용 (cloud egress 비용 참조)  
- **실제 사례와 수치 제시:**  
  - Storj payout: \~\$1.50–\$2.00 / TB-month, egress \~\$7 / TB (Storj 2024 Blog?)  
  - Filecoin: Collateral stake + slashing for downtime (Filecoin Project Documentation)  
- **Model Section 주석 추가:** θ, t가 어떻게 실무의 redundancy / uptime과 연결되는지 짧게 설명  
- **가능 시 Appendix에 현실 자료 표 포함 (Optional)**  

\vspace{1.2em}
% ===============================================================


\begin{quotation}
{\em
\noindent \textbf{Referee 1 (Point \#5):} “The redundancy rate, $m/k$, limits storage availability based on provider capacity.” – please explain.
}
\end{quotation}

\noindent \textbf{Authors' Response:} [...]
% ===============================================================
\noindent \textbf{Comment Analysis (코멘트 분석)}  

리뷰어는 문장 “The redundancy rate, $m/k$, limits storage availability based on provider capacity.”가 충분히 설명되지 않았다고 지적하였다.  
즉, 왜 redundancy ratio $m/k$ (또는 논문에서 사용하는 $\theta$값)이 공급자 입장에서 저장 가능 용량을 제한하거나, 전체 시스템의 이용 가능성(availability)에 영향을 주는지 더 명확히 서술하라는 요청이다.  
이는 (1) redundancy가 늘어날수록 파일 하나를 저장하는 데 필요한 물리적 용량이 증가하고, (2) 제공 가능한 provider의 총 스토리지 용량에 의해 구동 가능한 renter 수가 제한된다는 논리를 더 명확히 보여달라는 요구이다.

\vspace{0.8em}

\noindent \textbf{Response Strategy (답변/대응 방향)}  

- 이 코멘트에는 **명확히 동의**하고, redundancy 비율이 $\theta = m/k$일 때 “원본 1 단위 데이터를 저장하려면 $\theta$ 단위의 저장자원이 필요”하다는 직관을 명시적으로 설명한다.  
- 즉, provider 총 용량이 $C$라면, 저장 가능한 총 데이터는 $C / \theta$가 됨을 간단히 수식으로 보여주는 방식으로 수정한다.  
- 또한 “availability”는 $m$조각 중 $k$개만 있어도 복원 가능하므로 $\theta$가 커질수록 신뢰도는 올라가지만, 동시에 더 많은 provider 용량이 필요하게 된다는 trade-off를 명확히 설명한다.  
- 이 부분은 Model section 또는 Redundancy 설명 부분에 간단한 예시/표현을 추가하는 방식으로 반영한다.

\vspace{0.8em}

\noindent \textbf{Response (영어 원문)}  

Thank you for pointing out the need for clarification. In the revised manuscript, we now explain more explicitly why the redundancy rate $m/k$ (denoted as $\theta$ in our model) limits storage availability through provider capacity. Under an $(m,k)$ erasure code, a file of size 1 must be split and stored as $m$ encoded pieces, of which any $k$ suffice to reconstruct the file. Hence the total amount of storage consumed is $m/k = \theta$ times the original file size. If the platform has a total provider storage capacity of $C$, then at most $C/\theta$ units of original data can be stored.  

Therefore, a higher redundancy rate increases reliability (because more pieces can be lost without data failure) but simultaneously requires more total provider capacity, effectively limiting how many renters or how much data the platform can serve. We have added this explanation in the model section where redundancy is introduced.

\vspace{0.8em}

\noindent \textbf{Manuscript Changes (본문 수정 사항)}  

- Model section (redundancy 설명 문단)에 아래 내용 추가 예정:
  - “Under an $(m,k)$ erasure code, storing 1 unit of data requires $\theta = m/k$ units of total storage.  
     Thus, if the total provider capacity is $C$, the platform can accommodate at most $C/\theta$ units of original data.”  
- 문장 “The redundancy rate limits storage availability…” 뒤에 수식 예시 또는 간단한 예시(예: $m=6, k=3 \Rightarrow \theta = 2$이면 1TB 파일 저장에 2TB 필요함)를 추가.  
- 공급자 제한(capacity constraint)와 신뢰도 간 trade-off를 묶어서 직관적으로 보완.

\vspace{1.2em}
% ===============================================================



\begin{quotation}
{\em
\noindent \textcolor{red}{\textbf{Referee 1 (Point \#6):}} When alpha is first introduced, please discuss why it is assumed to be exogenous.
}
\end{quotation}

\noindent \textbf{Authors' Response:} \textcolor{red}{Thank you for raising this point. We agree that the role of $\alpha$, the platform’s revenue-sharing fraction with providers, should be clarified when it is first introduced. In the previous manuscript, these explanations were only included in \textbf{Section 6}, where we extend our main model by endogenizing commission rates. We admit that our earlier version did not sufficiently explain them in \textbf{Section 3.1}, where we introduce assumptions and their justifications, which can make readers concerned about the external validity of our main model.}

\textcolor{red}{To address this concern, we added our justification about using exogenous commission rates in the main model as follows:}

\begin{quotation}
{\em
\noindent \textcolor{red}{``Given that public controversies over commission rates often raise fairness and public-sentiment concerns beyond profitability (Gartenberg, 2021; Scheiber, 2015), platforms may be unable to implement profit-maximizing commission rates.''}
} \hl{(page XX in Section 3.1)}
\end{quotation}

\noindent \textbf{References:}

\vspace{0.2cm}
\noindent Gartenberg C (2021) Google will reduce play store cut to 15 percent for a developer’s first \$1m in annual revenue. \textit{The Verge}. \url{https://www.theverge.com/2021/3/16/22333777/
google-play-store-fee-reduction-developers-1-million-dollars}.

\vspace{0.2cm}
\noindent Scheiber N (2015) Growth in the ‘gig economy’ fuels work force anxieties. \textit{The New York Times}. \url{https://www.nytimes.com/2015/07/13/business/
rising-economic-insecurity-tied-to-decades-long-trend-in-employment-practices.
html}.


\vspace{1.2em}
% ===============================================================


\begin{quotation}
{\em
\noindent \textbf{Referee 1 (Point \#7):} Page 8: Isn't 100\% availability impossible? Also, do you assume a certain failure probability function (as a function of theta and t)?
}
\end{quotation}

\noindent \textbf{Authors' Response:} [...]
% ===============================================================
\noindent \textbf{Comment Analysis (코멘트 분석)}  

- 리뷰어는 Page 8에서 언급된 “100\% availability” 표현이 현실적으로 불가능하다고 지적한다.  
- 또한, 논문이 데이터 복구/실패 확률을 어떻게 모델링했는지 명확히 밝히지 않았으며,  
  특히 failure probability가 $\theta$ (redundancy)와 $t$ (provider availability or uptime probability)의 함수인지 질문하고 있다.  
- 이 코멘트는 모형의 신뢰성(availability vs redundancy) 정의를 명확히 하라는 요청이며 매우 타당하다.

\vspace{0.8em}

\noindent \textbf{Response Strategy (답변/대응 방향)}  

- “100\% availability”를 기술적으로 불가능하다는 점에 동의한다.  
- 해당 표현을 “availability approaching 1” 또는 “negligible failure probability”로 수정하겠다고 한다.  
- failure probability의 구체적 형태를 명시한다:  
  예) $P(\text{file failure}) = \sum_{i=0}^{k-1} \binom{m}{i} (1 - t)^i t^{m-i}$ 또는 단순화하여 “monotonically decreasing in t and increasing in \theta”로 두는 등.  
- 본 논문에서는 closed-form이 필요한 것이 아니라 “redundancy increases reliability, but increases storage need”라는 trade-off만 필요함을 설명한다.  
- 그렇지만 reviewer 요청에 따라, Appendix 또는 본문에 이 failure probability 구조를 명확히 설명하고 가정했다고 서술한다.

\vspace{0.8em}

\noindent \textbf{Response (영어 원문)}  

Thank you for this insightful comment. We agree that in practice 100\% availability is not achievable. In the revised manuscript, we no longer refer to availability as exactly 100\%, but rather as “approaching one” or “with negligible failure probability under high redundancy.”  

Regarding the second part of the comment, we clarify that we assume a standard reliability structure: if each provider is available with probability $t$, and the platform stores $m$ encoded pieces of each file (out of which any $k$ are sufficient to reconstruct the file), then the probability of file failure is the probability that fewer than $k$ providers are available, i.e.,  
\[
P_{\text{fail}} = \sum_{i=0}^{k-1} \binom{m}{i} (1 - t)^i t^{\,m-i}.
\]  
Equivalently, file availability is $1 - P_{\text{fail}}$, which increases in $t$ and in the redundancy ratio $\theta = m/k$. However, a higher redundancy ratio also increases total storage usage and thus reduces the number of renters the system can admit. We have added a brief explanation of this reliability assumption in the model section, and we refer the reader to the Appendix for the full expression.

\vspace{0.8em}

\noindent \textbf{Manuscript Changes (본문 수정 사항)}  

- Page 8에서 “100\% availability”라는 문구를 “(almost) perfectly reliable” 또는 “availability arbitrarily close to 1”으로 수정.  
- Model 섹션 또는 Appendix에 다음 내용을 추가:  
  - 공급자 각각이 $t$의 확률로 online, 총 $m$개의 조각 중 $k$개 이상이 남아 있으면 복구 가능.  
  - Failure probability 공식 또는 “monotonic in $t$ and $\theta$”라는 일반적 설명 포함.  
- “redundancy increases availability but consumes more provider capacity”라는 trade-off를 보다 명확히 설명.  
- 필요하면 footnote로 “Real systems such as Filecoin and Storj target 99.5–99.9\% availability rather than 100\%.” 추가 가능.

\vspace{1.2em}
% ===============================================================

\begin{quotation}
{\em
\noindent \textbf{Referee 1 (Point \#8):} Page 12: $v^S_{bp}$ should be defined. I think there are some other quantities also that are introduced without being defined.
}
\end{quotation}

\noindent \textbf{Authors' Response:} [...]
% ===============================================================
\noindent \textbf{Comment Analysis (코멘트 분석)}  

- 리뷰어는 Page 12에서 $v^S_{bp}$ 라는 기호가 정의 없이 처음 등장한다고 지적하였다.  
- 또한, 이와 유사하게 본문 전체에서 일부 기호(utility functions, payoff terms, threshold parameters 등)가 명확히 정의되지 않고 바로 사용되는 사례가 있다고 문제를 제기하였다.  
- 이는 논문의 exposition(설명력)과 명료성 부족을 지적한 것으로, MSOM 수준에서 반드시 수정해야 하는 부분이다.

\vspace{0.8em}

\noindent \textbf{Response Strategy (답변/대응 방향)}  

- 리뷰어 의견에 전적으로 동의하며, 모든 기호를 표기/정의한 후 사용하도록 수정하겠다고 명확히 답한다.  
- 특히 $v^S_{bp}$가 무엇을 의미하는지(예: subscription scheme에서 renter의 밴드폭 관련 순효용? or provider의 payoff?) 정확히 정의한다.  
- Model section 또는 Appendix에 **Notation Table** 또는 **“List of Symbols”**을 추가하는 것도 좋은 전략이다.  
- 동시에 본문 내에서 처음 등장할 때 즉시 정의되도록 문장 위치도 수정하겠다고 명시한다.

\vspace{0.8em}

\noindent \textbf{Response (영어 원문)}  

Thank you for pointing this out. We agree that $v^S_{bp}$ on Page 12 is introduced without a proper definition, and that a few other symbols appear before being explicitly defined. In the revised manuscript, we now clearly define $v^S_{bp}$ when it first appears. Specifically, $v^S_{bp}$ denotes the (net) utility of a renter who purchases the subscription plan and exceeds the free download allowance, thereby paying the per-unit bandwidth price.  

More broadly, we have carefully reviewed the manuscript and ensured that all symbols—such as utilities, payoff functions, and threshold parameters—are formally defined upon first use. To enhance clarity, we have also added a concise table of notation in the Appendix (and cross-referenced it in the text) so that readers can easily locate the meaning of each symbol.

\vspace{0.8em}

\noindent \textbf{Manuscript Changes (본문 수정 사항)}  

- Page 12에서 $v^S_{bp}$가 처음 등장하는 문장 바로 앞 혹은 해당 문장 내에 명확한 정의 추가:  
  예) “Let $v^S_{bp}$ denote the renter’s utility under the subscription scheme when bandwidth usage exceeds the free allowance and is billed at price $p_b$.”  
- 본문 전체를 다시 점검하여 정의 없이 등장하는 기호들을 모두 수정.  
- Appendix 또는 Model 섹션 끝에 **Notation Table** 추가: 모든 기호($v^S_{bp}, \psi, \theta, t, U_p, U_r$ 등)를 정리.  
- 본문 내에서 각 기호가 처음 등장할 때 hyperlink 또는 reference(\textit{see Table A1 in Appendix for notation}) 추가 예정.

\vspace{1.2em}
% ===============================================================

\begin{quotation}
{\em
\noindent \textbf{Referee 1 (Point \#9):} Lemma 1: I was initially very confused by this statement because I thought only $(p_s,p_b)$ are fixed and everything else was supposed to be an output. Then, I understood that many other things are also taken as a given in this lemma. So, Please reword this: 

Consider a given $(p_s,p_b), v_s,..., v_{bp}$. Then, the threshold sensitivity $\rho_m$ and the expected ….
}
\end{quotation}

\noindent \textbf{Authors' Response:} [...]
% ===============================================================
\noindent \textbf{Comment Analysis (코멘트 분석)}  

리뷰어는 Lemma 1의 서술 방식이 불분명하다고 지적한다.  
처음 읽을 때에는 $(p_s, p_b)$만 exogenous (주어진 값)이고 나머지 변수들은 endogenous (산출값)이라고 생각했으나, 실제로는 renter utility $(v_s, v_b, v_{bp})$ 등도 “주어진 값”으로 간주된 상태에서 threshold $\rho_m$과 기대 수익 등이 계산된다는 점이 나중에야 드러난다는 것이다.  
따라서 Lemma 1의 서술 자체를 다음과 같이 바꾸길 요청하고 있다:  
“Given $(p_s, p_b), v_s, \ldots, v_{bp}$, the threshold sensitivity $\rho_m$ and the expected … are …”

\vspace{0.8em}

\noindent \textbf{Response Strategy (답변/대응 방향)}  

- 리뷰어 의견에 동의하고, Lemma 1의 statement를 더 명확하게 다시 쓸 계획임.  
- 특히 “전제로 주어진 값들”과 “Lemma에서 도출되는 값들”을 명확히 구분하는 방식으로 문장을 수정.  
- notation 및 변수 정의가 명확히 정리될 수 있도록, Lemma 직전 또는 Notation Table에도 이 변수들이 고정값(given)임을 재차 명시할 계획.  
- 수식의 내용은 바뀌지 않지만, exposition을 더 직관적으로 수정한다는 점을 강조.

\vspace{0.8em}

\noindent \textbf{Response (영어 원문)}  

We appreciate the reviewer’s observation and agree that the original wording of Lemma~1 can cause confusion about which quantities are treated as given and which are derived. In the revised manuscript, we reword Lemma~1 to make these assumptions explicit. Specifically, we now begin the lemma with:

\emph{“Consider given values of $(p_s, p_b)$ and the corresponding utility levels $v_s, v_b$, and $v_{bp}$. Then, the threshold sensitivity $\rho_m$ and the expected number of participating renters (and providers) are characterized as follows: …”}  

This wording clarifies that the prices and utility expressions are taken as exogenous inputs for the lemma, and that $\rho_m$ and the subsequent expressions are outputs. We believe this resolves the confusion without altering any analytical result.

\vspace{0.8em}

\noindent \textbf{Manuscript Changes (본문 수정 사항)}  

- Lemma 1의 첫 문장을 다음과 같이 수정:  
  “\textit{Consider given values of $(p_s, p_b)$ and the utility terms $v_s, v_b, v_{bp}$. Then the threshold sensitivity $\rho_m$ and the expected number of … are …}”  
- 모델/Notation Section에서 $v_s$, $v_b$, $v_{bp}$ 등의 정의를 명확히 한 후, 이것들이 “price $(p_s, p_b)$에 의해 결정되며 Lemma에서는 주어진 값”으로 취급됨을 언급  
- Lemma 1 바로 앞 문단에 “In what follows, we treat $(p_s, p_b)$ and the induced utility expressions as given inputs, and derive the threshold $\rho_m$ and the equilibrium participation levels.”라는 문장을 추가  
- Notation Table 또는 Appendix에도 동일한 정보를 반영

\vspace{1.2em}
% ===============================================================

\begin{quotation}
{\em
\noindent \textbf{Referee 1 (Point \#10):} The sentence immediately preceding Lemma 2 refers to providers’ decisions. Why are these decisions needed for Lemma 2?
}
\end{quotation}

\noindent \textbf{Authors' Response:} [...]
% ===============================================================
\noindent \textbf{Comment Analysis (코멘트 분석)}  

- 리뷰어는 Lemma 2 직전에 “providers’ decisions”를 언급한 문장이 나오는데, Lemma 2 자체는 공급자 행동을 직접적으로 다루지 않기 때문에 “왜 공급자 의사결정이 필요하다고 언급했는지” 논리적으로 이해되지 않는다고 지적하고 있다.  
- 즉, **Lemma 2는 renter의 참여 임계값, 기대 인원 등을 다루는 내용인데, 그 직전에 공급자 이야기가 갑자기 등장하니 설명이 혼란스럽다**는 것이다.  
- 이는 텍스트의 흐름과 연결성(exposition clarity)의 문제이므로, 해당 문장 삭제 또는 명확한 이유 설명이 필요하다.

\vspace{0.8em}

\noindent \textbf{Response Strategy (답변/대응 방향)}  

- 이 코멘트에는 동의하는 것이 가장 안전하며, Lemma 2 직전 문장을 수정하거나 삭제하는 방식으로 대응해야 한다.  
- 만약 Lemma 2가 실제로 공급자 의사결정에 의존하지 않는다면 → “provider decisions” 언급을 삭제.  
- 만약 공급자 측 변수가 간접적으로라도 renter의 threshold에 영향을 준다면 → Lemma 2 문장에 직접적으로 그 의존성을 명시.  
- 논문 구조를 깔끔하게 만들기 위해, Lemma 2는 “주어진 공급자 참여 수준(or storage capacity)”을 전제로 하며, 공급자 선택은 이전 Lemma에서 이미 처리됐음을 명확히 안내하는 식이 좋다.

\vspace{0.8em}

\noindent \textbf{Response (영어 원문)}  

Thank you for pointing this out. We agree that the sentence preceding Lemma~2 could be confusing, because Lemma~2 itself does not require providers’ decision rules as an input. In the revision, we have clarified the logic. Specifically, we now explain that providers’ participation decisions are already determined in Lemma~1 (or in the preceding analysis), and Lemma~2 analyzes renters’ decisions \emph{conditional on a given level of provider participation and available capacity}.  

To avoid confusion, we have reworded the sentence so that it no longer suggests that providers’ decisions are directly needed to prove Lemma~2. Instead, the text now reads:  
\textit{“Given the provider participation outcome derived above, we next characterize renters’ participation via the following lemma.”}  

\vspace{0.8em}

\noindent \textbf{Manuscript Changes (본문 수정 사항)}  

- Lemma 2 바로 전 문장을 다음과 같이 수정:  
  “Given the provider participation outcome derived above, we now turn to renters’ decisions.”  
- 만약 기존 문장이 “providers’ decisions are needed to…”처럼 쓰여 있었다면, 이를 삭제 또는 수정.  
- Lemma 2의 증명이나 서술은 공급자 행동을 직접 사용하지 않으므로, 불필요한 참조 제거.  
- 공급자 결과가 필요한 경우라면, “Lemma 2 is conditional on the supply capacity established previously”라는 문구 추가.

\vspace{1.2em}
% ===============================================================


\begin{quotation}
{\em
\noindent \textcolor{red}{\textbf{Referee 1 (Point \#11):}} Last sentence of Lemma 2 about monotonicity with respect to $q$: Does this statement apply for hybrid pricing? My understanding is that $q$ is a parameter that appears only for hybrid pricing. But I think you are considering two-part tariff as $q=0$ and subscription as $q=\infty$. I don't recall this in the writing. Please include it. I know it is in the figure that follows.
}
\end{quotation}

\noindent \textbf{Authors' Response:} \textcolor{red}{Thank you for this important comment. We agree that our earlier wording could be interpreted as if we treat $q$ as a continuous spectrum from two-part to subscription pricing. Precisely, this is a \textit{design parameter} introduced to quantitatively express hybrid pricing in addition to the two extremes---i.e., two-part tariff and subscription pricing---instead of using it as a continuous policy variable that the firm can manipulate in our model. Specifically, a small allowance ($0 < q \le 1$) represents a capped or two-part–like design, while a large allowance ($q > 1$) corresponds to a subscription-like design. Our goal is to compare the performance of these distinct pricing regimes rather than to study infinitesimal changes in $q$.}  

\textcolor{red}{Accordingly, to avoid the potential confusion related to the role of $q$, we have revised the sentence preceding Lemma~2 to explicitly state the comparison across pricing schemes rather than the monotonicity within hybrid pricing, although this monotonicity is technically correct. By doing so, we clarify that two-part and pure subscription can be viewed as special, discrete designs ($q=0$ and sufficiently large $q$, respectively), distinguished from the points of a continuous axis within hybrid pricing.} 

\textcolor{red}{We have updated the text to reflect this interpretation as follows. First, we have inserted the overview of pricing schemes and design parameter $q$ in \textbf{Section 3.1} as follows:} 
\begin{quotation}
{\em
\noindent \textcolor{red}{``Cloud service providers commonly rely on two canonical nonlinear pricing schemes: two-part tariffs and flat-rate subscriptions \hl{[citations]}. Under a two-part tariff, the provider charges a fixed access fee in combination with a per-unit usage price, so that users internalize their marginal consumption through the usage fee while sharing common infrastructure costs through the fixed fee. In contrast, a flat-rate subscription specifies a single periodic payment that grants access to a prescribed level of service, effectively setting the marginal price of incremental usage to zero within the contracted scope. Prior analytical studies in centralized cloud markets have compared these two schemes in terms of provider profit, user surplus, and capacity utilization, highlighting that two-part tariffs can better align usage with costs when demand is heterogeneous, whereas subscriptions can be attractive for demand stimulation and risk reduction from the user's perspective \hl{[citations]}.}

\textcolor{red}{Motivated by both practice and this prior theoretical work, we consider a hybrid pricing scheme that is observed in centralized cloud offerings and is increasingly adopted in decentralized cloud environments \hl{[citations]}. In hybrid pricing, the provider specifies a fixed fee together with a free bandwidth allowance and applies a positive usage price only to consumption that exceeds this allowance. To capture and systematically analyze this design flexibility, we introduce a design parameter $q$ that determines the firm's pricing scheme by quantifying the free bandwidth allowance embedded in the contract. By doing so, we could compare the three pricing schemes in a unified analytical framework.''}
} \hl{(page XX)}
\end{quotation}

\textcolor{red}{Second, in \textbf{Lemma 2}, we have revised the last sentence to avoid the confusion as:} 
\begin{quotation}
{\em
\noindent \textcolor{red}{``Also, pricing schemes with higher bandwidth allowance have higher optimal fees; that is, $p_s^T < p_s^{Hl} < p_s^{Hh} < p_s^S$ and $p_b^T \le p_b^{Hl} \le p_b^{Hh} \le p_b^S$.''}
} \hl{(page XX)}
\end{quotation}

\vspace{1.0em}
% ===============================================================


\begin{quotation}
{\em
\noindent \textbf{Referee 1 (Point \#12):} Lemma 3: Please define “market clearing prices” before using that language.
}
\end{quotation}
% ===============================================================
\noindent \textbf{Comment Analysis (코멘트 분석)}  

리뷰어는 Lemma 3에서 “market-clearing prices”라는 용어가 정의 없이 처음 등장한다고 지적하였다.  
즉, “시장 균형 가격”이 정확히 어떤 조건을 만족하는 가격인지 (수요=공급? 공급 과잉이 없는 상태? 플랫폼의 최적 선택과 일치?) 명확하게 설명해야 한다는 요청이다.  
이는 개념적으로 중요한 용어이므로 반드시 논문 본문에서 정의를 먼저 제공해야 한다.

\vspace{0.8em}

\noindent \textbf{Response Strategy (답변/대응 방향)}  

- 리뷰어 지적에 동의하며, “market-clearing prices”를 Lemma 3 앞에서 명확히 정의할 것이다.  
- 이 정의는 다음을 포함해야 한다:  
  1) 지정된 가격 $(p_s, p_b)$ 또는 $(p_s, p_b, q)$에서,  
  2) 참여하는 renters의 총 수요가 providers가 제공하는 총 용량(또는 redundancy 고려된 실질 공급량)과 일치하는 상태, 즉 $D(p) = S(p)$,  
  3) 이 상태는 platform이 더 이상 추가적으로 수요를 수용하지 못하거나 초과 공급이 남지 않는 균형 상태라는 것.  
- 또한 논문에서 “first-best prices”(공급 제약이 없는 최적가격)과 “market-clearing prices”(공급 제약 존재, 수요와 공급이 일치하는 가격)를 구분하는 것도 명시할 계획.

\vspace{0.8em}

\noindent \textbf{Response (영어 원문)}  

Thank you for this remark. We agree that the term “market‐clearing prices” should be defined before it is used in Lemma~3. In the revised manuscript, we now explicitly define market‐clearing (or supply‐constrained) prices as follows:  

\emph{“For a given pricing scheme, a price pair (and allowance, if applicable) is said to be market‐clearing if (i) the number of renters who choose to participate at those prices exactly equals the total storage capacity supplied by participating providers (after accounting for redundancy), and (ii) no additional renter can be admitted without violating the supply constraint.”}  

This definition distinguishes the market‐clearing region from the first‐best region, where supply is abundant and the platform can set profit‐maximizing prices without being constrained by provider capacity. We have incorporated this definition immediately before Lemma~3.

\vspace{0.8em}

\noindent \textbf{Manuscript Changes (본문 수정 사항)}  

- Lemma 3 시작 직전에 “Definition: Market-Clearing Prices” 형태로 별도 문장 또는 boxed definition 추가:  
  - “A price pair $(p_s, p_b)$ (or $(p_s, p_b, q)$) is market-clearing if the induced renter demand for storage equals the total effective supply from providers, i.e., $D = S$, leaving no excess capacity and no unserved renters.”  
- First-best price와의 구분도 함께 명시:  
  - “First-best: supply is slack, platform chooses profit-maximizing price.  
     Market-clearing: supply becomes binding, price must satisfy $D = S$.”  
- 이후 Lemma 3과 Proposition 문장에서도 해당 정의된 용어를 정확하게 사용.

\vspace{1.2em}
% ===============================================================


\noindent \textbf{Authors' Response:} 

\begin{quotation}
{\em
\noindent \textbf{Referee 1 (Point \#13):} Lemma 3: “The platform’s optimal service fees belong to the market-clearing prices when”: I don’t understand this sentence, especially what you mean by “belong”. 
}
\end{quotation}

\noindent \textbf{Authors' Response:} [...]
% ===============================================================
\noindent \textbf{Comment Analysis (코멘트 분석)}  

- Lemma 3의 문장 “The platform’s optimal service fees belong to the market-clearing prices when …”에서  
  용어 “belong to”가 무엇을 의미하는지 명확하지 않다고 리뷰어는 지적함.  
- 즉, 플랫폼이 설정하는 최적 요금이 “시장 균형 가격들의 집합 안에 포함된다”는 의미인지,  
  “시장 균형 조건에서 결정된다”는 의미인지, 혹은 다른 의미인지 혼란스럽다는 것.  
- 이 문제는 단순 문법이 아니라 **Lemma의 logical statement의 정확성 문제**이기 때문에 명확히 재서술해야 함.

\vspace{0.8em}

\noindent \textbf{Response Strategy (답변/대응 방향)}  

- “belong to”라는 표현은 모호하므로 삭제하고, 더 정확한 논리적 표현으로 바꾼다.  
- 의도했던 의미는:  
  **“when capacity is binding, the platform’s profit-maximizing price must satisfy the market-clearing condition (i.e., demand = supply).”**  
- 즉, “optimal prices ∈ set of market-clearing prices”가 아니라,  
  → “optimal prices are determined under the market-clearing constraint.”  
- 본문에서는 “the platform chooses prices that satisfy market-clearing when…” 또는 “the optimal prices must lie on the market-clearing curve/boundary”처럼 명확하게 표현한다.

\vspace{0.8em}

\noindent \textbf{Response (영어 원문)}  

Thank you for pointing out this ambiguity. In Lemma~3, our intention was not to say that the optimal service fees “belong” to a set in a mathematical sense, but rather that, when provider capacity becomes binding, the platform’s profit-maximizing prices must satisfy the market-clearing condition—that is, renter demand equals the total effective supply from providers.  

To eliminate ambiguity, we have reworded the sentence in Lemma~3 as follows:  
\textit{“When the capacity constraint is binding, the platform’s optimal service fees are determined among the prices that satisfy the market-clearing condition (i.e., where renter demand equals provider supply).”}  

This wording avoids the unclear expression “belong to” and directly states the relationship between optimal prices and the market-clearing constraint.

\vspace{0.8em}

\noindent \textbf{Manuscript Changes (본문 수정 사항)}  

- Lemma 3의 해당 문장을 아래와 같이 수정 예정:  
  - 기존: “The platform’s optimal service fees belong to the market-clearing prices when …”  
  - 수정: “When the capacity constraint is binding, the platform’s optimal service fees are chosen from the set of prices that satisfy the market-clearing condition (demand equals supply).”  
- Lemma 3 앞에서 정의한 “market-clearing prices”와 연결되도록 cross-reference 추가.  
- 불필요한 집합 포함 표현 ‘belong to’는 모두 제거하고, constraint-based 경제적 표현으로 치환.

\vspace{1.2em}
% ===============================================================

\begin{quotation}
{\em
\noindent \textbf{Referee 1 (Point \#14):} Page 21: “Specifically, for a given pricing scheme, the platform is better off by increasing both $p_s$ and $p_b$ until the number of participating providers equals the number of renters willing to adopt the platform.”: There might be many ways of changing $(p_s,p_b)$. Which one do you pick?
}
\end{quotation}

\noindent \textbf{Authors' Response:} [...]
% ===============================================================
\noindent \textbf{Comment Analysis (코멘트 분석)}  

리뷰어는 Page 21에서 “플랫폼은 $p_s$와 $p_b$를 동시에 증가시켜 공급자 수와 렌터 수가 같아질 때까지 조정한다”고 쓴 문장을 문제 삼았다.  
이 문장은 “둘 다 올린다”는 방향성만 언급할 뿐,  
- 어떤 경로로 가격을 조정하는지,  
- $p_s$와 $p_b$ 중 어느 변수를 먼저/더 많이 올리는지,  
- 또는 실제로 플랫폼이 푸는 최적화 (maximize profit subject to capacity constraint) 문제에서 Lagrangian 조건을 통해 결정된다는 사실을 구체적으로 밝히지 않았기 때문에, “Which one do you pick?”이라는 질문이 나온 것이다.

\vspace{0.8em}

\noindent \textbf{Response Strategy (답변/대응 방향)}  

- “임의로 둘 다 올린다”는 식의 표현을 삭제하고, 명확히 **(1) 플랫폼은 profit maximization 문제를 풀며, (2) capacity constraint가 binding일 경우, 최적해는 market-clearing 곡선 위에 존재한다**는 구조로 다시 서술할 것.  
- 즉, $(p_s,p_b)$의 조정은 arbitrary가 아니라, 다음 최적화 문제를 따른다는 점을 명확히 설명해야 한다:

\[
\max_{p_s,p_b} \; \Pi(p_s,p_b)
\quad \text{s.t.} \quad \text{Demand}(p_s,p_b) = \text{Supply}(p_s,p_b).
\]

- 따라서, “둘 다 올린다”가 아니라 **“the optimal price pair lies on the market-clearing locus (curve/surface) characterized by D = S, and is determined by maximizing profit along that locus”**로 표현을 교체해야 함.

\vspace{0.8em}

\noindent \textbf{Response (영어 원문)}  

Thank you for this comment. We agree that the sentence was imprecise. In the revision, we clarify that the platform does not arbitrarily increase both $p_s$ and $p_b$. Instead, when provider capacity becomes binding, the platform solves a constrained optimization problem:
\[
\max_{(p_s,p_b)} \; \Pi(p_s,p_b)
\quad \text{subject to} \quad \text{Demand}(p_s,p_b) = \text{Supply}(p_s,p_b).
\]
Hence, the optimal prices are chosen \emph{on} the market‐clearing locus (where the number of renters equals the effective provider capacity), and not by freely or simultaneously raising $p_s$ and $p_b$ in any direction.  
We have revised the text on Page~21 accordingly to replace the phrase “increase both prices” with a statement that the optimal pricing solution is obtained by maximizing profit along the market‐clearing boundary.

\vspace{0.8em}

\noindent \textbf{Manuscript Changes (본문 수정 사항)}  

- 기존 문장 삭제/수정:  
  - “Specifically, for a given pricing scheme, the platform is better off by increasing both $p_s$ and $p_b$ until the number of participating providers equals the number of renters …”  
  - → 수정: “When capacity becomes binding, the platform chooses $(p_s,p_b)$ that maximize profit subject to the market-clearing constraint $D = S$. Thus, the optimal $(p_s,p_b)$ lies on the market-clearing boundary rather than being determined by arbitrarily increasing both prices.”  
- 필요 시 각주 또는 본문에 Lagrangian 접근 또는 best-response minimization 설명 추가.
- Figure 또는 Proposition에서 “capacity-constrained region”과 “market-clearing locus”라는 용어를 일관되게 반영.

\vspace{1.2em}
% ===============================================================

\begin{quotation}
{\em
\noindent \textbf{Referee 1 (Point \#15):} Theorem 1: Don't you need a constraint to ensure the target failure probability is not exceeded?: More generally, I think this theorem has to be worded much better and more clearly.
}
\end{quotation}

\noindent \textbf{Authors' Response:} [...]
% ===============================================================
\noindent \textbf{Comment Analysis (코멘트 분석)}  
리뷰어는 Theorem~1이 시스템 신뢰성(reliability) 또는 목표 failure probability 조건 없이 서술되어 있다는 점을 문제 삼았다.  
즉, redundancy ratio $\theta = m/k$를 선택할 때, $P_{\text{fail}} \le \varepsilon$ 와 같은 제약이 필요한데, Theorem에서 명시되어 있지 않다는 것이다. 또한 Theorem의 문장이 전체적으로 모호하며, 가정·제약·결론이 명확히 구분되지 않았다고 지적하였다.

\vspace{0.8em}

\noindent \textbf{Response Strategy (답변/대응 방향)}  
- 리뷰어의 지적에 동의하며, Theorem~1을 보다 명확히 재구성할 것이다.  
- 특히 다음 3가지를 포함해 재서술한다:  
  1) 플랫폼이 선택하는 의사결정변수 (pricing + redundancy ratio),  
  2) 목적함수 (profit maximization),  
  3) 신뢰성 제약조건 $P_{\text{fail}}(\theta,t) \le \varepsilon$ (또는 equivalently availability $\ge 1-\varepsilon$).  
- “the platform’s optimal decision lies on the market-clearing boundary”라는 표현도 병행하되, 신뢰성 제약이 추가됨을 명확히 서술.

\vspace{0.8em}

\noindent \textbf{Response (영어 원문)}  

Thank you for raising this essential point. We agree that Theorem~1 should explicitly incorporate the reliability (failure‐probability) constraint. In the revision, we have reworded the theorem to state that the platform chooses its pricing parameters and redundancy ratio $\theta = m/k$ to maximize profit, \emph{subject to} both (i) the market‐clearing condition (demand equals supply when capacity is binding) and (ii) a reliability constraint $P_{\text{fail}}(\theta,t) \le \varepsilon$, where $\varepsilon$ denotes the admissible failure probability.  

We now present Theorem~1 in the following structure:  
\[
\max_{(p_s,p_b,q,\theta)} \; \Pi(p_s,p_b,q,\theta)
\quad \text{s.t.} \quad 
\text{Demand} = \text{Supply}, \quad 
P_{\text{fail}}(\theta,t) \le \varepsilon.
\]  
This clarifies that the optimal pricing and redundancy decisions are made within the set of feasible, market‐clearing and reliability‐compliant designs. The theorem statement has been rewritten to clearly reflect these assumptions and conclusions.

\vspace{0.8em}

\noindent \textbf{Manuscript Changes (본문 수정 사항)}  
- Theorem~1을 아래와 같이 구조화하여 재작성: 목적함수–제약조건–결론 순으로 정리.  
- 문장 내 “optimal service fees belong to …” 또는 모호한 표현 삭제, 대신 “the platform selects $(p_s,p_b,q,\theta)$ to maximize profit subject to …”로 명확히 표현.  
- Appendix에서 failure probability 식 $P_{\text{fail}} = \sum_{i=0}^{k-1} \binom{m}{i}(1-t)^i t^{m-i}$을 명시하고, $\varepsilon$ 값 또는 industry standard 예시(Filecoin, Storj 기준)를 주석으로 추가.  
- 본문에서 “availability = 100\%” 표현은 이전 코멘트(Point~7)와 연결하여 “target availability (e.g., $1 - \varepsilon$)”로 통일.

\vspace{1.2em}
% ===============================================================

\begin{quotation}
{\em
\noindent \textbf{Referee 1 (Point \#16):} Page 23: Last paragraph: “Lemma ???”: Please proofread.
}
\end{quotation}

\noindent \textbf{Authors' Response:} [...]
% ===============================================================
\noindent \textbf{Comment Analysis (코멘트 분석)}  

- Page 23 마지막 문단에 “Lemma ???”라는 placeholder가 그대로 남아있는 상태로 제출되었음을 지적하고 있다.  
- 이는 단순한 오타 이상의 문제로, 전체 문서가 충분히 교정되지 않았다는 인상을 줄 수 있기 때문에 반드시 수정되어야 한다.  
- 특히 MSOM과 같은 저널에서는 이런 small editorial errors도 리뷰어가 “carelessness”로 판단할 수 있어 신뢰도에 영향을 준다.

\vspace{0.8em}

\noindent \textbf{Response Strategy (답변/대응 방향)}  

- 이 지적에 대해 전적으로 동의하며, 해당 부분을 올바른 Lemma 번호로 수정할 것을 명확히 밝힌다.  
- 더불어 논문 전체에 걸친 유사한 reference 오류(예: “Theorem ??”, “Figure ??”, citation placeholder 등)가 없는지 전면적으로 점검하고 수정하겠다는 내용을 추가하면 좋다.  
- 실제 수정에서는 Lemma 번호뿐 아니라 lemma와의 연결 문장도 명확히 하고 numbering consistency도 확인.

\vspace{0.8em}

\noindent \textbf{Response (영어 원문)}  

Thank you for catching this oversight. The phrase “Lemma ???” in the last paragraph of Page~23 was an unintended placeholder. In the revised manuscript, we have replaced it with the correct reference (Lemma~X.Y) and have carefully proofread the entire document to ensure that no similar placeholder or cross‐reference errors remain.

\vspace{0.8em}

\noindent \textbf{Manuscript Changes (본문 수정 사항)}  

- Page 23 마지막 문단의 “Lemma ???”를 정확한 Lemma 번호로 교체 (예: “Lemma 3” 또는 “Lemma 4”, 실제 numbering에 맞게 수정).  
- 전체 본문, Appendix, 그리고 figure/table references 전반에 대해  
  “Lemma ??”, “Fig. ??”, “Theorem ??”, “(cite)” 등 placeholder 형태의 잔여 표현을 모두 점검하고 수정.  
- Overleaf/LaTeX에서 \ref{}, \label{} 구조 재확인 및 compile 후 cross-reference 정상 출력 확인 예정.

\vspace{1.2em}
% ===============================================================

\begin{quotation}
{\em
\noindent \textbf{Referee 1 (Point \#17):} Page 24: Last paragraph of Section 5.1: “renters’ usage levels”: Renters don’t decide usage levels, right? So, I didn’t understand this sentence.
}
\end{quotation}

\noindent \textbf{Authors' Response:} [...]
% ===============================================================
\noindent \textbf{Comment Analysis (코멘트 분석)}  

- Page 24, Section 5.1 마지막 문단에서 “renters’ usage levels”이라는 표현을 사용했는데,  
  리뷰어는 “본 모형에서는 renter가 사용량(usage level)을 선택하는 의사결정자가 아닌데 왜 그런 표현을 썼는가?”를 지적했다.  
- 실제로 본 논문에서는 renter는 단순히 “참여 여부(adopt or not)”를 결정하고, 사용량은 확률 분포 또는 exogenous parameter로 주어진다.  
- 따라서 이 표현은 renter의 행동 메커니즘을 잘못 전달할 오해의 소지가 있다.

\vspace{0.8em}

\noindent \textbf{Response Strategy (답변/대응 방향)}  

- 리뷰어의 지적에 동의하며, “usage levels”이라는 표현은 부정확했음을 인정하고 수정한다.  
- 대신 “expected usage,” “demand intensity,” 또는 “realized consumption drawn from distribution” 등 모형 정의에 맞는 표현으로 교체.  
- 아울러, renter는 사용량을 전략적으로 선택하지 않으며, 사용량은 개별 타입의 특성 또는 분포에 의해 exogenously 결정된다는 점을 명확히 설명할 것.

\vspace{0.8em}

\noindent \textbf{Response (영어 원문)}  

We appreciate the reviewer’s observation. The reviewer is correct that in our model renters do not strategically choose their usage levels. Instead, each renter’s data demand (or bandwidth consumption) is determined exogenously by the type distribution, and the renter’s only decision is whether or not to participate in the platform.  

To avoid confusion, we have revised the sentence in the last paragraph of Section~5.1. The wording “renters’ usage levels” is replaced with “the expected data consumption of participating renters” (or “the demand intensity implied by each renter type”). We have also added a clarification that usage is not a decision variable for renters in our model.

\vspace{0.8em}

\noindent \textbf{Manuscript Changes (본문 수정 사항)}  

- Page 24, Section 5.1 마지막 문장 예시 수정:  
  - 기존: “renters’ usage levels affect …”  
  - 수정: “the expected data consumption (or bandwidth demand) of participating renters affects …”  
- 본문 중 관련 문단에 다음 설명 추가:  
  - “In our model, renters do not choose usage strategically; usage is determined by their type and is treated as exogenous. Their only decision is whether to join the platform.”  
- 유사하게 usage를 행동 변수로 잘못 표현한 문장이 있는지 전체 점검.

\vspace{1.2em}
% ===============================================================

%%%%%%%%%%%%%%%%%%%%%%%%%%%%%%%%%%%%%%%%%%%%%%%%%%%%%
% \begin{figure}
%     \centering
%     {
%     \includegraphics[width=0.7\linewidth]{.../...png}
%     }
%     \captionsetup{labelformat=empty}
%     \caption{Figure X. [...]}\label{fig:...}
% \end{figure}
%%%%%%%%%%%%%%%%%%%%%%%%%%%%%%%%%%%%%%%%%%%%%%%%%%%%%

%%%%%%%%%%%%%%%%%%%%%%%%%%%%%%%%%%%%%%%%%
\newpage
\noindent \textbf{\underline{Authors' Response to Referee 2}}\\
%%%%%%%%%%%%%%%%%%%%%%%%%%%%%%%%%%%%%%%%%
% ===============================================================
\section*{Summary of Responses to Referee 2}

\vspace{0.6em}
\noindent \textbf{1. Consistency Review (일관성 및 수정 방향 점검)}

\begin{itemize}
    \item 전반적으로 R2는 \emph{“결과 자체보다는 가정의 강도, 경제적 해석의 명확성, 모델 설명의 투명성”}을 중점적으로 지적하였다.  
    \item 이미 작성된 각 코멘트별 답변은 다음 네 가지 축에서 비교적 일관된 방향을 유지한다:
    \begin{enumerate}
        \item \textbf{Assumption Clarification}: Pareto 분포, 비용 조건(\(\xi>\frac{3}{4}\alpha\)), 단위 수요 가정 등은 “모델 단순화를 위한 것이며 결과의 본질을 해치지 않는다”는 프레이밍. 동시에 Numerical/Test로 robustness를 보강.
        \item \textbf{Equilibrium Explanation 강화}: Provider participation을 fixed-point equilibrium으로 재구성하고, market-clearing locus에서 최적화하는 구조를 본문에 명시적으로 언급.
        \item \textbf{$q$ Parameter의 Design Interpretation 유지}: $q$를 연속 variable이 아니라 \emph{pricing design parameter}로 간주하고, low-$q$ regime에서 two-part와 equivalence / high-$q$ regime에서 numerical 최적값 $q^\star$를 제시하는 방향.
        \item \textbf{본문-Appendix 연결 강화}: Reviewer가 지적한 “technical detail이 Appendix에만 숨어 있다”는 문제를 해결하기 위해, 본문에 핵심 논리–수식–직관을 요약 형태로 포함하기로 함.
    \end{enumerate}
    \item 현재 단계에서 R1과 비교했을 때 \textbf{논리 충돌이나 tone mismatch는 거의 없으며}, $q$에 대한 새로운 framing도 R1 Point~\#11에서 이미 조정되었고, 따라서 R2의 질문과도 모순되지 않는다.
    \item 다만, 실제 논문 수정 시 아래 두 가지는 특히 신경 써야 한다:
    \begin{itemize}
        \item (i) Abstract/Conclusion에서 “two-part always dominates”처럼 보이는 절대적 문장을 반드시 조건부(under baseline assumptions / market-clearing)로 조정.
        \item (ii) Simulation에서 high-$q$가 two-part를 이기는 경우가 존재한다는 점을 \underline{숨기지 말고}, “Theorem의 범위 밖에서 존재하는 numerical evidence”로 정직하게 명시.
    \end{itemize}
\end{itemize}

\vspace{1.0em}
\noindent \textbf{2. Referee 2의 톤, 난이도, 핵심 포인트}

\begin{itemize}
    \item \textbf{Tone}: R2는 직설적이고 비판적인 어조지만, \emph{“거부의 기조”는 아니다.} 오히려 “이 논문이 좋은 방향인데, 결과를 좀 더 robust하고 투명하게 보여달라”는 constructive stance.
    \item \textbf{난이도}: R2가 요구하는 변경사항은 대부분 \emph{서술 개선, 구조 명확화, 추가 numerical evidence}이다.  
          수학 자체를 완전히 다시 유도하거나 모델을 근본적으로 바꾸라는 요청은 아니다. → \textbf{Overall Revision 난이도: Moderate}.
    \item \textbf{가장 중요하거나 위험도가 높은 코멘트 세 가지}:
    \begin{enumerate}
        \item \textbf{Point \#1 (Assumption generality)}: distribution / cost assumption에 대한 robustness 요구. 제대로 대응하지 않으면 “모델 특수성” 비판으로 이어질 수 있음.
        \item \textbf{Point \#2+\#4 (Equilibrium clarity + market-clearing continuum)}: “참가자 균형이 어떻게 정의되는지 + 왜 그 가격쌍이 최적인지”를 명확히 보여줘야 신뢰 확보 가능.
        \item \textbf{Point \#8 (Optimal $q$ in hybrid)}: $q$를 진짜 design parameter로 다루고 싶다면, 그에 맞는 최적 혹은 경계 $q^\star$를 반드시 제시해야 함 (low/high regime 구분 포함).
    \end{enumerate}
    \item \textbf{향후 작성 시 주의사항}:
    \begin{itemize}
        \item “low-$q$ hybrid = two-part와 동일”이라는 결과는 \emph{증명 sketch + Appendix full proof + 필요시 numerical check}의 3단 구조로 설명해야 Reviewer가 납득할 수 있음.
        \item Section 5 (numerical 비교)에 \textbf{세 정책(two-part, subscription, hybrid)의 profit/welfare를 동일 조건에서 비교하는 figure/table}을 반드시 넣을 것 (R2가 명확히 요구).
        \item 문헌 리뷰에서 “file-sharing vs storage-sharing”의 차이를 분명히 하여 혼란을 제거할 것 (Point \#10).
    \end{itemize}
\end{itemize}

\vspace{1.0em}
\noindent \textbf{3. Overall Outlook}

\begin{itemize}
    \item Referee 2의 지적은 논문 전체의 논리 구조를 흔들기보다는, \emph{설명/구성/표현이 부족한 부분을 정확히 찌르는 유형}이다.
    \item 모든 코멘트는 충분히 대응 가능하며, R1보다 기술적/구조적 보완이 많지만 “major reconstruction” 수준은 아님.
    \item 따라서 본 논문은 R1, R2 양쪽에서 \textbf{긍정적 잠재력 → minor to moderate revision} 정도로 평가받을 가능성이 높음.
\end{itemize}

\vspace{1.2em}
% ===============================================================


\begin{quotation}
{\em
\noindent \textbf{Referee 2:} The paper studies three pricing schemes for the P2P storage platforms: two-part tariff, subscription, and hybrid. They claim that incentive schemes on such platforms have not been fully investigated. They compare the three schemes to identify when each scheme dominates the others from different stakeholder’s perspective.
}
\end{quotation}

\noindent \textbf{Authors' Response:} [...]
% ===============================================================
\noindent \textbf{Referee 2 – General Overview Comment}

\begin{quotation}
{\em
The paper studies three pricing schemes for the P2P storage platforms: two-part tariff, subscription, and hybrid. They claim that incentive schemes on such platforms have not been fully investigated. They compare the three schemes to identify when each scheme dominates the others from different stakeholder’s perspective.
}
\end{quotation}

\vspace{1em}

\noindent \textbf{Comment Analysis (코멘트 분석)}  
- 이 코멘트는 질문이나 비판이 아니라, \textbf{논문의 주제, 연구 범위, 기여를 요약한 진입 코멘트}이다.  
- 즉, Referee 2는 이 논문이 다루는 주제와 구조를 명확히 이해하고 있다는 것을 보여주며, 이후 구체적 비판 또는 보완 포인트를 제시하기 위한 도입부로 볼 수 있다.  
- 아직 긍/부정 판단은 드러나지 않았고, 본 논문이 주장하는 “세 가지 가격 정책 비교 + 플랫폼/공급자/사용자 관점”이라는 틀을 그대로 인식하고 있음을 의미한다.

\vspace{1em}

\noindent \textbf{Response Strategy (답변/대응 방향)}  
- 이 코멘트는 구체적 질문이 아니므로 별도의 반박 또는 수정 요청은 없다.  
- 하지만 우리는 감사 인사를 통해 Referee 2가 논문의 목적을 정확히 요약해주었음을 인정하고, 이후 상세 코멘트에 성실히 대응하겠다는 태도를 보여주면 충분하다.  
- R2가 논문의 framing을 이해했다는 것은 긍정적 신호이며, 이후 코멘트가 비판적이어도 논문 구조 전체를 바꾸라는 요구일 가능성은 낮음.

\vspace{1em}

\noindent \textbf{Response (영어 원문)}  

Thank you for accurately summarizing the focus and structure of our paper. Indeed, our objective is to study three implementable pricing schemes—two-part tariff, subscription, and hybrid—for peer-to-peer storage platforms, and to identify when each scheme is preferred from the perspectives of the platform, providers, and renters. We appreciate that you recognize the gap in the literature regarding incentive design in such platforms. We will address your specific comments in detail below.

\vspace{1em}

\noindent \textbf{Manuscript Changes (본문 수정 사항)}  
- 해당 코멘트는 단순 요약이므로 본문에서 변경할 사항은 없음.  
- 단, R2가 강조한 “incentive design has not been fully investigated”라는 문구는 Introduction에서 우리의 연구 동기와 정확히 일치하는지 다시 확인할 필요는 있음.  
- 필요 시 Introduction 1–2문장 정도를 ``We compare implementable pricing schemes from multi-stakeholder perspectives''라고 더 명확히 보완할 수 있음.

\vspace{1.2em}
% ===============================================================

\begin{quotation}
{\em
\noindent \textbf{Referee 2:} The paper studies a new business practice (P2P storage sharing) that seems to be relevant to the current world. The paper studies and compares three pricing schemes and they find that two-part tariff pricing always maximized platform profit, while other pricing schemes may be able to maximize total welfare depending on the size of potential providers.
}
\end{quotation}

\noindent \textbf{Authors' Response:} [...]
% ===============================================================
\noindent \textbf{Referee 2 – General Characterization of Results}

\begin{quotation}
{\em
The paper studies a new business practice (P2P storage sharing) that seems to be relevant to the current world. The paper studies and compares three pricing schemes and they find that two-part tariff pricing always maximized platform profit, while other pricing schemes may be able to maximize total welfare depending on the size of potential providers.
}
\end{quotation}

\vspace{0.8em}

% ===============================================================
\noindent \textbf{Referee 2 — Clarification on Profit Dominance (high-$q$ vs two-part under market-clearing)}

\noindent \textbf{Comment Analysis (코멘트 분석)} \\
질문 요지: 우리가 제시한 이론 결과(특히 profit dominance)에 대해, \emph{market-clearing 조건 하에서} high $q$ (generous allowance, subscription-like 디자인)와 two-part tariff를 \textbf{직접 비교하는 정리(Theorem)}가 본문에 존재하는지, 그리고 그 비교의 결론이 무엇인지가 불명확하다. 실제로 우리의 \textbf{시뮬레이션 결과에서는} 특정 파라미터 구간에서 \emph{high $q$가 two-part를 이기는} 사례가 나타난다. 이는 “two-part가 항상 최적”이라는 인상을 줄 수 있는 서술과 충돌할 수 있다.

\vspace{0.8em}

\noindent \textbf{Response Strategy (답변/대응 방향)} \\
(1) \textbf{범위(scope) 명확화}: 기존 Theorem은 market-clearing 경계에서 \emph{디자인 고정} 상태의 최적화(또는 within-design 특성)를 기술하며, \textbf{market-clearing 하에서 cross-design (high-$q$ vs two-part)의 일반적 우열을 정리하지 않는다}. \\
(2) \textbf{과한 일반화 축소}: 본문에서 two-part의 profit dominance를 \emph{조건부 결과}로 한정하고, 연속축 비교가 아니라 \textbf{디자인 비교}라는 프레이밍을 명확히 한다. \\
(3) \textbf{사실 전달}: 시뮬레이션에서는 \emph{high $q$가 우위}인 사례가 존재함을 인정하고, \textbf{그 조건}을 수치적으로 제시한다(예: provider capacity가 충분하고 baseline storage utility가 존재하거나, 사용 효용이 상한/볼록성 약화 등을 가질 때). \\
(4) \textbf{문헌 구조 보완}: 본문에 \emph{Remark/Proposition (counterexample-style)}를 추가해 “market-clearing 하에서도 두 디자인의 우열이 parameter-dependent임”을 명시. Theorem으로 포괄 우열을 주장하지 않음을 분명히 한다.

\vspace{0.8em}

\noindent \textbf{Response (영어 원문)} \\
Thank you for prompting this important clarification. In our baseline theory, Theorem~1 does not establish a general dominance result between a high-$q$ (subscription-like) design and a two-part tariff \emph{under market-clearing}; rather, our theorems characterize optimality and comparative statics \emph{within a given design} along the market-clearing boundary (together with the reliability constraint). We therefore do not claim that two-part tariff universally dominates high-$q$ pricing under market-clearing conditions. 

Consistent with this scope, our numerical analysis reveals parameter regions in which a high-$q$ design outperforms the two-part tariff in platform profit (e.g., when provider capacity is ample and renters derive a fixed baseline value from storage access, or when usage utility is bounded/less convex). In the revision, we will: (i) explicitly state that cross-design profit dominance is parameter-dependent, (ii) add a remark (and a counterexample-style proposition in the numerical section) demonstrating a case where a high-$q$ design yields strictly higher platform profit than a two-part tariff under market-clearing, and (iii) adjust the wording in the Introduction and Results so that two-part dominance is presented as conditional rather than universal.

\vspace{0.8em}

\noindent \textbf{Manuscript Changes (본문 수정 사항)} \\
\begin{itemize}
    \item \textbf{Scope statement 추가 (Model/Results 서두)}: “Our theorems characterize within-design optimality on the market-clearing boundary; they do not assert cross-design dominance between high-$q$ and two-part pricing under market-clearing.”
    \item \textbf{Wording 완화 (Intro/Conclusion)}: “two-part maximizes platform profit” → “under our baseline assumptions, two-part can maximize platform profit; however, cross-design profit dominance under market-clearing is parameter-dependent.”
    \item \textbf{Numerical Section 보강}: 
    \begin{itemize}
        \item \emph{Proposition (counterexample)}: Provide a calibrated parameter set (provider capacity high; baseline storage utility $u_0>0$; bounded/less-convex usage utility) where high-$q$ achieves $\Pi_{\text{high-}q}>\Pi_{\text{2-part}}$ under market-clearing.
        \item \emph{Figure/Table}: Profit heatmaps over $(q,p_s,p_b)$ with market-clearing enforced; highlight regions where high-$q$ dominates.
    \end{itemize}
    \item \textbf{디자인 프레이밍 일관화}: $q$는 연속축이 아니라 \emph{design parameter}. 본문 전반(특히 Lemma 2 부근)의 표현을 “comparison across implementable pricing designs”로 통일.
\end{itemize}
% ===============================================================

\newpage
%\vspace{1.2em}
% ===============================================================

\begin{quotation}
{\em
\noindent \textcolor{red}{\textbf{Referee 2 (Point \#1):}} The paper has a couple of strong technical assumptions that make the results hard to generalize. First of all, the paper assumes the bandwidth usage follows a Pareto distribution with $b = 2$. Second, they assume the unit operating cost is sufficiently large, i.e. $\xi > 3/4 \alpha$. In addition, the paper assumes that each renter has a demand of one unit volume and each provider rents out one unit of capacity. The distributional assumptions may be contributing to driving the results in the current paper, and I am not sure how robust the results will be for other distributions.
}
\end{quotation}

\noindent \textbf{Authors' Response:} \textcolor{red}{Thank you for clarifying major technical assumptions that need further justification and robustness checks to generalize our findings. In adherence to your comments, we carefully reviewed these assumptions and addressed their limitations as described below:}

\textcolor{red}{\textbf{Point \#1.1. Bandwidth usage distribution.} We acknowledge that while we adopted the Pareto distribution from the literature \citep{bandi2015robust, li2018should, ramirez2017adapt} to capture the heavy-tailed nature of cloud usage data, the specific assumption of $b=2$ limits generalizability. Moreover, with diverse cloud services and user behaviors, usage patterns may not always follow a Pareto distribution.}

\textcolor{red}{To address this concern, we relax the assumption by varying the Pareto shape parameter $b$ and by considering alternative distributions such as exponential and lognormal. A higher (lower) Pareto shape parameter produces a faster (slower) decay in the distribution, reducing (increasing) the likelihood of extreme usage. Since exponential and lognormal distributions decay faster than Pareto, they represent environments where extreme usage or outliers are relatively rare. This allows us to examine scenarios with differing prevalence of heavy users and frequent access to P2P storage.}

\textcolor{blue}{[영재 작업 후 작성 예정]} \hl{The results are presented in [...]} 

\textcolor{red}{The results indicate that the implications of pricing schemes do not change under different parametric conditions and distributions. In the revised manuscript, we describe this process and implications in \hl{Section 6.3} as follows:}

\begin{quotation}
{\em
\noindent \textcolor{red}{\textit{``First, our main model assumes that bandwidth usage follows the Pareto distribution with $b=2$ to account for the heavy-tail distribution and analytical tractability. To examine if our findings are misled by this assumption, we conduct a numerical analysis of different shapes of distributional tails. Our numerical experiments suggest that our findings remain consistent when we relax the assumption of bandwidth usage distribution of renters.''} (\hl{page XX})
}}
\end{quotation}

\textcolor{red}{\textbf{Point \#1.2. Large operating cost assumption.} Our model assumes that operating costs are sufficient enough to be considered in a provider's in-out decision. To be specific, $\xi \le \frac{3}{4}\alpha$ makes all potential renters join the P2P network across all possible pricing schemes in our model, which is highly inconsistent with the current practice. For example, the profitability and operating costs have been widely discussed among Storj's official forum The profitability and operating costs are widely discussed among Storj's official forum (\url{https://forum.storj.io/}) and other forum users (e.g., Reddit).}

\textcolor{red}{In this manuscript, we further clarify how operating costs can become significant enough to affect providers' decisions with more details as:}

\begin{quotation}
{\em
\noindent \textcolor{red}{``The profitability and operating costs are widely discussed among Storj's official forum (https://forum.storj.io/) and other forum users (e.g., Reddit). \textcolor{red}{In practice, there are several important sources of operating costs for sharing unused storage. In practice, providers face \textit{opportunity costs} from dedicating storage and bandwidth to the platform (vs. alternative uses). Also, they have to pay infrastructure costs, including electricity, internet, and hardware maintenance. For example, the global electricity price is USD 0.170/kWh for residential users on average (\url{https://www.globalpetrolprices.com/electricity_prices/}). The internet costs vary by countries, such as USD 0.23/Mbps in the U.S., USD 1.07/Mbps in Japan, and USD 2.28/Mbps in South Africa (\url{https://ceoworld.biz/2025/10/04/global-internet-prices-2025-the-worlds-cheapest-and-most-expensive-connections/})}.'' \hl{(footnote on page XX)}}
}
\end{quotation}

\textcolor{red}{To inform about the unexplored scenario, in the revised manuscript, we discuss the model's outcomes under $\xi \le \frac{3}{4}\alpha$ in \hl{Section 6.2} as follows:}

\begin{quotation}
{\em
\noindent \textcolor{red}{``First, our model excludes the case where operating costs are negligible for potential providers; that is, we assume that $\xi > \frac{3}{4}\alpha$. When $\xi$ is substantially small or $\xi \le \frac{3}{4}\alpha$ (i.e., low operating costs), all potential providers are willing to share their storage and participate in the platform. Since providers do not respond to bandwidth fees sensitively under small $\xi$, raising prices does not help renters find more capacity while it increases their financial burden. Therefore, the two-part tariff and the hybrid pricing, which compensate for providers' bandwidth services, are less effective in boosting system surplus than the subscription-based pricing. Consequently, the surplus implication of pricing schemes is similar to the first-best prices under substantially small $\xi$.''} \hl{(page XX)}
}
\end{quotation}

\textcolor{red}{\textbf{Point \#1.3. Unit volume assumption.} It is a valid concern that our assumptions of homogeneous storage supply across providers and demand across renters may have driven our findings. In our model, a storage unit represents a standardized storage package (e.g., 1TB) that serves as the basic trading unit in the platform. Multiple users can share a storage unit, or enterprise users can purchase multiple units.}

\textcolor{red}{One might be concerned whether the insights from a unit storage can be extended to multiple-unit cases. Given the formulation of our model, this extension does not change the model's insights for the following reasons. From a provider's perspective, it is straightforward that sharing multiple units lead to proportionally larger storage and bandwidth revenues. Also, given our assumption, the operating costs are proportional to provider's sensitivity to the bandwidth provision $\rho_j$, the expected bandwidth volume $\hat{\omega}_b$, and a function of uptime $\xi(t)$ (i.e., $\rho_j \hat{\omega}_b \xi$). Therefore, the shared storage volume itself does not directly affect provider decisions in our model.
} 

\textcolor{red}{Likewise, when a renter needs multiple units to store, given the same access frequency, the bandwidth volume also increases proportionally. Moreover, by the design of contracts, free bandwidth allowance also increases with the storage volume. Accordingly, the total payment amounts as well as the utility from storage and bandwidth services are proportional to a renter's contract volume, making the intuitions from our model remain unchanged.
}

\textcolor{red}{In the revised manuscript, we discuss this aspect as follows:
}

\begin{quotation}
{\em
\noindent \textcolor{red}{``Second, it is a valid concern that our assumption of homogeneous storage supply across providers may have driven our findings. In our model, a storage unit represents a standardized storage package (e.g., 1TB) that serves as the basic trading unit on the platform, and multiple users can share a storage unit or enterprise users can purchase multiple units. One might be concerned whether the insights from a single-unit setting extend to cases where providers share multiple units. However, given the formulation of our model, this extension does not change the insights on the provider side. Sharing multiple units leads to proportionally larger storage and bandwidth revenues, and, under our assumption, operating costs are proportional to the provider’s sensitivity to bandwidth provision $\rho_j$, the expected bandwidth volume $\hat{\omega}_b$, and a function of uptime $\xi(t)$ (i.e., $\rho_j \hat{\omega}_b \xi$). Therefore, the shared storage volume itself does not directly affect provider decisions in our model.''} \hl{(page XX in Section 6.2)}
}
\end{quotation}

\begin{quotation}
{\em
\noindent \textcolor{red}{``Second, it is reasonable to question whether the findings from the analysis based on a homogeneous unit storage demand extend to renters who require heterogeneous multiple units. When a renter needs multiple units, given the same access frequency, the bandwidth volume increases proportionally. Also, by the design of the contracts, the free bandwidth allowance also scales with storage volume. Accordingly, both the total payment and the utility from storage and bandwidth services are proportional to the renter’s contract volume. As a result, the qualitative intuitions from our model remain unchanged for heterogeneous renters who purchase multiple storage units.''} \hl{(page XX in Section 6.3)}
}
\end{quotation}

\vspace{0.5cm}
\begin{quotation}
{\em
\noindent \textcolor{red}{\textbf{Referee 2 (Point \#2):}} The formulation of the provider’s participation problem is poorly introduced. The paper should explain better the equilibrium model that drives the participation decision of providers. One better approach is to first conjecture on the number of participants (renters and providers). Given the conject, the provider can compute expected bandwidth usage charged to each provider and the net profit (contingent on the provider type). With this, the paper can calculate the proportion of providers that enjoy a positive profit and hence will participate. This in turn must conform with the initial conjecture to be an equilibrium.
}
\end{quotation}

\noindent \textbf{Authors' Response:} \textcolor{red}{We sincerely thank you for highlighting this important missing component. We agree that the participation decision of providers should be presented more clearly. Upon revisiting \textbf{Section 3.2.2} (``A Provider's Decision'') in the earlier version, we found that it only presented aggregate outcomes of providers under given conditions, without explaining how each provider individually forms and evaluates its participation decision.}

\textcolor{red}{In the revised manuscript, we have substantially clarified the equilibrium logic that drives providers' participation. Following your suggestion, we now explicitly describe the conjectural reasoning process. Given the market environment and the platform's pricing and redundancy decisions, each potential provider forms expectations about the participation levels on both the provider and renter sides. Based on this conjecture, the provider estimates the expected storage and bandwidth volumes that will be allocated to them, computes the corresponding expected revenue and operating costs, and decides whether participation yields a non-negative profit. We have added this explanation at the beginning of \textbf{Section 3.2.2}, which now reads as follows:}

\begin{quotation}
{\em
\noindent \textcolor{red}{``Each potential provider decides whether to offer their unused storage via the platform based on expected profitability. Given the market condition and the platform's service fees under the given free storage $q$, a provider first forms a conjecture about the number of active renters and other providers in the market. Using this conjecture, the provider estimates the expected storage and bandwidth volumes that will be assigned to them, computes the expected revenue net of operating costs, and determines whether participation is profitable.''} \hl{(page XX in Section 3.2.2)}
}
\end{quotation}

\vspace{0.5cm}
\begin{quotation}
{\em
\noindent \textcolor{red}{\textbf{Referee 2 (Point \#3):}} The provider’s profit function is questionable. In formulating $\pi_j$ (page 14), the paper assume that the storage revenue of a provider is proportional to the actual usage. However, when a provider joins the platform, he/she rents out one unit of capacity and he/she must set aside this capacity for usage by the platform. The service provider might expect a payment for the entire unit capacity rented to the platform. Hence this will change the object function presented on page 14. I am not sure what the current practice is regarding the storage payment, and it is not clearly explained in the paper. The authors need to better explain the revenue part.
}
\end{quotation}
% ===============================================================

\noindent \textbf{Authors' Response:} \textcolor{red}{Thank you for raising this important point. It is a valid concern that our original formulation of $\pi_j$ assumes providers are paid in proportion to actual usage, whereas providers might expect compensation for reserved capacity as well in practice (i.e., they commit to holding capacity regardless of usage). In the revised manuscript, we address this potential concern by (i) providing real-world practices among P2P storage platforms and (ii) discussing how reserved capacity might interact with our model and results.}

\textcolor{red}{Storj pays node operators for actual data stored or transferred, not reserved capacity (source: \url{https://storj.dev/node/payouts}). Sia also uses a usage‐based scheme or pays for what is stored with collateral and penalties tied to uptime and proofs, rather than paying simply for reserved capacity (source: \url{https://docs.sia.tech/provide-storage/about-hosting-on-sia}). Lastly, Filecoin combines usage fees, consensus block rewards, and collateral-based penalties. Providers do not simply get paid for reserved space, and they must actively prove storage over time and their rewards are tied to actual network performance and consensus incentives (source: \url{https://filecoin.io/blog/posts/the-economics-of-storage-providers/}).}

\textcolor{red}{To inform readers about such real-world practices, the revised manuscript offers these details as:}

\begin{quotation}
{\em
\noindent \textcolor{red}{``Following industry practice, we assume that providers are compensated for actual storage and bandwidth usage rather than merely for reserving capacity.''}

\vspace{0.4cm}
\noindent \textcolor{red}{``[Footnote] Storj pays node operators for the actual amount of data stored or transferred, not for reserved capacity (\url{https://storj.dev/node/payouts}). Sia also employs a usage-based scheme in which users pay for data that is actually stored, with collateral and penalties tied to uptime and proofs, rather than payments for reserved capacity (\url{https://docs.sia.tech/provide-storage/about-hosting-on-sia}). Lastly, Filecoin combines usage fees, consensus block rewards, and collateral-based penalties: providers are not simply paid for reserved space, but must actively prove storage over time, and their rewards are tied to realized network performance and consensus incentives (\url{https://filecoin.io/blog/posts/the-economics-of-storage-providers/}).''} \hl{(page XX in Section 3.2.2)}
}
\end{quotation}

\textcolor{red}{Second, we discuss how reserved capacity can affect the provider's utility function and our results. Suppose a provider is compensated for reserved capacity. Then, the platform has to pay for all reserved capacity regardless of its utilization, while renters pay for their usage only. Then, the functional form of $\pi_j$ can be rewritten as:}
\begin{equation*}
    \pi_j = R +\alpha (\hat{\omega}_s p_s + \hat{\omega}_{bp} p_b) - \rho_j \hat{\omega}_b \xi
\end{equation*}
\textcolor{red}{where $R$ indicates a fixed compensation for a provider's reserved capacity, which is independent of actual usage $\hat{\omega}_s$ and $\hat{\omega}_b$; the other notations are identically defined as the profit function in \textbf{Section 3.2.2}.}

\textcolor{red}{In this case, some providers becomes profitable only when the usage level is low because of their high operating costs from actual usage; that is, these providers may join the platform when $R$ is large and $\hat{\omega}_b$ is small, but they're incentivized to leave the platform right after the capacity becomes valuable. In other words, they would benefit when there are sufficiently large participating providers and the network's capacity is not utilized. In this circumstance, the platform's business model is very unlikely to be sustainable, which might explain why the current players don't adopt this strategy.}

\textcolor{red}{In the revised manuscript, we include this discussion as follows:}

\begin{quotation}
{\em
\noindent \textcolor{red}{``Third, one might be curious if it is a plausible option to consider compensating providers for reserved capacity rather than for realized usage. Suppose that a provider receives a fixed payment $R$ for the capacity it reserves, independently of the realized usage levels $\hat{\omega}_s$ and $\hat{\omega}_b$. Then the provider's profit can be written as $\pi_j = R + \alpha (\hat{\omega}_s p_s + \hat{\omega}_{bp} p_b) - \rho_j \hat{\omega}_b \xi$, where other notations are as in the profit function in Section~3.2.2. Under this scheme, some providers become profitable only when usage is low, because their operating costs associated with actual usage are high; that is, such providers may find it attractive to join the platform when $R$ is large and $\hat{\omega}_b$ is small, but they are then incentivized to leave once capacity becomes valuable and utilization increases. This mismatch between incentives and actual usage makes the platform's business model unlikely to be sustainable, which is consistent with the observation that leading decentralized storage platforms do not compensate providers solely for reserved capacity.''} \hl{(page XX in Section 6.2)}
}
\end{quotation}

\vspace{0.5cm}
\begin{quotation}
{\em
\noindent \textcolor{blue}{\textbf{[재웅-영재 논의 필요: 영재 의견대로 그대로 작성 예정]}} \textbf{Referee 2 (Point \#4):} There may be a continuum of prices that clear the market when demand exceeds demand, since the price decisions are two-dimensional (storage \& bandwidth, for the two-part tariff and hybrid pricing model). What does the market clearing prices look like and which pair maximizes the platform’s profit?
}
\end{quotation}
% ===============================================================

%-----------------------------
\noindent \textbf{Comment Analysis (코멘트 분석)}
%-----------------------------
리뷰어는 (two–part, hybrid에서) 가격 벡터가 $(p_s,p_b)$의 2차원이기 때문에, 공급 제약이 바인딩될 때 시장을 클리어하는 가격쌍이 \emph{연속체(곡선)}가 될 수 있음을 지적하고, 
(1) 그 \emph{market‐clearing 집합}의 구조가 어떻게 생겼는지, 
(2) 그중 \emph{어떤} 가격쌍이 플랫폼 이익을 극대화하는지 명확한 정식화와 해답을 요구한다.
이는 본 논문에서 이미 암시적으로 사용해 온 “market‐clearing 경계 위에서의 최적화”를 한층 더 \textbf{명시적이고 수학적으로} 보여 달라는 요청이다.

\vspace{0.6em}

%-----------------------------
\noindent \textbf{Response Strategy (답변/대응 방향)}
%-----------------------------
\begin{itemize}
  \item market‐clearing 조건을 $g(p_s,p_b;q)\equiv D(p_s,p_b;q)-S(p_s,p_b;q)=0$로 두고, (여기서 $q$는 \emph{디자인 파라미터})
  \item \emph{암시함수정리(IFT)}의 정규성 하에서, $M_q=\{(p_s,p_b): g=0\}$는 국소적으로 1차원 매니폴드(곡선). 
  \item 플랫폼은 제약최적화 $\max_{p_s,p_b}\Pi(p_s,p_b;q)\ \text{s.t.}\ g=0$를 푼다. 
        라그랑지안 $L=\Pi-\lambda g$의 FOC는 $\nabla\Pi(p_s,p_b;q)=\lambda\,\nabla g(p_s,p_b;q),\ g=0$.
  \item 등가적으로, $M_q$를 $p_b=\phi(p_s;q)$로 매개화할 수 있으면 1차원 문제 $\max_{p_s}\Pi\big(p_s,\phi(p_s;q);q\big)$로 축약 가능.
  \item (유일성) $M_q$ 위에서의 이익함수 $\Pi(\cdot)$가 \emph{강 준오목(strictly quasi‐concave)}이면 최적 가격쌍은 유일. 
        일반 경우 복수해 가능 → 본문에 \emph{tie‐breaking rule} 또는 \emph{정량 예시} 제시.
\end{itemize}

\vspace{0.6em}

%-----------------------------
\noindent \textbf{Response (영어 원문)}
%-----------------------------
Thank you for this useful comment. We now make the market‐clearing set and the associated optimization explicit.  
For a given pricing design $q$, let $g(p_s,p_b;q):=D(p_s,p_b;q)-S(p_s,p_b;q)$ denote the excess demand.  
The set of market‐clearing prices is
\[
M_q:=\{(p_s,p_b)\in\mathbb{R}^2:\ g(p_s,p_b;q)=0\}.
\]
Under standard regularity (e.g., $\partial g/\partial p_b\neq 0$), the Implicit Function Theorem implies that $M_q$ is locally a one‐dimensional curve.  
When capacity is binding, the platform solves
\[
\max_{(p_s,p_b)}\ \Pi(p_s,p_b;q)\quad\text{s.t.}\quad g(p_s,p_b;q)=0,
\]
with first‐order conditions $\nabla \Pi(p_s,p_b;q)=\lambda\,\nabla g(p_s,p_b;q)$ and $g=0$.  
Equivalently, if $M_q$ can be parameterized as $p_b=\phi(p_s;q)$, the problem reduces to
\[
\max_{p_s}\ \Pi\big(p_s,\phi(p_s;q);q\big),
\]
so the profit‐maximizing pair is the argmax of $\Pi$ \emph{along the market‐clearing locus}.  
If $\Pi(\cdot)$ is strictly quasi‐concave along $M_q$, the maximizer is unique; otherwise multiple profit‐maximizing pairs may exist, in which case we report a tie‐breaking rule and numerical illustrations.  
We have incorporated this formulation into the text and provide figures that plot $M_q$ together with iso‐profit contours to visually identify the maximizing point.

\vspace{0.6em}

%-----------------------------
\noindent \textbf{Manuscript Changes (본문 수정 사항)}
%-----------------------------
\begin{itemize}
  \item \textbf{Definition (Market‐Clearing Locus).} For pricing design $q$, define $M_q=\{(p_s,p_b): D(p_s,p_b;q)=S(p_s,p_b;q)\}$.
  \item \textbf{Proposition (Constrained Maximization on $M_q$).} The platform’s optimal price pair under a binding capacity constraint solves $\max_{(p_s,p_b)\in M_q}\Pi(p_s,p_b;q)$.  
        Provide the Lagrangian FOC $\nabla \Pi=\lambda \nabla g$ and (where applicable) the reduced‐form one‐dimensional program.
  \item \textbf{Uniqueness/Multiplicity Remark.} State sufficient conditions (e.g., strict quasi‐concavity of $\Pi$ along $M_q$) for uniqueness; otherwise document multiplicity and apply a stated selection rule. 
  \item \textbf{Numerical Illustration.} Add a figure that overlays the estimated $M_q$ and iso‐profit contours, marking the maximizing point(s) for two–part and hybrid designs.  
        This clarifies why a continuum of market‐clearing prices can exist and which pair maximizes profit.
\end{itemize}
% ===============================================================

\noindent \textbf{Authors' Response:} [...]

% ===============================================================

\vspace{0.5cm}
\begin{quotation}
{\em
\noindent \textcolor{blue}{\textbf{[재웅-영재 논의 필요: 영재 의견대로 그대로 작성 예정]}} \textcolor{red}{\textbf{Referee 2 (Point \#5):}} In Section 5, the comparison between different pricing models is not presented in full length.  For example, in theorem 3, the paper omits the two-part tariff in the comparison (in the presented statement). The abstract indicates the two-part tariff always dominates, and hence it must yield higher profit for the platform for the market-clearing price case.
}
\end{quotation}

% ===============================================================

%-----------------------------
\noindent \textbf{Comment Analysis (코멘트 분석)}
%-----------------------------
- 리뷰어는 Section 5 (numerical / comparative analysis)에서 세 가지 정책의 이익 비교가 충분히 “완결성 있게” 제시되지 않았다고 지적한다.  
- 특히 Theorem~3은 subscription vs hybrid만 비교하고 two-part는 statement에서 제외되어 있어, “two-part가 정말 market-clearing 상황에서 항상 더 이익인지”가 수학적으로 명확히 보이지 않는다는 문제를 제기한다.  
- Abstract에서 “two-part always dominates”라고 주장하는 것처럼 읽히는데, 본문 정리는 이를 명시적으로 뒷받침하지 않는다.  
- 즉, \textbf{논리 구조의 연결이 부족하고 claim이 너무 강하게 들릴 수 있음}.

\vspace{0.8em}

%-----------------------------
\noindent \textbf{Response Strategy (답변 방향)}
%-----------------------------
\begin{itemize}
    \item Theorem~2 (또는 앞선 결과)에서 two-part가 동일한 참여조건 하에서 이익 극대화를 제공한다는 점을 이미 사용했지만, 이를 Section~5나 Theorem~3에 명확히 연결하지 못한 점을 인정한다.  
    \item Abstract의 “always dominates”는 \textbf{baseline model + market-clearing + unlimited marginal utility} 하에서 조건부로만 성립함을 수정하여 표현을 완화하고, 본문에서 조건을 명확히 추가하겠다.
    \item Section~5에 “Profit ranking under market‐clearing”이라는 소단락 또는 Proposition을 신설해,   
    \[
    \Pi^{2\text{-part}}(p_s^*,p_b^*) 
    \ge 
    \max\Big\{ \Pi^{\mathrm{sub}}(p_s^{'},p_b^{'}),\ 
               \Pi^{\mathrm{hyb}}(p_s^{''},p_b^{''},q)\Big\}
    \]
    이 (모델 가정 아래) 성립함을 명확히 기술하겠다.  
    \item 또한 numerical section에서 동일한 market-clearing 조건 하에서 세 정책의 profit을 한 table 또는 figure로 병렬 비교하도록 보강한다.
\end{itemize}

\vspace{0.8em}

%-----------------------------
\noindent \textbf{Response (영어 원문)}
%-----------------------------
Thank you for pointing this out. You are correct that in Theorem~3 we focused on the comparison between subscription and hybrid pricing and did not restate the profit performance of the two‐part tariff, even though the Abstract refers to its dominance. Our intention was that the profit dominance of two‐part tariff under the market‐clearing condition had already been established earlier (in Theorem~2), and Theorem~3 then compares only subscription versus hybrid. However, we recognize that this connection is not sufficiently clear in the current draft.  

In the revised manuscript, we will:  
(i) explicitly state that the two‐part tariff serves as the profit‐maximizing benchmark among all three schemes under the baseline assumptions and market‐clearing participation;  
(ii) soften the statement in the Abstract to clarify that this dominance is conditional on the model assumptions (and not an unconditional universal statement);  
(iii) add a proposition or remark in Section~5 that formally reports the profit ranking across all three schemes under market‐clearing; and  
(iv) include a table/figure that directly compares $\Pi^{2\text{-part}}, \Pi^{\mathrm{sub}},$ and $\Pi^{\mathrm{hyb}}$ at their respective market‐clearing price pairs.

\vspace{0.8em}

%-----------------------------
\noindent \textbf{Manuscript Changes (본문 수정 사항)}
%-----------------------------
\begin{itemize}
    \item Abstract 수정: “always maximizes” → “maximizes under the baseline market‐clearing assumptions.”  
    \item Section 5에 Proposition/Remark 추가: two‐part profit is weakly greater than both subscription and hybrid at equilibrium prices.  
    \item Figure/Table 추가: 동일한 market‐clearing 조건 하에서 세 정책의 profit vector 시각화.  
    \item Theorem 3 앞/뒤에 “we now focus only on subscription vs hybrid because two‐part has already been shown to be profit‐maximizing in Theorem~2”라는 문장 추가.
\end{itemize}

% ===============================================================

\noindent \textbf{Authors' Response:} [...]

\textcolor{red}{Thank you for pointing this out. As you noted, our previous \textbf{Theorem 3} did not compare the firm's profit between the two‐part tariff and the subscription when $\psi^T \le n_p \le \psi^S$, while the previous abstract stated that ``the two-part tariff maximizes short-term profits'' in all cases. To inform thoroughly about this comparison, \hl{we expand our profit comparison as follows:}}

\begin{quotation}
{\em
\noindent \textcolor{red}{``\textbf{Theorem 2.} \hl{[...]}''} \hl{(page XX in Section 5.2)}
}
\end{quotation}

\textcolor{red}{Also, we admit that our previous \textbf{Theorem 3} (currently, \textbf{Theorem 2}) does not compare all pricing pairs (e.g., hybrid pricing and the two-part tariff) due to a lack of closed-form solutions. To help readers better understand the impacts of pricing schemes and obtain more practical insights, we conduct numerical experiments. The results show that \hl{[...]}}

\textcolor{red}{In the revised manuscript, we briefly introduce these results as follows:}
\begin{quotation}
{\em
\noindent \hl{``[...]''} \hl{(page XX, Section 5.2)}
}
\end{quotation}

\textcolor{red}{Moreover, to avoid overclaiming our findings, we revised several sentences to describe our theorems directly and mention numerical results as observations rather than final conclusions as follows:}
\begin{itemize}
    \item \hl{[...]} $\rightarrow$ \hl{[...]}
    \item \hl{[...]} $\rightarrow$ \hl{[...]}
    \item \hl{[...]} $\rightarrow$ \hl{[...]}
\end{itemize}

% ===============================================================


\vspace{0.5cm}
\begin{quotation}
{\em
\noindent \textcolor{red}{\textbf{Referee 2 (Point \#6):}} The prospect of the P2P storage sharing practice may not be bright due to the redundant nature of the sharing algorithm and the increased awareness on sustainability by consumers \& firms. I am concerned that the redundancy nature of the sharing mechanism will make the practice extinct in the future. If so, the impact of the findings in the paper will be quite limited.
}
\end{quotation}

% ===============================================================

\noindent \textbf{Authors' Response:} \textcolor{red}{This is a very great point. In concert with your concern, there has been rich discussion on the environmental sustainability of P2P storage platforms relative to centralized cloud infrastructures. The existing articles do not offer a unified conclusion, but it highlights several mechanisms through which decentralized storage may influence energy use, hardware demand, and carbon emissions.}

\textcolor{red}{On one hand, several studies point to potential positive environmental impacts of P2P or decentralized storage. Posani et al. (2018) suggested that the distributed architecture of cloud storage has a ~77\% reduction in carbon footprint for storage and roughly 50\% lower emissions for data transfers compared to conventional centralized cloud systems. Industry analyses further argue that decentralized architectures can reduce reliance on energy-intensive, continuously running data centers by leveraging underutilized existing devices rather than building new infrastructure (Lebens 2023), and importantly, it can reduce e-waste by lowering IT resource consumption, extending device lifespan, and eventually minimizing the need for new hardware production (Pezzuto 2025).}

\textcolor{red}{On the other hand, there have been several concerns about the environmental impacts of decentralized storage as well. In line with your perspective, Lebens (2023) noted that redundancy can increase total storage footprint and network traffic, offsetting some of the efficiency gains from decentralization. If replication is excessive or unoptimized, the cumulative energy consumption may even exceed that of a modern, well-optimized centralized data center. Additionally, the environmental impact of decentralized nodes depends heavily on the energy mix and hardware efficiency of the participating devices.}

\textcolor{red}{Taken together, it is difficult to conclude at this point whether P2P storage sharing will be environmentally superior or inferior in the long run. The direction of the sustainability impact appears to depend critically on implementation, redundancy strategy, energy source, and utilization patterns—factors that vary widely across platforms and may evolve with technological advances.}

\textcolor{red}{We believe these discussions actually suggest an interesting avenue for future research. Because the environmental impact of P2P storage is utilization-dependent, the externality generated by the sharing mechanism may vary endogenously with market conditions, adoption level, and algorithmic design. It would therefore be informative for future work to examine social welfare incorporating environmental externalities and to ask what optimal policies or platform designs should be in place to balance reliability, cost efficiency, and sustainability. Such analysis could meaningfully extend the relevance of our model to emerging debates on green digital infrastructure. In the revised manuscript, we incorporate this discussion in Section 7 as follows:}

\begin{quotation}
{\em
\noindent \textcolor{red}{``Lastly, while our model abstracts away from environmental externalities, ongoing debates on the sustainability of P2P storage highlight that redundancy, utilization patterns, and heterogeneous energy sources may meaningfully shape the net environmental impact of decentralized architectures. As suggested by Chen et al. (2023), sustainable cloud supply chain is an important but underexplored research area. In this regard, examining social welfare with explicit environmental externalities and identifying policies or design choices that balance reliability, cost efficiency, and sustainability would deepen our understanding of when P2P storage can serve as a socially desirable alternative to centralized cloud infrastructure.''} \hl{(page XX, Section 7)}
}
\end{quotation}

\vspace{0.2cm}
\noindent \textbf{References:}

\vspace{0.1cm}
\noindent Chen, S., Moinzadeh, K., Song, J. S., \& Zhong, Y. (2023). Cloud computing value chains: Research from the operations management perspective. Manufacturing \& Service Operations Management, 25(4), 1338-1356.

\vspace{0.1cm}
\noindent Posani, L., Paccoia, A., \& Moschettini, M. (2018). The carbon footprint of distributed cloud storage. arXiv preprint arXiv:1803.06973.

\vspace{0.1cm}
\noindent Lebens, W. (2023) Environmental Implications of Decentralized Hosting Platforms. Edge. \url{https://edge.network/resources/blog/environmental-implications-of-decentralized-hosting-platforms}

\vspace{0.1cm}
\noindent Pezzuto, I. (2025) The green cloud revolution: How distributed storage is reshaping digital sustainability. Illuminem. \url{https://illuminem.com/illuminemvoices/the-green-cloud-revolution-how-distributed-storage-is-reshaping-digital-sustainability}

\vspace{0.5cm}
\begin{quotation}
{\em
\noindent \textbf{Referee 2 (Point \#7):} There are quite a few serious typos in the paper. For example, on page 12, the calculation of $v_s^T$ and $v_b^T$ is missing a multiplicative factor “$n_r$”. The current expression calculates the fraction (proportion) of renters that will join the platform. There are also a few other places that the reference to theorems does not generate the right index number.
}
\end{quotation}

% ===============================================================

%-----------------------------
\noindent \textbf{Comment Analysis (코멘트 분석)}
%-----------------------------
리뷰어의 지적대로, p.12에서 $v_s^T, v_b^T$가 \emph{참여 비율(fraction)}을 산출하는 형태로 쓰여 있고, 
총 렌터 수 $n_r$를 곱해 \emph{참여 인원(expected count)}으로 변환해야 하는데 그 항이 누락되어 있었다. 
또한 Theorem/ Lemma 등 cross-reference 인덱스가 일부 깨져 있어 전반적 교정이 필요하다.

\vspace{0.6em}

%-----------------------------
\noindent \textbf{Response Strategy (답변/대응 방향)}
%-----------------------------
\begin{itemize}
  \item p.12의 $v_s^T, v_b^T$를 \textbf{비율}에서 \textbf{인원}으로 일관되게 정의: 
        $V_s^T := n_r \cdot \widehat{v}_s^T$, $V_b^T := n_r \cdot \widehat{v}_b^T$, 
        여기서 $\widehat{v}_\cdot^T$는 참여 \emph{비율} 표현.
  \item 본문 전체에서 “비율 vs 인원”을 명확히 구분하고, 표기와 단위를 통일.
  \item LaTeX \verb|\label|–\verb|\ref| 구조를 전면 점검해 Theorem/ Lemma/ Figure/ Table 참조번호 오류를 일괄 수정.
  \item Notation Table에 $n_r$ (잠재 렌터 총량), $V_\cdot$ (인원), $v_\cdot$ (비율)를 추가해 독자 혼동 최소화.
\end{itemize}

\vspace{0.6em}

%-----------------------------
\noindent \textbf{Response (영어 원문)}
%-----------------------------
Thank you for catching these errors. You are correct that on page~12 our expressions for $v_s^T$ and $v_b^T$ effectively compute \emph{fractions} of renters who adopt, and we inadvertently omitted the multiplicative factor $n_r$ to obtain the corresponding \emph{counts}. In the revised manuscript we now write
\[
V_s^T \;=\; n_r \cdot \widehat{v}_s^T, 
\qquad 
V_b^T \;=\; n_r \cdot \widehat{v}_b^T,
\]
where $\widehat{v}_s^T$ and $\widehat{v}_b^T$ denote the adoption fractions under scheme $T$ (two–part, hybrid, or subscription as specified). 
We have audited all downstream formulas (market‐clearing, profit, and welfare expressions) to maintain consistency between fractions and counts. 
We also carefully proofread the entire document and fixed all cross‐reference issues so that theorem, lemma, figure, and table indices are generated correctly.

\vspace{0.6em}

%-----------------------------
\noindent \textbf{Manuscript Changes (본문 수정 사항)}
%-----------------------------
\begin{itemize}
  \item \textbf{p.12 수식 교정:} 
        비율 $\widehat{v}_\cdot^T$에서 인원 $V_\cdot^T = n_r \cdot \widehat{v}_\cdot^T$로 명시. 
        동일 논리로, 공급자 측도 비율–인원 표기 혼용이 없는지 점검.
  \item \textbf{Market‐clearing 정리 재점검:} 
        $D(p_s,p_b;q)$가 \emph{인원} 기준 수요로 일관되게 쓰이도록 정리(필요 시 $n_r$ 인자 명시).
  \item \textbf{Notation Table 보강:} 
        $n_r$ (potential renters), $v_\cdot$ (fraction), $V_\cdot$ (count) 추가.
  \item \textbf{Cross‐reference 일괄 수정:} 
        \verb|\label|–\verb|\ref|, \verb|\autoref| 재컴파일 확인. 
        “Lemma ???” 류 placeholder 전면 제거.
\end{itemize}
% ===============================================================

\vspace{0.5cm}
\begin{quotation}
{\em
\noindent \textcolor{red}{\textbf{Referee 2 (Point \#8):}} The hybrid pricing with $q \le 1$ yield quite a few results that becomes equivalent to the two-part tariff case (from a profit stand viewpoint, indicated by Theorem 3). I am wondering if the price decisions of the hybrid model also reduce to the same as two-part tariff model. Moreover, what is the optimal $q$ for the two versions of the hybrid model? The paper does not explain the optimal $q$ level for the low and high q model cases. In my opinion, the optimal $q$ must be presented in order for readers to fully understand the differences of the pricing schemes under optimality.
}
\end{quotation}

% ===============================================================
\noindent \textbf{Authors' Response:} \textcolor{red}{Thank you for guiding us to make these points explicit. In this revision, we have clarified two points. First, we provided a formal result suggesting that low-$q$ hybrid reduces to two–part in both outcomes and optimal price choices in a new lemma (\textbf{Lemma X}) as follows:}

\medskip
\begin{quotation}
{\em
\noindent \textcolor{red}{\noindent\textbf{Lemma X (Low-$q$ Hybrid Equivalence).} Two-part tariff and hybrid pricing with free bandwidth allowance $q \le 1$ have the same profit and system surplus.}
} \hl{(page XX)}
\end{quotation}

\textcolor{red}{This lemma formalizes the equivalence hinted by our results: in the low–$q$ regime the free allowance is not large enough to change the marginal pricing seen by the marginal renter, so the hybrid plan implements the same effective entry and marginal prices as the optimal two–part scheme. We also describe this intuition when we interpret \textbf{Lemma X} on \hl{page XX}.}

\textcolor{red}{Second, we revised our manuscript to clarify that our goal is to compare potential outcomes across pricing schemes determined by design parameter $q$. Our results, which postulate $q$ as a given parameter, inform about relative profits across different $q$ levels eventually as follows:}
\begin{itemize}
    \item \textcolor{red}{In the low–$q$ regime, the profit–maximizing outcome is obtained in the range of $0 < q^\star \le 1$ and is equivalent to the two–part optimum.}  
    \item \textcolor{red}{In the high–$q$ regime, the profit-maximizing allowance $q^\star$ cannot be found analytically thus has to be numerically explored by maximizing platform profit along the market–clearing locus subject to the reliability constraint.}
    \item \textcolor{red}{Throughout the entire $q$ range, two-part tariff and the low-$q$ regime present the highest profit (\textbf{Theorem 2} and \textbf{Theorem 3}).}
\end{itemize}

\textcolor{red}{Although our findings eventually inform about the profit-maximizing $q$, this design parameter is not part of the optimization by the problem definition. Specifically, $q$ was introduced to quantitatively express hybrid pricing in addition to the two extremes---i.e., two-part tariff and subscription pricing---instead of using it as a continuous policy variable that the firm can manipulate in our model. Specifically, a small allowance ($0 < q \le 1$) represents a capped or two-part–like design, while a large allowance ($q > 1$) corresponds to a subscription-like design. Our goal is to compare the performance of these distinct pricing regimes rather than to study infinitesimal changes in $q$. Accordingly, we revised the manuscript as follows:}

\textcolor{red}{First, we have inserted the overview of pricing schemes and design parameter $q$ in \textbf{Section 3.1} as follows:} 
\begin{quotation}
{\em
\noindent \textcolor{red}{``Cloud service providers commonly rely on two canonical nonlinear pricing schemes: two-part tariffs and flat-rate subscriptions \hl{[citations]}. Under a two-part tariff, the provider charges a fixed access fee in combination with a per-unit usage price, so that users internalize their marginal consumption through the usage fee while sharing common infrastructure costs through the fixed fee. In contrast, a flat-rate subscription specifies a single periodic payment that grants access to a prescribed level of service, effectively setting the marginal price of incremental usage to zero within the contracted scope. Prior analytical studies in centralized cloud markets have compared these two schemes in terms of provider profit, user surplus, and capacity utilization, highlighting that two-part tariffs can better align usage with costs when demand is heterogeneous, whereas subscriptions can be attractive for demand stimulation and risk reduction from the user's perspective \hl{[citations]}.}

\textcolor{red}{Motivated by both practice and this prior theoretical work, we consider a hybrid pricing scheme that is observed in centralized cloud offerings and is increasingly adopted in decentralized cloud environments \hl{[citations]}. In hybrid pricing, the provider specifies a fixed fee together with a free bandwidth allowance and applies a positive usage price only to consumption that exceeds this allowance. To capture and systematically analyze this design flexibility, we introduce a design parameter $q$ that determines the firm's pricing scheme by quantifying the free bandwidth allowance embedded in the contract. By doing so, we could compare the three pricing schemes in a unified analytical framework.''}
} \hl{(page XX)}
\end{quotation}

\textcolor{red}{Moreover, to avoid the potential confusion related to the role of $q$, we have revised the sentence preceding \textbf{Lemma 2} to explicitly state the comparison across pricing schemes rather than the monotonicity within hybrid pricing, although this monotonicity is technically correct, as follows:} 
\begin{quotation}
{\em
\noindent \textcolor{red}{``Also, pricing schemes with higher bandwidth allowance have higher optimal fees; that is, $p_s^T < p_s^{Hl} < p_s^{Hh} < p_s^S$ and $p_b^T \le p_b^{Hl} \le p_b^{Hh} \le p_b^S$.''}
} \hl{(page XX)}
\end{quotation}

\vspace{1.0em}
% ===============================================================

\begin{quotation}
{\em
\noindent \textbf{Referee 2 (Point \#9):} I also feel that a lot of the technical stuff are hidden in the appendix, and as a result, it makes it very hard to understand the main drivers for the presented results. Maybe the authors should include some selective technical details in the paper.
}
\end{quotation}

\noindent \textbf{Authors' Response:} [...]
% ===============================================================
\noindent \textbf{Referee 2 (Point \#9)}

\begin{quotation}
{\em
I also feel that a lot of the technical stuff are hidden in the appendix, and as a result, it makes it very hard to understand the main drivers for the presented results. Maybe the authors should include some selective technical details in the paper.
}
\end{quotation}

\vspace{0.8em}

%-----------------------------
\noindent \textbf{Comment Analysis (코멘트 분석)}
%-----------------------------
- 리뷰어는 본 논문이 핵심 기술적 논리(참여 임계값 유도, market-clearing 조건, 최적화 논리 등)를 지나치게 Appendix로 보냈다고 지적한다.  
- 이는 독자가 본문만 보고서는 왜 이런 결과(예: two–part 우위, hybrid 구조, $q$ 효과 등)가 도출되는지 쉽게 이해하기 어렵다는 의미다.  
- 즉, \textbf{“결과만 있고, 왜 그런 결과가 나오는지의 핵심 메커니즘이 본문에서 충분히 보이지 않는다”}는 비판으로 해석할 수 있다.

\vspace{0.8em}

%-----------------------------
\noindent \textbf{Response Strategy (답변/대응 방향)}
%-----------------------------
우리는 다음 방향으로 조정할 것을 약속한다:
\begin{itemize}
  \item Appendix에 있던 수학 전개 중, \textbf{핵심 직관 및 필수적인 중간단계} (예: threshold $\rho_m$의 유도 방식, market-clearing에서 $\Pi(p_s,p_b;q)$가 최적화되는 구조, low–$q$ equivalence의 핵심 step 등)을 \textbf{간결한 형태로 본문에 옮겨온다.}
  \item 전부를 옮기기보다는, \textbf{(i) 독자가 결과를 이해하는 데 필요한 핵심 기제만} 하위 Lemma나 Remark 형태로 본문에 포함. 
  \item Appendix는 여전히 full proof와 확장 결과를 담되, 본문에 “logic flow 요약 + 핵심 식”을 삽입해 이해 가능성을 개선한다.
\end{itemize}

\vspace{0.8em}

%-----------------------------
\noindent \textbf{Response (영어 원문)}
%-----------------------------
We appreciate this observation. We agree that, in the current version, many of the derivations are placed entirely in the Appendix, which makes it less transparent how the key results—such as the threshold for provider participation, the structure of the market‐clearing prices, and the equivalence of low–$q$ hybrid and two–part pricing—are generated.

In the revised manuscript, we will bring selected technical steps (in a concise form) into the main text, particularly:
\begin{itemize}
    \item a short derivation or intuitive explanation of the provider participation threshold and the fixed‐point logic behind market clearing;
    \item a concise statement of how the Lagrangian condition on the market‐clearing locus leads to the optimal $(p_s,p_b)$ for each design;
    \item a brief sketch of the low–$q$ hybrid equivalence proof so that the reader can see why profits match the two‐part tariff.
\end{itemize}
The full proofs will remain in the Appendix, but the main paper will now contain enough intermediate steps and economic intuition to make the analytical “drivers” of the results self‐contained and easier to follow.

\vspace{0.8em}

%-----------------------------
\noindent \textbf{Manuscript Changes (본문 수정 사항)}
%-----------------------------
\begin{itemize}
  \item Section 3에 provider 참여 임계값 $\rho_m$ 유도에 관한 핵심 수식과 fixed-point 구조를 요약한 “Remark” 삽입.
  \item Section 4 앞부분 또는 Lemma 2 이후, market‐clearing 제약 하 최적화의 Lagrangian 직관(한 단락 정리) 추가.
  \item Low–$q$ hybrid equivalence 관련 핵심식 및 요지(두세 줄로 된 proof sketch)를 본문에 포함.
  \item Appendix는 full proof/확장내용로 유지하되, 본문에 증명 구조 요약을 두어 독자가 연결고리를 놓치지 않도록 구성.
\end{itemize}
% ===============================================================

\vspace{0.5cm}
\begin{quotation}
{\em
\noindent \textcolor{red}{\textbf{Referee 2 (Point \#10):}} The authors claim in the literature review that this paper is very different from the works in the P2P sharing of computing resources. I am not exactly sure what is the key difference based on the description. For example, on page 9, the paper states that “We now further explain three essential players in the P2P file-sharing service”, while earlier in the paper the authors claim storage sharing is very different from file-sharing. It is just confusing for a paper to have statements that seem to contradict with each other.
}
\end{quotation}

% ===============================================================

\noindent \textbf{Authors' Response:} \textcolor{red}{Thank you for the great comment on our literature review and positioning. First of all, we apologize for the confusion from the inconsistency. The sentence, ``We now further explain three essential players in the \textit{P2P file-sharing service:} renters, providers, and a service platform'', was evidently misleading. Again, we are sincerely sorry for causing unnecessary confusion. In this revision, we have rewritten this sentence as ``We now further explain three essential players in the P2P storage-sharing service: renters, providers, and a service platform.''}

\textcolor{red}{Besides, we admit that our previous manuscript did not properly explain how P2P cloud sharing is differentiated from traditional P2P file sharing and other computing-resource sharing. To address this concern, our new manuscript clarifies this aspect as shown below:} 

\begin{quotation}
{\em
\noindent \textcolor{red}{``Moreover, P2P storage sharing differs from both traditional P2P file-sharing and other computing models in several important aspects. In typical P2P file-sharing systems (e.g., BitTorrent-type networks), users temporarily exchange copies of existing files, and no peer is required to reserve capacity, retain data, or guarantee uptime after a transfer. Likewise, P2P computing platforms that share CPU or GPU cycles allocate short-lived processing tasks without any long-term obligation to store data or ensure its future availability. By contrast, P2P storage sharing requires providers to commit disk capacity over time, maintain availability, and bear redundancy and retrieval costs so that renters' data remain intact and accessible. These persistent capacity and reliability requirements make storage sharing different from a one-shot exchange of content or computation, and they critically shape the platform's pricing, redundancy, and service design problems studied in this paper.''} \hl{(page XX in Section 2.2)}

\vspace{0.4cm}
\noindent \textcolor{red}{``In addition, in contrast to CPU/GPU sharing, which involves only short-lived computation tasks with no ongoing data responsibility, P2P storage sharing imposes persistent capacity, availability, and redundancy obligations on providers. Our modeling framework enables us to offer unique insights into these services by incorporating these unique aspects of P2P storage sharing, such as providers' subsequent commitments of capacity, joint pricing of storage and bandwidth under heterogeneous costs, and considering redundancy algorithms as an extension.''} \hl{(page XX in Section 2.4)}
}
\end{quotation}

\textcolor{red}{In addition, following the review team's guidance, we have strengthened our papers' relationship with the recent OM literature on centralized cloud and interdisciplinary studies on subscription and other pricing schemes in service operations as follows:}

\begin{quotation}
{\em
\noindent \textcolor{red}{``Compared with private clouds, which are deployed for the exclusive use of a single organization, public clouds that serve the general public pose greater challenges for capacity management \citep{li2022managing}. Reflecting this, a growing operations management (OM) literature examines how to design pricing schemes for public cloud services.''} \hl{(page XX in Section 2.1)}

\vspace{0.4cm}
\noindent \textcolor{red}{``Recent OM research has emphasized that capacity and supply decisions are central in centralized cloud services. \citet{chen2023cloud} highlighted that cloud providers must plan data center capacity and hardware purchases under long lead times and uncertain demand, so decisions about when and how much to expand are critical. Along these lines, \citet{arbabian2021capacity} studied how a cloud provider expands capacity when servers come in fixed configurations that bundle CPU and memory, which can create imbalances between what is supplied and what users demand.''} \hl{(page XX in Section 2.1)}

\vspace{0.4cm}
\noindent \textcolor{red}{``In our setting, the platform does not directly choose data center investments or server configurations; instead, it must use pricing to attract and coordinate capacity from many decentralized providers, making pricing and capacity management inherently intertwined.''} \hl{(page XX in Section 2.1)}

\vspace{0.4cm}
\noindent \textcolor{red}{``\citet{balasubramanian2015pricing} compared selling and pay-per-use contracts for information goods, modeling usage as consumption of an information service with negligible marginal cost, and showed that pay-per-use can yield higher profits than selling for a monopolist. Our setting is also related to \citet{essegaier2002pricing}, who compared flat-rate, usage-based, and two-part-tariff pricing for centralized access services under capacity constraints and heterogeneous usage rates.''} \hl{(page XX in Section 2.1)}
}
\end{quotation}

\textcolor{red}{We hope these improvements successfully address your concerns about this paper's position in the literature and clarify our contributions to the literature.}

% ===============================================================

%%%%%%%%%%%%%%%%%%%%%%%%%%%%%%%%%%%%%%%%%

%\setlength{\itemsep}{-1pt}%
\setlength{\baselineskip}{15pt}
	\bibliographystyle{informs2014}
%	\bibliographystyle{apacite}
%	\bibliographystyle{pomsref}
	\bibliography{bib}

\end{document}
