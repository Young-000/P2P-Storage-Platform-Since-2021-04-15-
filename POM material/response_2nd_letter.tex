\documentclass[11pt]{article}
\usepackage[T1]{fontenc}
\usepackage{times}%, babel}
\usepackage{secdot,natbib}
\usepackage{latexsym, amsthm, amsmath, amssymb, color, rotating, multirow, graphicx, hhline, array, tablefootnote}
\usepackage{scalefnt, enumerate}
\usepackage{footnote}
\usepackage{threeparttable,booktabs}
\usepackage{natbib}
%\usepackage{bm, url}
%\usepackage{slashbox}
\usepackage{tikz,pgfplots}
\usepackage{subcaption}
\usepackage[margin=10pt,font=small,labelfont=bf,labelsep=endash]{caption}
\usepackage{kotex} %Korean TeX
%\usepackage[latin9]{inputenc} % kotex package와 충돌. 향후 kotex 삭제하면 될 듯.




%%% ----------------------------------------------------------------------
%\clubpenalty=10000 \widowpenalty=10000
%\renewcommand{\baselinestretch}{1.5}
% \renewcommand{\theequation}{{\rm \thesection.\arabic{equation}}}
\renewcommand{\theequation}{{\rm \arabic{equation}}}
\oddsidemargin 0in  %.10in
\evensidemargin 0in
\hyphenpenalty 2000

\textwidth 6.5in    %6.5
\textheight 8.75in     %8.5
\topmargin 0.0in      %0in
\headsep 0in \makeatletter
\newcommand{\singlespacing}{\let\CS=\@currsize\renewcommand{\baselinestretch}{1.1}\tiny\CS}
\newcommand{\doublespacing}{\let\CS=\@currsize\renewcommand{\baselinestretch}{1.5}\tiny\CS}
\newcommand{\realdoublespacing}{\let\CS=\@currsize\renewcommand{\baselinestretch}{2.0}\tiny\CS}
\newcommand{\mydoublespacing}{\let\CS=\@currsize\renewcommand{\baselinestretch}{1.499}\tiny\CS}

\newtheorem{theorem}{Theorem}[section]
\newtheorem{proposition}[theorem]{Proposition}
\newtheorem{lemma}[theorem]{Lemma}
\newtheorem{corollary}[theorem]{Corollary}

\def\N{\mathbb{N}}
\def\1{\mathbf{1}}
\def\P{\mathbf{P}}
\def\E{\mathbf{E}}
\newcommand{\maximize}{\mathop{\mbox{{\rm maximize}}}\limits}
\newcommand{\minimize}{\mathop{\mbox{{\rm minimize}}}\limits}
\newcommand{\argmax}{\mathop{\mbox{{\rm arg\,max}}}\limits}


%\newcommand\ks[1]{{\textbf{#1}}}
\newcommand{\ks}[1]{{\color{blue} #1}}
\newcommand{\ju}[1]{{\color{magenta} #1}}
\newcommand{\yj}[1]{{\color{red} #1}}


\setlength\parindent{0cm}
\setlength{\parskip}{12pt}%

\renewcommand\ttdefault{cmvtt}

%%%%%%%%%%%%%%%%%%%%%%%%%%%%%%%%%%%%%%%%%%%%%%%%%%%%%%%%%%%%
\begin{document}

\begin{center}
{{\Large \bf Response to Review Reports on POM-Jul-20-SI-0934}\\[6mm]
{\LARGE ``Sharing Economy in the Cloud:\\ Pricing Schemes for Peer-to-Peer Storage Platforms''}\\[15mm]}
\end{center}

\baselineskip 18pt

%\begin{center}
\noindent {\large \bf \underline{General Response to the Review Team}}
%\end{center}

We would like to thank Department Editor, Senior Editor, and the two reviewers for a very thorough and constructive review process. We appreciate the review team's insightful comments that have greatly helped us improve the paper. %One of the main comments that we received this time was.... lack of convincing justification on our solution concept, i.e., a mechanism design approach. Here, we first provide general response to the review team regarding this issue: why we strongly believe that a mechanism design approach could be a very effective solution concept in our setting compared to other methods such as post-order allocation schemes with / without information management. We then provide detailed response to the comments from Senior Editor and each of the two reviewers. \\[-4mm]


[Describe how we changed our paper here. For example, 1. New feature, 2. Revised feature, 3. omitted feature?.]

%\textbf{\large 1.  Our solution concept: A Mechanism Design Approach}

%To address simultaneous capacity allocation problems, Cachon and Lariviere (1999a) take a mechanism design approach and discuss on the performance of their proposed mechanisms. In this work, we also take a mechanism design approach to address capacity allocation problems in a sequential setting. We design optimal mechanisms as well as near-optimal ones with much simpler formats. In this section, we first provide a more general description of our approach.


1. New Feature of the Model

Inclusion of in-out decisions of renters and providers: To incorporate the main feature of two-side platform in the decentralized P2P file sharing system, in this revision, we consider service adoption decision of a storage renter and service entrance decision of a storage provider. 

2. New Results

2.1. Comparisons of Pricing Schemes

As we consider service adoption decisions of storage providers and renters in this revision, we investigate various pricing schemes that the P2P service platform may consider to properly incentivize both parties. Specifically, we incorporate three pricing schemes: 

1) the two-part tariff pricing scheme with a storage fee for storage capacity and a bandwidth fee for download requests for stored data. 

2) the subscription-based pricing where renters gain unlimited access to stored files after paying for the storage capacity. 

3) the hybrid pricing, which provides limited free use of bandwidth service and pay-per-use for excess downloads.

2.2. Comparisons of platform's profit and system surplus

We analyze the profit of the platform to understand its pricing decision. 

We also investigate the total surplus of the system that includes renters, providers, and the platform and see which pricing scheme enhances the social welfare.

2.3. Platform's Endogenous Operations Decisions

In addition to the platform's pricing decision, we also investigate other operations decisions of the platform.  Specifically, we analyze the impact of the profit's endogenous decision of its commission rate. Also, we examine how the platform would set its redundancy algorithm in making file fragments in storing the files. 

3. Deleted Model Feature

We removed uptime decision of renters since in practice, the platforms essentially require a sufficiently high uptime to ensure a negligible file transfer error to renters such that the P2P file sharing service is competitive to the conventional storage services.



\newpage

%
%%%%%%%%%%%%%%%%%%%%%%%%%%
%%%%%%%%%%%%%%%%%%%%%%%%%%
%%%%%%%%%%%%%%%%%%%%%%%%%%
%
\noindent \underline{\large \bf Authors' Response to SENIOR EDITOR}\\[-11mm]


%%%%%%%%%%%%%%%%%%%%%%%%%%%%%%%%%%%
\begin{quotation}
{\em
\noindent \textbf{Senior Editor: } I sent this manuscript to two experienced referees. The referees provide valuable comments. While they find the topic interesting and see some merits of the research, both referees raise major concerns. I went over the paper and deliberated for some time. On the one hand, I find the topic to be interesting. On the other hand, I agree with the referees that the contribution of the paper seems to be thin, not mentioning the other issues raised by the referees. In case that the authors could enhance the contribution and convince the review team, I would like to recommend a major revision. Meanwhile, I want to alert the authors, given the nature of the concerns, this revision will be risky.
}
\end{quotation} \vspace{-4mm}
First of all, we are happy to hear that you find the topic of this paper interesting. Also, we are grateful for the senior editor's own constructive comments. In this revision, we seriously incorporated all of your comments. We hope that you find this revision satisfactory.\\[-4mm]

%%%%%%%%%%%%%%%%%%%%%%%%%%%%%%%%%%%



%%%%%%%%%%%%%%%%%%%%%%%%%%%%%%%%%%%%%%%%%%%%%%%%%%%%%%%%%%%%%%%%%%%%%%%%%
\noindent\textbf{Main Concern: Motivation}\\[-11mm]
%%%%%%%%%%%%%%%%%%%%%%%%%%%%%%%%%%%%%%%%%%%%%%%%%%%%%%%%%%%%%%%%%%%%%%%%%

\begin{quotation}
{\em
\noindent \textbf{Senior Editor: } The authors need to provide a clear and strong motivation. It seems the paper primarily uses Sia and Storj as the motivating examples. If this is true, I have a few suggestions which might help make the motivation and background clearer.
}
\end{quotation} \vspace{-4mm}

Thank you for your comments and suggestions on the industrial backgrounds. For each of these comments, we provide detailed responses and updates in the revised manuscript as shown below.

%%%%%%%%%%%%%%%%%%%%%%%%%%%%%%%%%%%
\begin{quotation}
{\em
\noindent \textbf{Senior Editor: }  1. These two companies are introduced first time in a footnote on page 2. Then, the authors assume readers see the fine print in the footnote and start using these two example from page 3 on. This arrangement is confusing; at least, it confused me---when I saw these two names on page 3, it sounds the authors have introduced them already. After I searched these two names, I figured indeed two names have been introduced already, but in a footnote. Given the importance of these two examples, the authors should clearly introduce them in the main body of the paper.}
\end{quotation} \vspace{-4mm}

That is a great point. In keeping with your suggestion, we move the detailed description of Sia and Storj to the manuscript's main body as shown below:

\textcolor{red}{(향후 수정 예정) 산업 조사 추가 + 본문 작성 후 채울 예정}
\\[-4mm]

\begin{quotation}
{\em
\noindent \textbf{Senior Editor: }  2. More background is needed about these two motivating examples. Are P2P storage sharing platforms their main products/services? Do they make good profits from the platforms? How many users are on both sides of each platform? Who are the typical renters and providers?}
\end{quotation} \vspace{-4mm}

\textcolor{red}{[산업조사내용][재웅]} (영재가 업로드한 Notion 문서 참고) 두 회사의 최근 현황 + newly emerging services 언급

Sia\\
%https://sia.tech/ 
- Storage capacity: 3.3PB (May 26, 2021)\\
- Storage providers: 549 (May 26, 2021)\\
- Used storage: 829TB (May 26, 2021)\\
- Downloads: 1.2M (May 26, 2021)\\
%https://coinmarketcap.com/currencies/siacoin/
- Market capitalization: \$954.4 million  (May 26, 2021)\\
%https://siastats.info/storage_pricing
- Storage price (per TB and month): \$4.79 (May 26, 2021)\\
- Upload bandwidth (per TB): \$1.25 (May 26, 2021)\\
- Download bandwidth (per TB): \$1.71 (May 26, 2021)\\
- Contract formation fees (one-time fee): \$0.29 (May 26, 2021)\\
- Network fees (***): 8.54\% (of the allowance formed)\\
(***) Network fees represent an estimation of the \% of the allowance that will be paid as a fee for SiaFund holders based on the average paid by the contracts formed on the network during the last 5 days.\\
%https://siawiki.tech/wallet/siafund
- "Sia has a second cryptocurrency called the Siafund (SF). 3.9\% of all successful storage contract payouts go to the holders of the Siafunds."\\
%https://blog.sia.tech/cloud-storage-for-2-tb-mo-8a34043e93bb
- "The Sia network only expects hosts to have a 95–98\% uptime. Despite this, the Sia network is able to achieve 99.9999\% uptime for files."\\
- "Sia datacenters are restricted by the fact that they need to have access to good network connections, so we will budget 10 cents per kilowatt hour for electricity. At 500 watts per rig, this comes out to \$450 per year in electricity expenses."\\
- "The final expense is the siafund fee. All storage on the network incurs a roughly 10\% fee, which means the renter is going to be paying more than the host is earning."\\
%https://siasetup.info/learn/hosting
- "Hosts are held to several standards in order to ensure that the Sia network remains a reliable and high-quality system for renters to purchase storage space. For that reason, basic requirements to becoming a host on Sia include:" reliability, Sia prerequisites, storage (> 4TB)\\
- "If you fail to meet any of these requirements, you may be unable to host, you may not receive any storage contracts from renters, or you may even risk losing your own Siacoins as a penalty."\\
- "In general, Sia storage for renters is usually priced around the equivalent of \$2 per Terabyte per Month."\\
- "This is on the renter end, though, and since data is uploaded to 3x redundancy, host price settings will generally be about a third of that (about \$0.66/TB/Month) because a host will only have 1/3 of a renter's data."\\
- "The Sia network ranks all hosts according to an overall Host Score. This score is based on a number of metrics, listed below." (Host Uptime, Storage Pricing, Collateral, Storage Remaining, Host Age, Interaction Weight, Version Adjustment)\\
- "Host uptime is one of the most important metrics. If you're not online when your renters need their data, you're not doing them very much good. You're allowed a small amount of downtime in order to address minor maintenance issues (restarting for updates, etc) which amounts to approximate 14 hours per month, but in general you should plan for your hosting computer to be turned on and online 24/7. If you can't commit to this, you shouldn't try to host on the Sia network."\\
- "If you go offline for too long (less than 95\% uptime) or lose renter data (by deleting it or experiencing a hardware failure), you can lose money by losing your collateral for active contracts. You can also become responsible for SiaFund fees for each contract."\\
- "Greater than 98\% uptime results in no penalty, which is the 14 hours a month explained above (2\% of 720 hours in a month = 14 hours)."

Storj\\
- Previous Storage (Bandwidth) Price: \$0.015 (\$0.05) per GB per month (per GB downloaded) (2018-02-19)\\
%https://www.storj.io/pricing
- Current Storage (Bandwidth) Price: 150GB free for storage and bandwidth, respectively + \$4 per TB (\$7 per TB)\\
%https://coinmarketcap.com/currencies/storj/
- Market capitalization: \$370.2 million (2021-05-26)\\
%https://www.prnewswire.com/news-releases/storj-labs-introduces-new-integrations-as-decentralized-cloud-storage-market-heats-up-301175974.html
- "Tardigrade is the world's first decentralized cloud object storage solution backed by enterprise service level agreements (SLAs). The service launched to the general public in March 2020.  Since the general launch, Storj Labs has grown the network's capacity from 22 petabytes to 45 petabytes and Tardigrade users from 3,600 to 11,200—an increase of more than 300 percent. Today, there are 171 million files stored on Tardigrade."
%https://support.storj.io/hc/en-us/articles/360026612272-What-are-the-requirements-for-a-Storage-Node-on-V3-
- Minimum (preferred) requirement for uptime: Online and operational 99.3\% of the time per month or MAX total downtime of 5 hours monthly (Online and operational 99.5\% of the time per month)\\
- Other requirements: Processor, available space, bandwidth per month, bandwidth up/downstream\\
%https://edumontoya.medium.com/storj-review-pros-cons-alternatives-23341db52acd
- No free version, previously: "First of all, it does not have a free version that many of the other alternatives do. These days, with so much information floating around the Internet, if you don’t give a free trial, many potential users will first go looking for reviews. Aren’t you reading this without being registered?" (2020-11-09)\\
%https://www.storj.io/storj2014.pdf
%https://www.storj.io/storjv2.pdf
%https://www.storj.io/storjv3.pdf
- As the technical level of Storj matures, their interest has shifted from the success of transmission through erasure coding to keeping the provider on the platform and improving the service experience.

Other companies - Filecoin\\
%Previous manuscript
- More recently, in 2017, Filecoin raised \$257 million via its initial coin offering (ICO): breaking the previous record for ICO funding.\\
%https://file.app/
- 707 providers, 537 clients (2021-05-27)\\
- "Filecoin is 0.38\% the cost of Amazon S3 - Infrequent Access per GiB per month." (2021-05-27)\\
- 439 filtered miners with 3.76 PiB of total storage size\\
%https://coinmarketcap.com/currencies/filecoin/
- Market capitalization: \$1.078 billion (2021-05-26)\\
%https://filecoin.io/filecoin.pdf
- Decentralized storage network, novel proofs-of-storage (proof-of-replication \& proof-of-spacetime), verifiable markets (the storage market, the retrieval market), useful proof-of-work (miners can participate in the creations of new blocks for the underlining blockchain)\\
%https://filecoin.io/blog/posts/how-storage-and-retrieval-deals-work-on-filecoin/#:~:text=In%20order%20to%20store%20files,is%20transferred%20to%20a%20miner.
- "When a fault is registered for a sector, the Filecoin network will slash the storage miner that is supposed to be storing the sector; that is, it will assess penalties to the miner (to be paid out of the collateral fronted by the miner) for their failure to uphold their pledge of storage."
%https://docs.filecoin.io/mine/mining-rewards/#storage-fees
- Reward I (Storage fees): "PoSt (Proof-of-Spacetime) window checks are performed on 24 hour intervals across the network to ensure that miners are continuing to host their required sectors as normal."
- Reward II (Block rewards): "Block rewards are large sums that are given to the miner credited for a new block. Unlike storage fees, these rewards do not come from an associated client; rather, the network "prints" new FIL as both an inflationary measure and an incentive to miners advancing the chain."
- Reward III (Verified clients): "To further incentivize the storage of "useful" data over simple capacity commitments, storage miners have the additional opportunity to compete for special deals offered by verified clients."
- Reward IV (Retrieval fees): "Retrieval fees are paid incrementally using payment channels as the retrieval deals are fulfilled (by sending portions of the data to the client. This happens off-chain."

We summarize this investigation in the revised manuscript as follows:

\textit{"This is how we write in the manuscript."}

\begin{quotation}
{\em
\noindent \textbf{Senior Editor: }  3. More importantly, the authors should provide more details regarding the pricing schemes used by these two platforms. Do they use this two-part tariff consisting of a unit storage fee and per-unite access fee? Are the two parts indeed in a unit-price format? Do the platforms take the commission as what has been model in the paper?}
\end{quotation} \vspace{-4mm}

\textcolor{red}{[산업조사내용][재웅's turn]} (영재가 업로드한 Notion 문서 참고) Storj의 two-part tariff -> 최근에 hybrid 형태로 변한 내용 서술 + Sia도 two-part tariff의 형태로 부과 및 fixed commission rate 부과 언급

[Clarify this based on their white papers. The pricing scheme should be clearly explained and we also need to clarify that our objective is to consider the firm's best pricing under this two-part tariff structure. I assume that our answers to all of the above questions are YES.]
\yj{(youngjae to Prof. Park 2021-04-15) More precisely, there are more types of pricing available on both platforms. For example, Sia is charging for both uploads and downloads bandwidth fee. And there is also a difference between Sia and Storj for pricing policies within two-part tariffs structure. Storj provides services with a fixed price, but Sia is provided in the form of a bidding. Also, Filecoin has a quite different pricing structure. Filecoin also adopts two-part tariff but it is consisting of a storage price and reterive price (need to check more). }-> \ks{각 회사의 form을 ps pb로 일반화해서 표현할 수 있는지 검토해보고, 안되는 종류는 따로 discussion하는 부분에서 다루면 좋을듯.}


\begin{quotation}
{\em
\noindent \textbf{Senior Editor: }  4. Regarding the consumer-welfare maximization, the authors need to provide good motivation. Using Uber as a motivation is a stretch, because Uber is in very different business and its pricing scheme is different from what is studied in this paper. In addition, as R2 correctly points out, ``for a two-sided marketplace like this, it is equally important to consider the welfare of the providers.}
\end{quotation} \vspace{-4mm}

\textcolor{red}{[Motivation][재웅's turn]} 리뷰팀 얘기를 듣고 문제를 바꿨음. 우리는 다양한 pricing scheme들을 비교하는 방향으로 바꿨고, 각 scheme이 platform, provider, renter, 그리고 social welfare에 미치는 영향을 분석하였음. 특히 P2P market이 형성된지 얼마 안된 two-sided market이라는 점에서, 잠재적 provider가 얼마나 존재하는지에 따라 가격 체계가 미치는 영향이 어떻게 달라지는지를 파악하기로 함. 그리고 주어진 scheme 하에서 platform은 profit maximizing을 하는 것이 보다 현실적인 가정이라고 생각하였음.

%[Revise the paper such that we incorporate both the providers and the renters from the perspective of the platform. Specifically, the platform may seek to maximize its profit or intentionally seek to expand the size of the market (both numbers of providers and renters). We want to understand the optimal pricing of these two different strategies of the platform. We, however, do not intend to incorporate them in the objective function of the platform since determining when to switch from one strategy to another can be a complex problem and is may be worth considering this as a separate study.]
%  \yj{2021.04.22
%\begin{itemize}
%    \item 기존에 growth strategy의 일환으로 consumer-welfare maximization을 제시(provider welfare는 고려하지 않음
%    \item 현재의 revision에서는 welfare maximization은 다루지 않고, 기존의 profit-maximization에서 $\gamma$라는 유휴공간 확보라는 것을 constraint로 넣으며서 growth strategy로 해석하는 방향으로 수정을 진행함
%    \item 하지만, 밑의 reviewer 2에서도 welfare maximization에 대한 comment가 있어서 이 부분에 대한 문제를 다루어야 할지에 대한 협의 필요
%\end{itemize}
%}-> \ks{$\gamma$를 $\alpha$로 정리하기로 했음}
%%%%%%%%%%%%%%%%%%%%%%%%%%%%%%%%%%%  



%%%%%%%%%%%%%%%%%%%%%%%%%%%%%%%%%%%%%%%%%%%%%%%%%%%%%%%%%%%%%%%%%%%%%%%%%
\noindent\textbf{Main Concern: Contribution}\\[-11mm]
%%%%%%%%%%%%%%%%%%%%%%%%%%%%%%%%%%%%%%%%%%%%%%%%%%%%%%%%%%%%%%%%%%%%%%%%%

\begin{quotation}
{\em
\noindent \textbf{Senior Editor: } Based on my own reading, the takeaways from the current analyses are limited. Echoing the reviewers’ comments, I also think that the contribution is thin. The authors need to think through what are the main interesting results and articulate the contribution. In particular, what are the main insights and surprising findings? The current discussion on the contribution has been either on technical side or over claimed. To move forward, the authors have to think deeper about what insights they aim to deliver and make a case regarding the contribution.
}
\end{quotation} \vspace{-4mm}

\textcolor{red}{[Contribution][재웅's turn]} 리뷰팀의 의견을 검토하여, P2P storage platform이 취할 수 있는 다양한 pricing scheme에 따라 각 stakeholder들에게 어떠한 영향이 있는지, 그리고 P2P market의 특성상 provider의 잠재적 크기에 따라 이러한 영향은 어떻게 달라지는지 살펴보았다.

우리는 잠재적 provider의 규모에 따라 socially optimal pricing scheme이 달라지는 것을 확인하였다. Provider가 충분할 때에는 소비자들에게 무료로 이용할 수 있는 bandwidth service를 많이 제공하는 가격체계가 더 높은 welfare를 만들어냈지만, provider가 충분히 많지 않을 때에는 그러한 가격체계가 bandwidth demand가 높은 renter들만을 끌어들이게 되어 provider에게는 더 큰 부담이 되고, 이에 따라 platform이 더 높은 가격을 매기게 되어 사회적으로 더 많은 비용을 초래하는 것을 확인하였다. 결과적으로, 잠재적 provider가 낮은 경우 two-part tariff 혹은 적은 무료 용량을 제공하는 hybrid pricing이 사회적 최적과 플랫폼의 profit maximization을 동시에 달성하는 것을 확인할 수 있었다.

%[In this regard, the general response in the front page of this response should include how our model changed and why with new useful insights.]\\
%\yj{2021.04.22
%\begin{itemize}
%    \item 우리가 주고싶은 insight에 대한 고민 필요. 현재 논문에서 다루는 내용은 
%    \item 1. pricing에 따른 renter의 in-out decision / provider의 in-out decision
%    \item 2. platform의 profit maximization pricing schemes(potential renter와 provider이 있다는 가정 아래) 
%    \item 어떤 항목에 따른 optimal pricing을 다룰 것인가에 대해서도 더 좋은 insight를 위해 고민 필요 - 현재는 potential renter와 provider의 비율에 따른 pricing의 변화를 다루었음.
%    \item 3. platform이 유휴공간을 확보하는 constraint를 둘 때($\gamma \theta v_{rs}^t \le v_{ps}^t$)의 optimal pricing scheme
%    \item 이외의 다룰 수 있는 내용
%    \item - $\alpha$에 따른 변화(review comment 중 있음)
%    \item - consumer welfare or total welfare maximization
%\end{itemize}
%\begin{itemize}
%    \item reviewer 2가 reference로 제시해 준 논문들에서 제공하는 insight들 확인 중 (미팅 때 설명 예정)
%\end{itemize}
%}

%%%%%%%%%%%%%%%%%%%%%%%%%%%%%%%%%%%%%%%%%%%%%%%%%%%%%%%%%
\begin{quotation}
{\em
\noindent \textbf{Senior Editor: } 1. The claimed contribution that “we extend the existing studies in the literature by proposing optimal pricing schemes for multi-objective platforms” does not sound right to me.

(a) First, I do not see multi objectives. The objective can be the weighted average of the consumer welfare and platform profit. But, that is still one objective. This claim on “multi-objective” is confusing.
}
\end{quotation}\vspace{-4mm}

\begin{quotation}
{\em
\noindent \textbf{Senior Editor: }
(b) Second, as explained, the consideration of consumer welfare needs some motivation. Why are we interested in the consumer welfare, not the social welfare of the system, or the welfare of both sides? The consideration on consumer welfare is not unusual. For example, in "Recommender Systems Rethink: Implications for an Electronic Marketplace with Competing Manufacturers," the authors also consider an objective with both consumer welfare and platform profit in mind, but it comes with a good motivation.
}
\end{quotation}\vspace{-4mm}

\begin{quotation}
{\em
\noindent \textbf{Senior Editor: }
(c) The paper highlight that “the optimal pricing scheme of a multi-objective platform systematically deviates from a simple linear mixture of the two single-objective pricing schemes.” In other words, the paper implies that the conventional approach is to use “the simple linear combination of the profit-maximizing pricing and the pricing that maximizes consumer welfare always” and concludes that the conventional approach yields suboptimal outcomes. Why/how do the authors think or assume this is the conventional approach? If anything, this is a naïve approach. Essentially, the authors say the optimal approach is better than the naïve approach.
}
\end{quotation}\vspace{-4mm}

\textcolor{red}{[Motivation][재웅's turn]} 리뷰팀의 코멘트를 듣고 다시 검토해보니 profit maximization이 platform의 objective로 보는 것이 적절하다고 판단하였다. 이에 consumer welfare를 함께 objective로 삼는 플랫폼을 고려하는 대신, profit-seeking 하는 플랫폼의 의사결정이 결과적으로 소비자들에게 어떤 영향을 미치는지를 보는 것으로 방향을 선회하였다.

[Explain how we changed our model to address these concerns. Then, (a) and (b) will be addressed. We can then claim that (c) is not valid any more. ]


%%%%%%%%%%%%%%%%%%%%%%%%%%%%%%%%%%%%%%%%%%%%%%%%%%%%%%%%%

\begin{quotation}
{\em
\noindent \textbf{Senior Editor: } 2. I have to point out that this paper does not contribute anything to blockchain technologies. I am fine with the authors using blockchain as a motivation, story, or background. However, this paper bears no connection with blockchain in a sense that if we remove blockchain from the current paper it is still the same paper. We can argue that blockchain improves whatever business (e.g., online banking or rider-sharing), but blockchain does not play a central role in the current model or analyses. The authors may not want to over-claim the contribution along blockchain.}
\end{quotation}\vspace{-4mm}

\textcolor{blue}{[Motivation][영재's turn]} 새로 작성 예정(not yet)인 introduction을 아래에 작성한 관점들을 기반으로 검토 바람. (Literature review에서는 blockchain 내용 삭제 예정)

[remove block chain part. This was added just to fit better for the topic of special issue and we no longer need this]

\yj{2021.04.22
\begin{itemize}
    \item 현재 논문에서 다루는 내용에는 blockchain에 관련된 요소가 크게 필요하지는 않음
    \item 하지만, sia 및 storj 외의 decentralized storage platform을 제시할 때는 조심해야 할 필요가 있음
    \item 기본적으로 business가 엄청나게 빠르게 발전하고 있어, 비교적 비슷한 시기(?)에 출범되고 현재 운영이 되고 있는 sia와 storj는 우리 논문에서 다루는 것과 비슷한 형태로 운영되고 있으나 당장 새로 생겨나고 있는 business들만 하더라도 새로운 형태의 service들이 많음
    \item 전체적으로 기술이 다르지만 여러 사례를 하나로 묶기 위해서는 다 같은 decentralized storage platform이다 라는 요소와 blockchain을 기반으로 만들어지고 있는 storage platform의 새로운 버전이다 라는 형태가 되어, blockchain에 대한 언급이 빠질 수가 없음
    \item sia와 storj에 대해서 조금 더 집중하여 사례를 풀어나갈지, 조금 더 전체적인 decentralized storage platform market을 다루면서 blockchain과 관련된 내용을 introduction에서라도 언급을 하고 지나갈지에 대해서는 결정 필요 
\end{itemize}
} \ks{완전히 제거하기 보다는 blockchain덕분에 decentralized storage platform이 활성화 되었다는 선에서 언급.}

\begin{quotation}
{\em
\noindent \textbf{Senior Editor: } 3. The explanation for the different practices of Sia and Storj needs to be cautious. Because these two platforms compete with each other, one more natural explanation is that competition leads to differentiation in their price schemes. This work cannot tease out this alternative explanation or conclude the explanation offered in this paper.}
\end{quotation}\vspace{-4mm}

[Update their pricing scheme and if they are still different, try to be modest such that there are several other reason why they end up with different pricing scheme. Since we don't need to stick to differences of pricing schemes, we can choose not to emphasize this in this revision.
]

\textcolor{red}{[산업조사내용][재웅's turn]} Thank you for the constructive comment. 리뷰해준대로 기존 플랫폼의 pricing에 대해 신중하게 접근하기로 함. 아직 이 산업이 emerging하는 단계이기 때문에 여러 플랫폼들이 등장하고 pricing 정책을 변경하는 등 시행착오를 진행하고 있음. 한편 초창기에 등장한 Sia 및 Storj의 경우 storage fee 및 bandwidth fee를 부과하는 two-part tariff 및 이의 변형된 형태를 adopt한 상황이다. 이에, 기존의 플랫폼들이 현재 설정하고 있는 가격을 비교하고 평가하기 보다는, 이러한 시장이 선택할 수 있는 여러 pricing scheme (P2P cloud에서 현재 쓰고 있는 + centralized cloud에서 쓰였던)의 선택이 
플랫폼의 profit 및 social value 창출에 어떠한 영향을 미치는지를 이해하는 것으로 초점을 바꾸었다.

\yj{(youngjae to Prof. Park 2021-04-15) I think it would be better not to emphasize the differences between the two platforms. }
\yj{2021.04.22
\begin{itemize}
    \item 현재 플랫폼들의 종류가 너무 많아지며 각 플랫폼들의 pricing을 모두 다루게 되면 오히려 애매할 수가 있음
    \item 아직은 immature한 business형태라서 그런지는 모르겠는데, 전체 decentralized storage platform들이 자신들끼리 경쟁한다기보다는 storage platform과 비교하여 가격, technology 차원에서 어떤 요소들이 더 좋은지 이야기하는 형태가 더 강함(하지만 직접적으로 경쟁을 안한다고 말하기는 reference를 찾기가 힘들어 만약 이와 같은 서술을 할 때는 조심해야 할 필요가 있음)
\end{itemize}
}\ks{두 회사가 실제로 많이 다른지? 초창기에는 다르다가 현재는 비슷하게 가고있다면 경쟁에 의한 differentiation이 중요한지 대답할 수 있음.}\\
\yj{sia와 storj만을 비교한다면 초창기부터 지금까지 플랫폼의 pricing 차원에서는 비슷하다고 볼 수 있습니다. 가장 큰 차이점은 초창기에는 storj는 renter는 달러단위의 fixed price 지불, provider는 coin으로 보상 획득 / sia는 provider가 자신의 가격을 coin단위로 bidding 형태로 지정해서 제공 / renter는 coin 단위로 지불 정도였었습니다. 그 이외의 차별점은 전부 technology 영역으로써 어떠한 ui를 제공할 지, 그리고 플랫폼에서 제공하는 서비스들의 안정성이나 편의성등을 어떠한 기술을 통하여 더 높일지를 경쟁하는 느낌이 강합니다. }

\begin{quotation}
{\em
\noindent \textbf{Senior Editor: } 4. Also, I do not see how this study sheds light on other types of platforms. The current discussion is loose and could be misleading. Instead, the authors should discuss in a more serious manner.
(a) First, the authors should identify and clearly present the unique features with the practices of these P2P storage sharing platforms. For example, are the two-part tariff and unit price in each part unique here for two-sided platforms?
}
\end{quotation}\vspace{-4mm}

\textcolor{blue}{[Contribution][영재's turn] (말이 되는지 검토 + 아이디어 추가 바람)} (1) Cloud platform의 의사결정에의 기여: 기존에 Cloud platform이 활용하던 pricing 체계들이 P2P 기반의 시장에서 어떠한 영향을 주는지에 대해 처음으로 밝혔음. 

(2) Sharing economy에 대한 기여: 많은 논문들이 pay-per-use 혹은 subscription에 대해 다루었고, two-part tariff에 대한 sharing economy 연구는 많지 않았음. 우리 논문에서 다룬 two-part tariff vs. subscription vs. hybrid pricing에 대한 시사점은 이동 거리에 비례해 추가 요금을 지불하게 되는 ride sharing 플랫폼과 같은 상황에서도 적용될 수 있다.

다만 우리의 결과를 적용함에 있어 신중한 접근이 필요한데, 이는 P2P storage market의 다음과 같은 특징이 모형에 반영되어 있기 때문이다: 최초 storage service의 계약은 on-demand로 이루어지지만, 이후의 계약은 not temporary but continuous, storage contract가 맺어진 상황에서 bandwidth service가 제공되지 않았을 때의 loss가 기존의 sharing economy보다 critical (provider 편할 때에만 제공 X), redundancy의 도입을 통해 service level을 유지함으로써, 알고리즘 부분의 이유로 demand보다 더 많은 supply가 필요하며, 여러 provider가 동시에 서비스를 제공해는 특징이 있다.

위의 서비스 특징으로 인해 idle resource가 간헐적으로 발생하는 provider들이 시장에 진입할 수 없도록 시장이 설계되어 있으며, 한 단위의 demand를 충족 시키는 과정에서 얼마나 여러 provider가 필요하냐에 따라 개별 provider가 감당하는 operating cost가 달라진다(numerical analysis). 

\yj{20210510 to 재웅}
(a), (b), (c)에 대한 답변 방향을 아래와 같이 정해야할 듯 함\\
(a) - business에 대한 background에 조금 더 집중\\
(b) - 실제로 다른 sharing economy literature에서 다룬 모델 파트와 cost term에서의 차이점을 비교 강조(추후 진행 - 영재)\\
(c) 재웅이형이 위에 써놓았는 (2)와 관련된 내용(타 business에 간접적으로는 implication이 있으나, sharing economy의 성질이 조금 달라 신중하게 접근해야한다)\\
위에서 쓴 contribution들이 좋은 내용이라 생각이 되는데, 이 질문에 대한 답변으로 좋은지는 위치 고민 필요

\yj{2021.04.22
\begin{itemize}
    \item 기존 버전에서는 논문의 정체성이 지금보다 더 모호했던 느낌이어서 타 플랫폼으로의 implication을 제시하였는데, 현재 버전에서는 decentralized storage platform 자체에 대해서 집중하는 것이 더 필요할 것 같음.
    \item 현재 타 sharing platform(p2p product sharing platform / uber, lyft, airbnb등의 플랫폼)과 비교해서 p2p storage sharing platform만이 가지고 있는 unique feature를 정리해보면 다음과 같음 
    \item 기존의 sharing platform은 정말 '사용하지 않는 것'을 '사용하지 않는 시기'에 share해주는 반면, p2p storage sharing platform은 '사용하지 않는 것(용량)'을 share하지만, renter들이 빌린 것을 이용하는 시점은 renter의 own decision이 됨.
    \item 또한, renter가 빌린 것을 이용하는 기간이 기존의 sharing platform에 비해 long term이지만 간헐적이라고 할 수 있음
    \item 즉, renter이 상시로 빌린 용량을 이용할 수 있기에, platform은 provider에게 uptime이라는 requirement를 부여할 수 밖에 없음.(redundancy라는 방법도 하나의 해결 방법)
    \item redundancy를 부여하는 이유는 크게 두 가지로 정리 가능 - 1. availability, renter이 자신들의 파일을 이용할 때 access할 수 있어야 함. 2. durability, renter의 파일을 저장하고 있는 provider가 갑자기 이탈(provider로써의 역할을 안함)하게 되었을 때도 파일의 손실이 발생하지 않음
    \item provider는 uptime과 동시에, renter들이 저장한 용량을 이용하게 되었을 때 computing power를 비롯한 cost가 발생할 수 있음 - 다른 coin들을 채굴하는 느낌으로 정말 안쓰는 컴퓨터를 빌려주는 것이라는 전혀 상관이 없지만, 만약 실제로 사용하는 컴퓨터의 남는 공간을 빌려주게 된다면 이로 인한 cost가 충분히 발생 가능.
    \item 다른 unique feature로는 computing resource인 만큼 훨씬 더 renter 및 provider의 in-out decision이 real time으로 나타남. -> 이 부분이 현재 우리는 given potential provider and renter을 바탕으로 문제를 풀지만 유휴공간($\gamma$)를 어느정도 고려하는 것이 필요하다 라는 이야기를 할 수도 있을 것이라 생각.
\end{itemize}
}

\begin{quotation}
{\em
\noindent \textbf{Senior Editor: } 
(b) Second, I would encourage the authors to think deeper about unique features of this specific setting and what new insights these unique features lead to. Further, as suggested by R2, the authors should “highlight the distinct modeling features of the P2P storage services, as compared to the sharing economy models and the P2P files sharing services.” Doing so might help authors to properly position the paper and bring out the contribution.
}
\end{quotation}\vspace{-4mm}

\textcolor{blue}{[Contribution][영재's turn] (Senior Editor의 4(a)(b)(c)가 어떻게 나뉘어서 답변되면 좋을지 아이디어 부탁함)} 

\yj{2021.04.22 
\begin{itemize}
    \item 타 논문 model feature
\end{itemize}
}
\begin{quotation}
{\em
\noindent \textbf{Senior Editor: } (c) Third, if the authors want to discuss the implications of this study to other types of platforms, the authors need to ensure it is indeed applicable. The authors should avoid loose discussion.
}
\end{quotation}\vspace{-4mm}

\textcolor{blue}{[Contribution][영재's turn] (Senior Editor의 4(a)(b)(c)가 어떻게 나뉘어서 답변되면 좋을지 아이디어 부탁함)} 

\yj{2021.04.22 
\begin{itemize}
    \item 타 platform에 대한 implication을 아예 이번 논문에서는 배제하고 현 platform에 집중하는 것도 좋은 방법으로 생각됩.
\end{itemize}
}
\begin{quotation}
{\em
\noindent \textbf{Senior Editor: } 5. I agree with R2 that that paper focuses on two-part tariff and cannot claim optimal pricing schemes in general. To establish the relevance and importance, a strong motivation should be in place, as I explained in the comment on motivation.}
\end{quotation}\vspace{-4mm}

\textcolor{red}{[Contribution][영재's turn] (추가하고 싶은 얘기 있으면 작성 바람)} R2's comment를 따라 다양한 pricing scheme을 비교하는 방향으로 논문을 수정함. 다른 cloud platform에서도 활용되는 subscription 및 hybrid pricing (literature) (R2 추천)를 초창기 플랫폼인 Storj 및 Sia에서 쓰이는 two-part tariff와 비교함으로써 플랫폼에게 어떠한 scheme이 유리한지, 이때 창출되는 social welfare는 어떻게 다른지를 논문의 focus로 하였다.\\
\textcolor{blue}{YJ} 위의 내용은 좋은듯. 지난 번에 미팅할 때 나온 이야기기는 한데, 우리가 모든 pricing scheme를 비교하지는 않았지만, 'sharing economy'이기도 하지만 'cloud storage platform'이기도 한 이 플랫폼의 성질을 살려 기존의 플랫폼들이 채택하고 있는 가격체계를 도입했을 때 어떠한 영향을 미칠 수 있는지, 쪽으로 방향을 확실히 정해서 writing하면 좋을 듯


%%%%%%%%%%%%%%%%%%%%%%%%%%%%%%%%%%%%%%%%%%%%%%%%%%%%%%%%%%%%%%%%%%%%%%%%%
\noindent\textbf{Main Concern: Model and Analysis}\\[-11mm]
%%%%%%%%%%%%%%%%%%%%%%%%%%%%%%%%%%%%%%%%%%%%%%%%%%%%%%%%%%%%%%%%%%%%%%%%%


\begin{quotation}
{\em
\noindent \textbf{Senior Editor: } Both reviewers have questions and comments on the model assumptions and analyses, which I do not repeat her. In addition to the reviewers’ comments, I have the following two specific suggestions:\\

\noindent 1. All the model assumptions should be presented in the model section. Currently, some model assumptions are presented in the analysis part. These assumptions include that V and lambda are independent and their distribution are publicly known, providers are homogeneous and independent, and V and lambda are uniformly distributed.
}
\end{quotation}\vspace{-4mm}

\textcolor{blue}{[Model][영재's turn] (더 쓰면 좋은 말이 있을지?)} That is a great point. In keeping with your comment, we move all assumptions to Section 2 ("Model Description"). 또한 모델 세팅 및 가정들이 변경된 부분을 반영하였다.
\textcolor{red}{YJ} 이 부분은 위에서 쓰신 것과 같이 그냥 코멘트를 반영하였다 수준으로만 해도 충분하지 않을까요?\\

%%%%%%%%%%%%%%%%%%%%%%%%%%%%%%%%%%%  

\begin{quotation}
{\em
\noindent \textbf{Senior Editor: } 2. It will be helpful if the authors could clearly describe the timing of the game.
}
\end{quotation}\vspace{-4mm}

\textcolor{blue}{[Model][영재's turn] (구체적으로 서술 부탁!)} Thank you for this comment. Following your suggestion, we add the sequence of events with a time line in Section 2. Specifically, (timeline 내용)

\textcolor{red}{YJ}\\
모든 player는 forward looking하여 rational하게 의사 결정 -> 우리는 backward로 문제를 풀 예정\\
0. Platform이 어떤 pricing 체계를 쓸지 결정(이걸 timing of the game에 넣을지 말지는 고민 필요.)\\
0. Platform이 required uptime 및 redundancy를 failure가 사실상 발생하지 않는 수준으로 설정한다(이 것도 timing of the game에 넣을지 고민 필요)\\
1. Platform이 한 pricing 체계 내에서 optimal storage price and bandwidth price 설정\\
2. Provider는 required uptime 및 redundancy로 부터 발생할 cost와 platform에 in 하였을 때 얻을 수 있는 profit을 비교하여 platform에 in할지 out할지 결정\\
3. Renter는 failure probability($\approx 0$)와 storage price 및 bandwidth price를 확인한 후 자신의 utility 및 frequency에 따라 platform in-out 결정\\
* failure probiability는 renter utility function에 반영하여 genral하게 잡았다가 0으로 근사할 수도 있고, 아예 처음부터 무시하는 형태로도 서술 가능\\

\ju{(Jaeung to YoungJae 2021-04-22) 코멘트 위치 확인바람.
}

\yj{(youngjae to Prof. Park 2021-04-15) Previously, we adopted two-sided market since provider's uptime was very important role. However, in this model, the decision of the provider(in-out decision) does not affect the renter's decision. We need to figure out our model's positioning.}
\yj{2021.04.22
\begin{itemize}
    \item 위에 대한 이유를 조금 더 구체적으로 남기면, 현재 business의 발전에 따라 platform 자체를 이용할 때 renter들이 느끼는 차이(기존의 centralized platform인 dropbox 등과 비교하였을 때)는 갈수록 없어짐
    \item 실제로 platform들을 이용해본 경험상으로 provider로써 in을 하는 것은 상당히 복잡도가 높고, requirement도 많지만, renter로써 platform을 이용하는 과정은 엄청 간단해지고 불편함을 느낄만한 요소는 ui등의 차이 뿐임 - decentralized storage platform의 한계점이 될 수 있다고 제시되었던 availability와 durability 등은 더 이상 큰 문제가 아님
    \item 따라서, renter의 입장에서는 provider의 용량만 있다면 storage platform을 이용하는데 느끼는 quality 차이가 없다고 본 논문에서 가정
    \item quality의 차이가 없기 때문에, renter에서 느끼는 차이는 결국 이 platform을 이용할 수 있느냐 없느냐 뿐인데, 이 요소는 renter의 utility function에 들어가지 않음 -> 따라서, 현재는 two-sided platform이라고는 하지만, renter의 decision에서 provider의 decision이 반영되는 요소가 없어짐
    \item Note that, 기존의 논문에서는 renter의 in-out decision -> provider의 uptime decision / provider의 uptime decision -> renter의 in-out decision이라는 쌍방향의 작용이 고려되었음.
\end{itemize}
}
%%%%%%%%%%%%%%%%%%%%%%%%%%%%%%%%%%%%%%%%%%%%%%%%%%%%%%%%%



\newpage
%%%%%%%%%%%%%%%%%%%%%%%%%%%%%%%%%%%%%%%%%%%%%%%%%%%%%%%%%%%%%%%
\noindent \underline{\large \bf Authors' Response to Reviewer 1}
%%%%%%%%%%%%%%%%%%%%%%%%%%%%%%%%%%%%%%%%%%%%%%%%%%%%%%%%%%%%%%%


%%%%%%%%%%%%%%%%%%%%%%%%%%%%%%%%%%%%
\begin{quotation}
{\em
\noindent \textbf{Reviewer 1: }
The paper studies the pricing scheme under an emerging context of sharing economy, i.e., the P2P storage platforms. The paper proposes models to describe the renters, providers, and the platform's decisions. Equilibrium is analyzed under profit- and consumer welfare- maximization schemes as well as their combinations. 

The topic is very interesting and I read it with great interest. The paper indeed identifies an innovative problem under the sharing economy context. I list my major concerns (mainly on the modeling perspectives):
}
\end{quotation}\vspace{-4mm}
We are happy to hear that you find our topic very interesting. Your comments helped us better approach our model for the sharing economy platform. 
%%%%%%%%%%%%%%%%%%%%%%%%%%%%%%%%%%%%


%%%%%%%%%%%%%%%%%%%%%%%%%%%%%%%%%%%%
\begin{quotation}
{\em
\noindent \textbf{Reviewer 1: } 1. Penalty and compensation when the renters cannot access to the file are missing. For the renters, if, by some chance, they cannot access the file due to fewer numbers of online providers than required, it leads disastrous user experience. So, the model should consider penalty and compensation in such a case. Namely, a penalty on the platform, and a generalized downside cost of the renters to capture the negative impact of access deny when they need the file.
}
\end{quotation}\vspace{-4mm}

\textcolor{red}{[Model Setting][재웅's turn]} This is a great point. We investigate recently updated policies of P2P storage platforms and found that (...) 굉장히 높은 수준의 uptime을 요구하고 있고(아래 내용), 이를 유지하지 않는 provider들에게 강력한 페널티(예: requiring deposit, reducing assignment probability)를 부과하고 있다. 또한 이러한 uptime checking이 실제 contract 및 service requirement가 이루어지기 전에도 ping을 쏘는 등의 기술적인 방법을 통해 가능하기 때문에, 사실상 renter와 contract를 맺을 때에는 이미 선정된 provider들만 platform에 남아있게 되는 구조이다.

[Clarify which one is correct: 1. we incorporate penalty for missing a file in the renter's utility or 2. we assume that the platform requests the provider a sufficiently high uptime so this issue can be ignored.]

\textcolor{blue}{(Jaeung, 2021-04-15) In the revised version, we assume the second.}\\
\yj{(youngjae, 2021-04-15) Platform sets the required uptime of provider to satisfy target probability $\bar{p}$} 
\yj{2021.04.22
\begin{itemize}
    \item storj's required uptime -> 99.3\% ( Is online and operational no less than 99.3\% of the time per month;)
    \item sia's required uptime -> 95\% (If you go offline for too long (less than 95\% uptime) or lose renter data (by deleting it or experiencing a hardware failure), you can lose money by losing your collateral for active contracts. You can also become responsible for SiaFund fees for each contract.)
\end{itemize}
}
%%%%%%%%%%%%%%%%%%%%%%%%%%%%%%%%%%%%

\begin{quotation}
{\em
\noindent \textbf{Reviewer 1: } 2. If I read the paper correctly, the parameter $k$ is assumed to be exogenous in the model, which I do not agree with. For example, on page 14, when the paper models the provider utility, $k$ should depend on the equilibrium that how many providers participate and how are their active levels ($t^*$), and hence depend on the number of providers to store the data as a decision by the platform. Intuitively, fewer providers participation means a smaller $k$, and less active (smaller $t^*$) of providers induces a larger $k$ (to increase the robustness and access).
}
\end{quotation}\vspace{-4mm}

\textcolor{blue}{[Model Setting][영재's turn]} We appreciate for your constructive comment. 지적한대로 플랫폼이 이러한 algorithm parameter를 직접 결정할 수 있다. 해당 부분은 closed form으로 풀어내기 매우 어렵기 때문에, numerical analysis를 통해 시사점을 다음과 같이 이끌어내었다(결과 필요)

또한 모델의 가정이 달라지는 과정에서, redundancy와 uptime 간의 관계는 다음과 같이 수정되어 반영되었다(내용 필요)

Thank you for your comment. In this revision, we incorporated your comment by making $k$ as a platform's choice. However, since the platform currently requests sufficiently high uptime for providers to participate into the platform, we decided not to include $t$ as the decision variable any more. [correct? Also, add how $\theta$ in the new model plays a role in this context.]
\yj{(youngjae to Prof. Park 2021-04-15) Currently, the meaning of $k$ is the number of files that are split, and the meaning of $n$ is the number of split files that are distributed. If we change the reviewer's comment to $n$ instead of $k$, then it is correct. We are going to treat this issues by using numerical study. In my opinion, it would be good to fix $k$(maybe it depends on the technology levels) and analyze it with trade-off between $n$ and $t$. As you said, $\theta$ is very important role but it is calculated by $\theta = \frac{n}{k}$ and even if $\theta$ is equal, required uptime $t$ depends on the $k$. 
}

\begin{quotation}
{\em
\noindent \textbf{Reviewer 1: } 3. In the provider's decision model, the heterogeneity on the provider participation decision is ignored, which gives some results that are not convincing. For example, on page 21, "an increase in storage prices doe not provide any indirect benefit, it only damages renters". However, we may expect more providers will decide to participate if the storage price is higher, which also affects the robustness/chance of file access.
}
\end{quotation}\vspace{-4mm}

\textcolor{red}{[Model Setting][재웅's turn]} Provider's entry를 반영한 모형을 제시하였으며, probability of file access의 경우 Comment 2와 같이 platform이 결정하는 형태로 변경하였음. Provider의 entry를 고려함에 있어 operating cost의 heterogeneity를 반영하였고, 또한 잠재적 provider의 수가 충분하지 않을 수 있는 시나리오를 함께 고려하였음. 이러한 provider의 entry decision 및 잠재적 provider의 수를 모형에 반영함에 따라 다양한 시사점을 도출할 수 있었음.

[We have incorporated the providers' decision on participation into the platform in this revision. Correct? Then explain here]
\yj{(youngjae to Prof. Park 2021-04-15) Yes, in the new model, we consider the providers' participation decision. It depends on their (cognitive) cost.}

\textcolor{blue}{This is a very great point. We revise our model to incorporate providers' decisions on participation into the platform. Specifically, (...)}

\begin{quotation}
{\em
\noindent \textbf{Reviewer 1: } 4. If the commission ration alpha is an endogenous decision by the platform, what's the difference comparing to an exogenous commission rate. 
}
\end{quotation}\vspace{-4mm}

\textcolor{blue}{[Model Setting][영재's turn]} Closed form solution을 도출하기에 너무 복잡한 세팅이라서(확인 필요) numerical analysis를 진행해봄. 분석 결과 given $\alpha$에 대해 구한 Lemma, Theorem과 거의 동일한 패턴을 보이는 것을 확인하였음(확인 필요).

[This is an important comment. We need to discuss. If we endogenies alpha in this model, then it can be related to the platform's strategy on growth or profit. If we fix this, then it should be justified well with a compelling example in practice.]

\textcolor{blue}{(Jaeung to YoungJae 2021-04-15) How can we address alpha? If it is too challenging to get the closed form of $(p_B, p_S, \alpha)$ decision, then should we conduct a numerical analysis on this?}

\yj{(youngjae to Jaeung 2021-04-15) We can address $\alpha$. It is much easier than address $f$ and $\theta$. Since, we have already derived optimal pricing schemes and optimal profit under given $\alpha$, it would be sufficient to change $\alpha$ later.}

\newpage
%%%%%%%%%%%%%%%%%%%%%%%%%%%%%%%%%%%%%%%%%%%%%%%%%%%%%%%%%%%%%%%
\noindent \underline{\large \bf Authors' Response to Reviewer 2}
%%%%%%%%%%%%%%%%%%%%%%%%%%%%%%%%%%%%%%%%%%%%%%%%%%%%%%%%%%%%%%%


%%%%%%%%%%%%%%%%%%%%%%%%%%%%%%%%%%%%
%\begin{quotation}
%{\em
%\noindent \textbf{Reviewer 2: }
%The study examines pricing schemes of a peer-to-peer (P2P) storage sharing platform, which is a two-sided marketplace of renters with various storage and bandwidth needs and providers with available storage capacity. The pricing scheme considered is a two-part tariff: a storage fee that is dependent of the volume of the files and a bandwidth fee that is dependent of the access rate.
%Renters decide whether to adopt the P2P platform, while providers determine the uptime that influences the service level. Based on a model that captures the renters’ and providers’ utilities, the optimal pricing parameters are derived under different objectives of the platform: profit-seeking, consumer-welfare maximizing, and a mixed objective that combines both.
%}
%\end{quotation}\vspace{-4mm}
%%%%%%%%%%%%%%%%%%%%%%%%%%%%%%%%%%%%
\begin{quotation}
{\em
\noindent \textbf{Reviewer 2: }
I read this paper with lots of interests. The authors find an interesting topic that has not been fully investigated in the existing literature, which is about the pricing schemes of the P2P storage platforms in the cloud. This topic is new and relevant, so I think it fits the theme of the special issue well.
}
\end{quotation}\vspace{-4mm}

We are happy to hear that you find our topic interesting and new and worthwhile investigation. We are also grateful that you appreciate the topic and relevant to the special issue.


\begin{quotation}
{\em
\noindent \textbf{Reviewer 2: }
Having said that, I feel that there is significant room for improvement in the rigor of the model including the objective of the platform’s optimization problem, formulation of the renters’ utility function, and certain model assumptions. In addition, the writing of this paper should be improved to better explain and justify the formulations and model assumptions. Last but not the least, I also have some high-level concerns on the positioning of this paper regarding its theoretical novelty and contribution.
}
\end{quotation}\vspace{-4mm}

\textcolor{blue}{[Model][영재's turn] (검토바람)} Thank you for your comments, which helped us improve this paper significantly. In this revision, we seriously study your comments and addressed your concerns. We hope you find this revision satisfactory. 각각의 포인트에 대해 상세히 답변하기에 앞서, 우리 revision의 주요 내용을 간략히 요약하면 다음과 같다:

\begin{itemize}
    \item \textbf{Platform's Objective.} Profit-seeking platform이 보다 일반적인 플랫폼의 형태라는 것에 동의한다. 이에 우리는 platform이 consumer welfare 자체를 maximize하고자 하는 상황을 고려하는 대신, platform이 택할 수 있는 여러 pricing scheme들(two-part tariff, subscription, hybrid pricing)에 따라 플랫폼의 profit 및 social welfare가 어떻게 달라지는지에 집중하고자 하였다.
    \item \textcolor{red}{YJ} 위 내용을 조금 만 더 다듬으면 될 것 같아요. 우리는 profit-seeking platform을 가정하였고, 여러가지 pricing scheme들을 고려하며 이들의 전략이 platform/renter/provider 및 전체 social welfare에 어떠한 영향을 끼치는지 다루었다. 등등 내용은 충분하니 쓰면서 다듬으면 될듯합니다.
    \item \textbf{Model Assumptions.} 1) Penalty 관련 반영. 산업 조사 추가 -> 최근 Uptime을 충분히 유지하지 않는 provider는 사실상 퇴출시키는 것을 발견 -> 기술을 활용해 renter에게 서비스가 제공되기 전에 미리 service level 확인 후 살아남는 provider만 남김. 이에 따라 platform이 충분히 높은 bandwidth service의 availability를 제공할 수 있는 상황 하에서만 operation을 하도록 pricing을 결정하는 것으로 가정을 변경하였음. 2) Operating cost 관련(영재 작성 요청). 3) Provider의 in-out decision 반영(but uptime decision은 1)의 이유로 배제), 4) (또 있는지 확인)
    \item \textcolor{red}{YJ} Operation cost는 크게 두 가지로부터 발생. 1. platform의 required uptime을 만족시키기 위해 발생하는 코스트 $f(t)$, 그리고 자신의 컴퓨터에 renter들이 access하면서 발생하는 cost $v_{rb}^t$ - 내 컴퓨터에 access되는 총량. 우리는 이 두가지로부터 발생되는 provider의 cost를 곱의 형태인 $f(t)v_{rb}^t$로 가정하였고(cognitive cost라고 보는게 더 안전할듯합니다), 이 cost에 대해 얼마나 민감한지를 heterogeneity하게 두었다. 예를 들어 컴퓨터를 많이 사용하는 사람들은 $v_{rb}^t$로부터 발생하는 cost가 높아질 수 있다. 
    \item 이때 쓸 때 조심해야할 것 같은 부분이 컴퓨터를 많이 사용하는 사람들은 그럼 $f(t)$가 낮아지는 것 아닌가? - 어차피 많이 사용하기에 $t$라는 시간을 유지하는게 더 쉬운게 아닌가? 라는 형태의 question이 나오면서 $f(t)$랑 $c_i$가 반대 방향으로 움직이는 것 처럼 해석이 될 수도 있을 듯 한데, required uptime $t$가 애초에 충분히 크다고 이야기를 하면서 $f(t)$는 불시에 발생할 수 있는 window update나 internet connection 불량 등으로부터 컴퓨터가 꺼질 수 있기에, 이를 방지하며 발생하는 cost라는 형태로 조금 더 구체화시켜주는것이 필요할듯합니다.
    \item \textbf{Positioning.} Optimal pricing 대신 pricing scheme의 선택에 따른 effect를 설명하는 것으로 변경. Cloud platform literature에는 P2P 기반에서 provider의 in-out decision, potential size of providers를 고려하였을 때, 기존의 cloud pricing들이 적용되었을 때 어떤 일이 벌어지는지 알려줌. Sharing economy literature에는 pricing structure를 고려하였을 때 유사한 시사점을 제공할 수 있는 서비스(e.g. Uber)가 있음. 그러나 추가적으로 모형의 특수한 점이 고려되었으므로, 시사점을 일반화하는 데에 있어 주의가 필요.
\end{itemize}

[We may explain how we change our model including objective, utility function, and assumption here. We can also explain how we revise the writing of this paper. All of these can be explained with bullet points.]

%%%%%%%%%%%%%%%%%%%%%%%%%%%%%%%%%%%%%%%%%%%%%%%%%%%%%%%%%%%%%%%%%%%%%%%%%
\noindent\textbf{Contribution of this paper}\\[-11mm]
%%%%%%%%%%%%%%%%%%%%%%%%%%%%%%%%%%%%%%%%%%%%%%%%%%%%%%%%%%%%%%%%%%%%%%%%%
\begin{quotation}
{\em
\noindent \textbf{Reviewer 2: }
Although the paper claims that the ``optimal pricing schemes" are developed for the platform, this statement is not precise. Indeed, this paper focuses on only the specific form of pricing schemes (i.e., a two-part tariff) as described in its introduction, and then the optimal parameters of the given form of pricing schemes are derived.

a) At least the authors should emphasize that their focus is on a specific form of pricing schemes, instead of developing the optimal pricing scheme among all possible forms. Note that the two-part-tariff has been well studied in the economics and management science literature, but I can understand that the authors can justify that its application to the P2P storage businesses has practical values due to the novel trade-offs.}
\end{quotation}\vspace{-4mm}

\textcolor{red}{[Model][재웅's turn]} 너무 좋은 코멘트이고, 우리도 이에 맞춰 다른 possible form을 탐구하였음. 아래에 추가적인 코멘트에 해당하는 scheme들을 directly and indirectly 반영하였고, 또한 다루지 않은 pricing에 대해서는 discussion에서 아래와 같이 명시적으로 논의하였음.

"It is also worth investigating other pricing schemes, such as (...)"

\begin{quotation}
{\em
\noindent \textbf{Reviewer 2: }b) Although I am fine if the authors want to mainly focus on the specific form of pricing schemes (as it is popular among P2P storage platforms nowadays), it is a pity that this study is completely silent on other possible forms of pricing schemes. Note that the P2P storage business is new and evolving rapidly, some other pricing schemes may rise to prominence as they become more suitable and effective for this marketplace. 
}
\end{quotation}\vspace{-4mm}

\textcolor{red}{[Model][재웅's turn]} 아주 좋은 포인트임. P2P storage 시장이 새롭게 생겨나고 있고, 아직 establish 되지 않았기에 여러 pricing scheme들이 발생할 수 있음. 이에 우리는 기존의 Cloud platform에서 이용되던 주요 pricing들을 비교하고, 이들의 profit 및 social welfare를 비교함을 통해 플랫폼 및 비즈니스 전반의 가치에 대한 시사점을 제공하고자 함.

\yj{2021.04.22
\begin{itemize}
    \item three-part tariff
    \item subscription (storage and bandwidth limitation?)
\end{itemize}
}

\begin{quotation}
{\em
\noindent \textbf{Reviewer 2: } To provide a few ideas: (i) Since service level plays a key role in this marketplace, it is natural to develop pricing schemes based on service levels. Specifically, between the platform and the providers, the platform may include a penalty-award term to motivate the providers to maintain a reasonable service level. In the same spirit, between the platform and the renters, the platform may make the bandwidth fee dependent on a commitment of service level. Also, it is worth noting that the service-level based, penalty-award scheme has attracted attentions from studies of the centralized public cloud (Yuan et al. 2018).
}
\end{quotation}\vspace{-4mm}

\textcolor{red}{[Model][재웅's turn]} Penalty 관련해서 조사한 결과, 플랫폼들이 다음과 같은 형태로 반영하고 있음을 알게 되었음(실제 예시 상세하게 서술)

이에 우리는 provider가 renter에게 assign되기 전에 service level을 기준으로 self-selected 또는 platform에 의해 screen out 되는 상황을 상정하였고, 이러한 방법을 통해 platform은 주어진 redundancy 하에서 목표 availability를 달성하는 최소한의 uptime을 요구하고, 이를 만족하는 provider만 플랫폼에 남는다고 가정하였다.

[Cite Yuan et al. (2018) and discuss the service-level based pricing scheme. We can choose to discuss either 1) it is not suitable for our P2P file-sharing platform or 2) it is possible and has a potential.]

\begin{quotation}
{\em
\noindent \textbf{Reviewer 2: }
(ii) Another possibility is to refine the access fee for the bandwidth service. For example, a subscription-based pricing scheme (as in the cell phone data plans) is relevant. The storage fee in this paper essentially plays the role of the subscription fee. Once the storage fee is paid (i.e., once subscribed), the renter is granted a limit of free access to the stored files, and any access beyond the limit will be charged an additional access fee. Also, it is worth noting that variants of the subscription-based pricing schemes have been studied (Lambrecht et al. 2007, Li and Kumar 2018) and implemented in similar business environments (mobile communication, centralized public cloud, etc.)}
\end{quotation}\vspace{-4mm}

\textcolor{red}{[Model][재웅's turn]} Thank you for this constructive comment. We adopt the suggested model in addition to two-part tariff (previous) and subscription model. We name the combination of these pricing schemes "hybrid" model following the literature (cite here).

By comparing these pricing schemes, we find that (...)

[It has indeed a potential. Can we interpret our bandwidth price such that the more frequently the renter needs an access to a file, the higher the renter pays the bandwidth price. Actually, this reviewer suggests analyzing a bandwidth price with several steps (discrete). For example, level 0 (basic), level 1 (moderate), level 2 (advanced), etc. We can refer to Netflix's pricing scheme for this. We have to respond to this comment seriously.]

\begin{quotation}
{\em
\noindent \textbf{Reviewer 2:  }I was not suggesting that the authors have to consider all possibilities. However, I am hoping that the authors would acknowledge the limitations of their focus on the specific form of pricing schemes in this paper. Moreover, it would be much better if they can demonstrate the flexibility of their model framework by showing some extensions to other pricing schemes (e.g., service-level based or subscription-based).}
\end{quotation}\vspace{-4mm}

\textcolor{red}{[Model][재웅's turn]} Describe how we discuss this limitation.

[We need to start responding to this ASAP as it can take some time. Need to discuss it with Jaeung and Prof. Cho.]

%%%%%%%%%%%%%%%%%%%%%%%%%%%%%%%%%%%%%%%%%%%%%%%%%%%%%%%%%%%%%%%%%%%%%%%%%
\noindent\textbf{Objective of the platform}\\[-11mm]
%%%%%%%%%%%%%%%%%%%%%%%%%%%%%%%%%%%%%%%%%%%%%%%%%%%%%%%%%%%%%%%%%%%%%%%%%
\begin{quotation}
{\em
\noindent \textbf{Reviewer 2:  }One of the main highlights of this study is to consider the platform’s different objectives (profit vs consumer-welfare). With that said, for a two-sided marketplace like this, it is equally important to consider the welfare of the providers. Maximization of only the renters’ welfare may lead to an imbalanced marketplace where demand largely exceeds supply. For example, Uber in the ride-sharing industry must consider welfare of not only the passengers
but also the drivers. In the same vein, I suggest considering an alternative objective of the platform that is to maximize the welfare of the entire marketplace (providers and renters).}
\end{quotation}\vspace{-4mm}

\textcolor{red}{[Model][재웅's turn]} Platform의 직접적인 objective는 profit만 고려하는 것으로 하고, pricing scheme 마다의 profit maximization point에서 social welfare가 어떻게 달라지는지, 이러한 profit vs. social welfare max 전략이 potential number of peers willing to share their computing resources가 달라짐에 따라 어떻게 바뀌는지 시사점을 주고자 하였다.

[Explain our new objective from the perspective of the platform. Actually, our growth strategy considers both renters and providers. Correct?]

%%%%%%%%%%%%%%%%%%%%%%%%%%%%%%%%%%%%%%%%%%%%%%%%%%%%%%%%%%%%%%%%%%%%%%%%%
\noindent\textbf{Formulation of the renters' utility function}\\[-11mm]
%%%%%%%%%%%%%%%%%%%%%%%%%%%%%%%%%%%%%%%%%%%%%%%%%%%%%%%%%%%%%%%%%%%%%%%%%
\begin{quotation}
{\em
\noindent \textbf{Reviewer 2:  }Formulation of the renters’ utility function. In the current model, the renters’ utility function is formulated in a way that implies the following assumption holds: if the renter attempts to access to the file stored but the access fails, then the renter would give up. This can be seen from the formulation in the paper, 
\begin{equation*}
U_i^{rt}(x; \mathbf{t}, p_S, p_B) = (u_S - p_S) V_i + (u_B - p_B)(1-g(\mathbf{t}))V_i \lambda_i
\end{equation*}
However, one may argue that in reality, the renter may attempt to access the file again until the access is granted. As a result, the renter will need to pay the access fee anyway, while the renter incurs an extra inconvenience cost that can be viewed as a loss of utility due to the delay, which
depends on the failure probability. In this spirit, the renter’s utility function should be as follows:
\begin{equation*}
U_i^{rt}(x; \mathbf{t}, p_S, p_B) = (u_S- p_S) V_i + [u_B(1-g(\mathbf{t}))-p_B] V_i \lambda_i
\end{equation*}

In my opinion, the latter (as proposed) is more reasonable than the former (as in the paper). Indeed, the result of Lemma 6 sounds the alarm. This is because when the prices are set to extract the maximum surplus utility from the renters (i.e., $p_S = u_S$ and $p_B = u_B$), one may argue that, intuitively, no renter will find it attractive to use the P2P storage services, as their utilities will be negative due to the inconvenience cost. As can be seen, when $p_S = u_S$ and $p_B = u_B$, the utility function in this paper is equal to zero and fails to capture the additional inconvenience cost due to the failure of access}
\end{quotation}\vspace{-4mm}

\textcolor{red}{[Model][재웅's turn]} Failure 관련해서 penalty scheme을 현재 플랫폼이 활용하는 방식으로 반영함에 따라, utility function에서는 failure term을 고려하지 않았음. 그러나 이는 failure가 발생하더라도 결국은 download가 성사되어야 한다는 reviewer의 코멘트와 동일선상에 있는 것으로, 우리의 새 utility fundtion은 reviewer가 suggest한 버전을 platform의 정책을 반영하여 단순화 시킨 것이라고 할 수 있다.

[Explain our new utility model of renter here and see how it differs from the proposed one above. We need to say that your concerns is mitigated with our form.]
\yj{2021.07.13}
만약, 기존의 모델 form 을 그대로 사용한다면, 다운로드 실패했었을 때의 cost는 다음과 같은 특징을 가짐. 1. online platform이라는 것의 특성 상, 다운로드가 실패되었을 때는 비용이 부과되지 않음. 2. 하지만, 실패했었을 때, 고객들이 failure로 부터 느끼는 cost가 있을 수 있음. 이를 반영하면 다음 cost형태가 더 적절\\
\begin{equation*}
    U_i^{rt}(x; \mathbf{t}, p_S, p_B) = (u_S - p_S) V_i + [(u_B-p_B)(1-g(\mathbf{t}) - c_f g(\mathbf{t})] V_i \lambda_i
\end{equation*}

현재의 utility 모델에서 이를 마찬가지로 적용한다면, 
\begin{equation*}
    U_i = \lambda_i (u_i - p_b)(1-g(\mathbf{t})) - \theta p_s - c_f g(\mathbf{t})
\end{equation*}
형태가 됨. 하지만, uptime으로 인한 failure probabilty가 충분히 작으므로 우리는 $g(\mathbf{t}) \approx 0$으로 가정.\\

\yj{2021.04.22
\begin{itemize}
    \item 위의 가정들은 다운로드가 실패했었을 때의 cost를 다루는 요소들인데, 이제는 다운로드 실패 자체를 고려하지 않아서 위의 변화를 다룰 필요는 없음
    \item 다만, 만약 고려를 한다면 개인적으로는 $(u_B-p_B)(1-g(\mathbf{t}) - c_f g(\mathbf{t})$ 형태가 더 맞다고 봄
    \item 이렇게 가정을 하는데, '저장 플랫폼' 특성상, $c_f$가 너무 크기에 다운로드를 실패 할 확률인 $g(\mathbf{t}) \sim 0$이 되도록 uptime $t$를 require 한다고 해석 가능. (물론, 지금 utility 모델은 이전과 많이 달라져서 조금 수정해서 적용하여야 할 필요성은 있음
    \item 첫번째로 제출하기 전에 두번째 모델을 고려해서 문제를 푸는 것을 시도하였으나 제대로 마무리가 되지 않아서 기존 버전으로 제출
\end{itemize}
}
%%%%%%%%%%%%%%%%%%%%%%%%%%%%%%%%%%%%%%%%%%%%%%%%%%%%%%%%%%%%%%%%%%%%%%%%%
\noindent\textbf{Model assumptions}\\[-11mm]
%%%%%%%%%%%%%%%%%%%%%%%%%%%%%%%%%%%%%%%%%%%%%%%%%%%%%%%%%%%%%%%%%%%%%%%%%
\begin{quotation}
{\em
\noindent \textbf{Reviewer 2:  }I can understand that the current model is complicated due to the interplays among the renters’ utilities, the providers’ utilities, and the platform’s pricing decisions. Thus, I am fine with most of the model assumptions. However, I would like to point out two main concerns}
\end{quotation}\vspace{-4mm}

Thank you for your understanding on our model assumption. In this revision, we carefully revisited our assumptions and justify them better. We hope you see they are revised reasonably.

\begin{quotation}
{\em
\noindent \textbf{Reviewer 2:  }a) To be honest, I feel the description of the redundancy algorithm (page 9) is somewhat vague. Specifically, how are the $m$ shards distributed across selected providers in the network, how are the providers selected, and how does the distribution of the shards affect the probability of failure to reconstruct the original file? Perhaps, the authors can kindly provide in the appendix a more detailed explanation of the underlying algorithm, with a focus on the impact on the failure probability. For example, suppose that there are $m$ = 2 shards to be distributed to a network with 10 providers with redundancy k = 4, then how to distribute the shards, and what is the resulting probability? The vague description of the redundancy algorithm hinders the reader’s ability to appreciate the simplified assumption $m$ = 1. That is, without a clear understanding of how the redundancy algorithm affects the failure probability, it is still hard to convince the reader that the simplification will not cause any loss of main insights into the problem. Moreover, according to the introduction, in reality no single provider has enough fragmented pieces to reconstruct the original file; thus, assuming m = 1 actually contradicts the introduction.}
\end{quotation}\vspace{-4mm}

\textcolor{red}{[Model][재웅's turn]} (아래 영재의 설명을 영어로 옮길 예정)

[Explain our new model on this redundancy and uptime decision. I can then revise it to correctly respond to this comment.]
\yj{2021.04.22
\begin{itemize}
    \item 일단 첫번째로 현재 파일을 재조립하는 방식은 erasure coding 방식을 주로 적용하고 있음(우리가 첫 논문을 작성하고 제출할 당시에는 erasure coding 방식 적용을 시도하고 있고, 단순하게 redundancy를 높이는 방향으로 시작하였으나 현재는 erasure coding 방식을 모두 적용하고 있는 것으로 보임)
    \item 이 방식은 파일을 $k$개로 쪼갠 후, 이를 토대로 $n-k$개의 복원 파일을 형성. 그리고 총 $n$개의 파일을 $n$명에게 배포하는데, 이때 $n$개 중 임의의 $k$개만 있어도 기존의 파일이 복원 가능한 방식.
    \item 기존에는 uptime에 따른 실제 다운로드 확률은 $g(\mathbf{t})$라는 함수를 직접 계산해서 식에 대입하였기 때문에, $2 \le k$만 되더라도 복잡도가 높아지고, $k=1$일때도 $n=2$인 경우만 적절한 복잡도의 함수가 나와서 이 경우를 채택하여 문제를 풀었으나, 이 형태의 경우 기존의 알고리즘을 표현하기에는 부족했던 것이 사실임
    \item 현재는 똑같은 $k-n$ erasure coding을 고려하지만, 기업이 현실적으로 다운로드 실패가 발생하지 않는다고 봐도 무관할 수준으로 실패할 확률 $\bar{p}$를 달성하게 $k-n$에 따른 required uptime $t$를 설정. (이때 $\bar{p}$는 binomial distribution으로 계산 - 논문 appendix에 첨부)
    \item 파일을 '몇개로 나누는 것 - $k$'는 기술의 영역이라고 볼 수 있기 때문에 '얼마나 파일을 많이 분배하는지 - $n$'와 '각 provider에 얼마만큼 긴 uptime 을 요구하는지 - $t$' 사이의 trade-off를 다룰 예정
    \item redundancy rates를 $\theta = \frac{n}{k}$라 했을 때, $\theta$를 높이고 $t$를 낮추는 것은 각 provider에게 부담을 낮추는 대신에 더 많은 용량을 확보하는 것이 필요한 전략이고, $\theta$를 낮추며 $t$를 높이는 것은 용량 확보에 대한 어려움은 적어지지만, 각 provider에게 부담을 많이 주는 trade-off가 있음.
    \item  sia와 storj 모두 각 provider에게 score를 매김. score는  평판(이런 decentralized storage platform을 넘어서 보통 community 단위로 운영되는 blockchain system의 경우 평판제도가 있는 경우가 많음)과 각 provider들이 제공하는 uptime, 그리고 그 외에도 다양하게 provider이 플랫폼에 기여할 수 있는 역할들이 있어서 이러한 요소들로 정해짐
    \item 따라서 이 score들은 각 provider의 신뢰도를 표현하는 지표로 사용이 되며, 플랫폼에서는 renter가 파일을 저장할 때 provider의 score가 높을수록 더 많은 파일 조각을 저장시킴
    \item 하지만, 이런 요소까지는 구체적으로 담을 수 없기에 본 논문에서는 임의로 배정된다고 가정.
\end{itemize}

}

\begin{quotation}
{\em
\noindent \textbf{Reviewer 2:  }b) The assumption that all renters have the same willingness-to-pay for the storage and bandwidth service (i.e., $u_S$ and $u_B$ are homogeneous) is restrictive. Intuitively, some customers may have relatively high willingness-to-pay for the storage service and relatively low willingness-to-pay for the bandwidth service, while some other customers may have the opposite valuations of the storage and bandwidth services. I wonder whether the current model can be easily extended to incorporate the heterogeneity of the willingness-to-pay?}
\end{quotation}\vspace{-4mm}

Thank you for your comment on our homogeneous utility assumption. In this revision, we incorporated this and make the utilities heterogenous across renters. 

[This is incorporated well in this revision. Explain our new model and new insights obtained from this if possible.]
\yj{2021.04.22
\begin{itemize}
    \item 현재 논문에서 이 플랫폼을 이용하며 얻는 utility $u_i$와 사용하는 빈도 $\lambda_i$에 heterogenity를 부여
    \item $u_i$를 어떻게 해석하는지에 대한 고민을 지속하여 하여야 하는데, 이 부분에서 '이 플랫폼의 특성 - 보안성 등'을 반영한 해석을 어느 정도 넣는 것도 의미성을 부여할 수 있지 않을까 생각
\end{itemize}
}

%%%%%%%%%%%%%%%%%%%%%%%%%%%%%%%%%%%%%%%%%%%%%%%%%%%%%%%%%%%%%%%%%%%%%%%%%
\noindent\textbf{Other issues}\\[-11mm]
%%%%%%%%%%%%%%%%%%%%%%%%%%%%%%%%%%%%%%%%%%%%%%%%%%%%%%%%%%%%%%%%%%%%%%%%%
\begin{quotation}
{\em
\noindent \textbf{Reviewer 2:  } a) The literature review on the sharing economy can be expanded further. Also, there are quite a few studies in this area which examined the pricing strategies of the platform (on the contrary to ``scant attention"). See Hu (2019) as an excellent source.}
\end{quotation}\vspace{-4mm}

\textcolor{red}{[Model][재웅's turn] (수정 시 SE의 4번 코멘트와 align 필요) (뒤에 b)하고 어떻게 나누어 쓸지도 고민 필요)} (2) Sharing economy에 대한 기여: 많은 논문들이 pay-per-use 혹은 subscription에 대해 다루었고, two-part tariff에 대한 sharing economy 연구는 많지 않았음. 우리 논문에서 다룬 two-part tariff vs. subscription vs. hybrid pricing에 대한 시사점은 이동 거리에 비례해 추가 요금을 지불하게 되는 ride sharing 플랫폼과 같은 상황에서도 적용될 수 있다.

다만 우리의 결과를 적용함에 있어 신중한 접근이 필요한데, 이는 P2P storage market의 다음과 같은 특징이 모형에 반영되어 있기 때문이다: 최초 storage service의 계약은 on-demand로 이루어지지만, 이후의 계약은 not temporary but continuous, storage contract가 맺어진 상황에서 bandwidth service가 제공되지 않았을 때의 loss가 기존의 sharing economy보다 critical (provider 편할 때에만 제공 X), redundancy의 도입을 통해 service level을 유지함으로써, 알고리즘 부분의 이유로 demand보다 더 많은 supply가 필요하며, 여러 provider가 동시에 서비스를 제공해는 특징이 있다.

위의 서비스 특징으로 인해 idle resource가 간헐적으로 발생하는 provider들이 시장에 진입할 수 없도록 시장이 설계되어 있으며, 한 단위의 demand를 충족 시키는 과정에서 얼마나 여러 provider가 필요하냐에 따라 개별 provider가 감당하는 operating cost가 달라진다(numerical analysis). 

\yj{2021.04.22 참고할 예정, 각 chapter 모두 공유폴더에 다운로드 완료 }

\begin{quotation}
{\em
\noindent \textbf{Reviewer 2:  } b) Also, in the literature review, please highlight the distinct modeling features of the P2P storage services, as compared to the sharing economy models and the P2P files sharing services.}
\end{quotation}\vspace{-4mm}

[Important: This should be done well.]
\yj{위의 SE에서도 비슷한 피드백이 있어서 같이 진행 할 필요성이 있음}

\begin{quotation}
{\em
\noindent \textbf{Reviewer 2:  } c) In the second paragraph of Section 3: ``As long as a provider has redundant storage space, sharing the storage always benefits him due to the initial storage fee $p_S$ even when he does not contribute bandwidth to the network at all." I am wondering in reality, whether the provider will receive any form of penalty if his contribution to the network is maintained at a very low level; the penalty can be explicit as written in a contract with the platform or implicit as the threat of losing future business from the platform in the future. Note that this question is also related to my previous point 1-b-(i).}
\end{quotation}\vspace{-4mm}

\textcolor{red}{[Model][재웅's turn]} 좋은 포인트임. 앞서 지적해주었던 penalty scheme에 대한 아이디어에서 출발해 플랫폼의 현재 행태를 조사하였고, provider를 사실상 반강제적으로 이탈시키는 scheme을 활용하고 있음을 확인함. 

[We need to investigate what happens in practice and then respond]
\yj{(youngjae, 2021-04-15) When provider signs up, provider must set some deposits and some of deposits are deducted as penalty cost if provider maintain at a very low level. However, if provider continues to maintain the low level, he lose his qualifications as a provider. }
\yj{2021.04.22 현재 제도를 조금 더 찾아본 결과, 아직 강제적으로 provider 자격을 박탈하는 정책까지는 펼치지 못하고, 앞서 언급한 각 provider의 점수를 떨어뜨리는 방식으로 반강제적 이탈을 시킴. }

\begin{quotation}
{\em
\noindent \textbf{Reviewer 2:  } d) For Lemma 1, it would be better to define the notations and the meaning of the cases before the statement of Lemma 1.
}
\end{quotation}\vspace{-4mm}

\textcolor{red}{[Model][재웅's turn]} Section 2에서 관련 notation, assumption 들을 미리 제공함으로써 이후에 등장할 Lemma 및 Theorem에서 혼란이 없도록 수정하였음.

[We can make a table for explaining notations]

\begin{quotation}
{\em
\noindent \textbf{Reviewer 2:  } e) For the provider’s problem (on top of page 14), please explain why the operating cost is $c \tilde{\lambda} t_j^2$ instead of $ct_j^2$? I am confused as I thought the operating cost is dependent on the uptime only; in other words, the operating cost would be the same given the same uptime, no matter how many accesses to the files are granted during the uptime.
}
\end{quotation}\vspace{-4mm}

$t$에 대해서 

[Explain this as physical and mental penalty as the penalty of using computing resources for granting access to renters from the provider's computer.]

\yj{2021.04.22
\begin{itemize}
    \item 기존의 cost term은 $c \tilde{\lambda} t_j^2$으로, 현재와 비슷한 논리 형태로써 나의 컴퓨터에 접속되는 빈도(사실 더 정확하게는 다운로드 되는 용량이 맞았던 것으로 보여짐)와 time의 quadratic term으로 정의했었음. 참고로 이때의 $t_j$는 provider 의 decision
    \item 현재는 provider가 decision하는 요소는 오직 in-out decision으로 바뀌고(기존에는 provider가 homogeneous하다고 가정하였기에 모두다 in or out하는 형태) cost term은 $c_j \hat{v}_{rb}^t f(t)$ 형태로 표현
    \item $c_j$는 개인이 느끼는 cost의 정도(heterogeneous, $c_j \in [0, 1]$
    \item $f(t)$는 platform이 required uptime을 $t$로 지정하였을 때 provider가 느끼는 cost - $t$만큼 유지하기 위한 순수한 전기비를 비롯하여 $t$라는 uptime을 유지하기 위한 노력들이 포함(update 등으로 인하여 자동으로 컴퓨터가 꺼지는 경우들을 비롯하여 다양한 요소들이 발생할 수 있을 것이라 생각)
    \item 이때의 $f(t)$는 어떤 form으로 잡는게 좋을 지 모르겠어서 일단 general form으로 잡았고, 추후에 $f(t)$와 관련하여 분석할 때는 convex 형태로 잡는 것이 가장 reasonable하지 않을까 생각 - 특히 1에 가까워질수록 훨씬 더 가파르게 증가하는 형태가 적절하지 않을까 싶음
    \item $\hat{v}_{rb}^t$는 provider의 컴퓨터에서 다운로드 될 때 발생하는 cost. computing power가 이 플랫폼에 사용됨으로 발생하는 코스트.
    \item 두 term의 곱으로 표현되어 있는 부분들에 대해서는 충분한 설명이 필요할 듯 보이고, 둘 다 physical 혹은 cognitive한 cost 두 형태로 다 설명이 가능하기는 함.
\end{itemize}
}

\begin{quotation}
{\em
\noindent \textbf{Reviewer 2:  }f) For Lemma 6, in addition to my concern on the homogeneous willingness-to-pay, I also wonder how the platform can accurately measure the renters’ willingness-to-pay in reality in order to implement the proposed pricing scheme.}
\end{quotation}\vspace{-4mm}

\textcolor{blue}{[Model][재웅's turn]} That is a valid concern. 현재 시장이 충분히 establish 되지 않은 상황에서 정확히 어느 지점이 optimal price인지 시사점을 주는 것은 쉽지 않다고 판단된다. 따라서 핵심적인 연구 질문을 주어진 scheme 안에서 최적의 price point를 찾는 것 대신, 기존의 cloud platform에서 활용되던 scheme들 간의 비교로 변경하였다.

[Do we need to know them exactly? We can respond to this by claiming that we only need a distribution of this.]
\yj{2021.04.22 현재의 $u_i$라는 heterogeneity term이 willingness-to-pay(와는 다르긴 하지만) review와 연결되는 형태로 보임. 이때 $u_i$에 대한 해석은 위에서 언급한 것 처럼 주의해서 해야 할 필요는 있음. 하지만, 현재 utility모델 자체가 조금 더 renter 단위보다는 각 파일 단위로 해석하기 좋아서, 보안이 필요한 정도 등에 따른 utility 차이라는 형태로도 충분히 해석은 가능해 보임 }

\begin{quotation}
{\em
\noindent \textbf{Reviewer 2:  }g) Perhaps it is just my personal taste, but there are too many lemmas in this paper. Please consider consolidating some of the lemmas to highlight the most important results.}
\end{quotation}\vspace{-4mm}

\textcolor{red}{[Model][재웅's turn]} That is a valid point. 이번 revision에서 우리는 연구의 방향성 및 contribution을 review하였고, 이 과정에서 pricing scheme 선택의 영향력(Theorem 2)을 우리의 핵심 메시지로 가져가기로 하였다. 그리고 앞서 등장하는 Lemma들이 Theorem 2를 이해하는 데에 중요한 내용들만 담는 것으로 하여, 이전의 8개(double check 필요)에서 5개(영재 확인 필요)로 줄였다.

[This is also my personal taste. We can try...]

\begin{quotation}
{\em
\noindent \textbf{Reviewer 2:  }In summary, my view on this paper is mixed. As noted in the beginning, I like the topic of this study and appreciate that the authors have identified an interesting, novel, and relevant problem. But on the other hand, there are main concerns on the rigor of the model (formulation of objective and utility functions, model assumptions, etc.) and the novelty of the pricing schemes considered in this paper.

Having realized that this paper tackles a new research topic and the current model is complicated, I would like to encourage the authors to improve their paper, and thus I recommend a major revision of this paper to the editors, although it may be a risky one in my opinion. I am hoping that the authors could enhance the rigor of their core model based on the current pricing scheme and also extend the scope to explore (discuss) other forms of pricing schemes. I also hope that the authors would find my feedback helpful and wish them good luck in revising the paper.}
\end{quotation}\vspace{-4mm}

\textcolor{blue}{[Model][TBD]} 위의 부분들 confirm한 이후에 적당히 요약 + 감사 인사 

%%%%%%%%%%%%%%%%%%%%%%%%%%%%%%%%%%%%%%%%
%\bibliographystyle{ormsv080}
%\bibliography{_ref_SMSA}
%%%%%%%%%%%%%%%%%%%%%%%%%%%%%%%%%%%%%%%%

\end{document} 
