\documentclass[11pt]{article}
\usepackage[T1]{fontenc}
\usepackage{times}%, babel}
\usepackage{secdot,natbib}
\usepackage{latexsym, amsthm, amsmath, amssymb, color, rotating, multirow, graphicx, hhline, array, tablefootnote}
\usepackage{scalefnt, enumerate}
\usepackage{footnote}
\usepackage{threeparttable,booktabs}
\usepackage{natbib}
%\usepackage{bm, url}
%\usepackage{slashbox}
\usepackage{tikz,pgfplots}
\usepackage {subcaption}
\usepackage[pdftex]{graphicx}
\usepackage[margin=10pt,font=small,labelfont=bf,labelsep=endash]{caption}
\usepackage{kotex} %Korean TeX
%\usepackage[latin9]{inputenc} % kotex package와 충돌. 향후 kotex 삭제하면 될 듯.




%%% ----------------------------------------------------------------------
%\clubpenalty=10000 \widowpenalty=10000
%\renewcommand{\baselinestretch}{1.5}
% \renewcommand{\theequation}{{\rm \thesection.\arabic{equation}}}
\renewcommand{\theequation}{{\rm \arabic{equation}}}
\oddsidemargin 0in  %.10in
\evensidemargin 0in
\hyphenpenalty 2000

\textwidth 6.5in    %6.5
\textheight 8.75in     %8.5
\topmargin 0.0in      %0in
\headsep 0in \makeatletter
\newcommand{\singlespacing}{\let\CS=\@currsize\renewcommand{\baselinestretch}{1.1}\tiny\CS}
\newcommand{\doublespacing}{\let\CS=\@currsize\renewcommand{\baselinestretch}{1.5}\tiny\CS}
\newcommand{\realdoublespacing}{\let\CS=\@currsize\renewcommand{\baselinestretch}{2.0}\tiny\CS}
\newcommand{\mydoublespacing}{\let\CS=\@currsize\renewcommand{\baselinestretch}{1.499}\tiny\CS}

\newtheorem{theorem}{Theorem}[section]
\newtheorem{proposition}[theorem]{Proposition}
\newtheorem{lemma}[theorem]{Lemma}
\newtheorem{corollary}[theorem]{Corollary}

\def\N{\mathbb{N}}
\def\1{\mathbf{1}}
\def\P{\mathbf{P}}
\def\E{\mathbf{E}}
\newcommand{\maximize}{\mathop{\mbox{{\rm maximize}}}\limits}
\newcommand{\minimize}{\mathop{\mbox{{\rm minimize}}}\limits}
\newcommand{\argmax}{\mathop{\mbox{{\rm arg\,max}}}\limits}


%\newcommand\ks[1]{{\textbf{#1}}}
\newcommand{\ks}[1]{{\color{blue} #1}}
\newcommand{\ju}[1]{{\color{magenta} #1}}
\newcommand{\yj}[1]{{\color{red} #1}}


\setlength\parindent{0cm}
\setlength{\parskip}{12pt}%

\renewcommand\ttdefault{cmvtt}

%%%%%%%%%%%%%%%%%%%%%%%%%%%%%%%%%%%%%%%%%%%%%%%%%%%%%%%%%%%%
\begin{document}

\begin{center}
{{\Large \bf Response to Review Reports on POM-Jul-20-SI-0934.R1}\\[6mm]
{\LARGE ``Sharing Economy in the Cloud:\\ Pricing Schemes for Peer-to-Peer Storage Platforms''}\\[15mm]}
\end{center}

\baselineskip 18pt

\noindent \underline{\large \bf Authors' Response to SENIOR EDITOR}\\[-11mm]


%%%%%%%%%%%%%%%%%%%%%%%%%%%%%%%%%%%
\begin{quotation}
{\em
\noindent \textbf{Senior Editor: } I sent the revised paper to the same review team. While they acknowledge and appreciate the effort exerted by the authors, both reviewers continue to have major concerns with the paper. As a result, both reviewers suggest another round of “major revision.” I went over the revised paper and the reviewers’ comments, and I concur with the reviewers’ comments. I would like to recommend another “major revision” and give the authors the opportunity to address the concerns. 
 
Given the reviewers’ expertise and the thorough reports they provide, I have little to add. Rather, I would refer the authors to the detailed comments in the reviewers’ reports. Next, I would like to highlight three major concerns. 
}
\end{quotation} \vspace{-4mm}


%%%%%%%%%%%%%%%%%%%%%%%%%%%%%%%%%%%



%%%%%%%%%%%%%%%%%%%%%%%%%%%%%%%%%%%%%%%%%%%%%%%%%%%%%%%%%%%%%%%%%%%%%%%%%
\noindent\textbf{Main Concern: Uniqueness of Storage Sharing}\\[-11mm]
%%%%%%%%%%%%%%%%%%%%%%%%%%%%%%%%%%%%%%%%%%%%%%%%%%%%%%%%%%%%%%%%%%%%%%%%%

\begin{quotation}
{\em
\noindent \textbf{Senior Editor: }  
As R2 correctly notes, it remains unclear in this revised version what are the unique features of storage sharing (compared with other peer-to-peer platforms) and whether these unique features are captured by the main model. If this study is positioned to focus on storage sharing, the authors may want to elaborate on its distinct features, ensure the model captures some (if not all) of these features, and explain how these distinct features connect with the main results. Of course, the alternative is to position the paper to study peer-to-peer platforms in general. In that case, it will be fine to assume away some specifics of a particular type of platforms. In either way, the authors should be upfront and present clearly about that to readers, and articulate the contribution beyond the existing literature. 
}
\end{quotation} \vspace{-4mm}
\begin{itemize}
    \item redundancy chapter를 앞으로 빼는 것 +  reviewer 2에게 답하는 내용으로 충분히 답변 가능 
\end{itemize}


%%%%%%%%%%%%%%%%%%%%%%%%%%%%%%%%%%%%%%%%%%%%%%%%%%%%%%%%%%%%%%%%%%%%%%%%
\noindent\textbf{Main Concern: Model}\\[-11mm]
%%%%%%%%%%%%%%%%%%%%%%%%%%%%%%%%%%%%%%%%%%%%%%%%%%%%%%%%%%%%%%%%%%%%%%%%%

\begin{quotation}
{\em
\noindent \textbf{Senior Editor: } Both reviewers continue to have concerns with the model setup. 
(a) Heterogeneity of users: As R2 points out, users (both renters and providers) can be heterogeneous in other dimensions as well, which is a good observation. The authors are encouraged to explore possible models which consider heterogeneity in multiple dimensions. However, I would think that the review team will also be fine if the authors can only incorporate some heterogeneity into the model (e.g., the heterogeneity the current model considers). If they take this approach, the authors should elaborate on at least the following two aspects---why the authors consider this specific heterogeneity against other possible heterogeneity, and how the chosen model setup (i.e., ignoring the other possible heterogeneity) affects the main insights.  }
\end{quotation} \vspace{-4mm}
\begin{itemize}
    \item heterogeneity에 대한 측면에서는 storage volume에 대한 numerical study 추가한 것 언급
    \item 이 답변에 맞추어 R2 답변을 쓰는 식으로 하면 될듯
\end{itemize}

% \ju{R2가 제시한 storage volume의 heterogeneity는 매우 좋은 포인트임. 우선 기존 모델에서 해당 heterogeneity를 고려하여 다시 분석해보았음. 이때 two-part tariff와 subscription-based pricing에는 아무런 변화가 없고, hybrid pricing의 경우 계약 형태에 따라 제공되는 무료 용량이 달라질 수 있어 다음의 두 시나리오를 고려하였다: 1) 많은 저장 용량을 계약하면 많은 무료 전송 용량을 제공하는 경우, 2) 저장 용량과 무관하게 무료 전송 용량을 고정하는 경우. 첫 번째 시나리오의 경우 기존 모형과 아예 달라지는 바가 없었으며, 두 번째 시나리오의 경우 closed-form 형태를 얻을 수 없었으나, numerical analysis 결과를 살펴보아 pricing 간 상대적 우열의 차이는 없는 것으로 파악되었다. 따라서 우리는 main model에서는 기존의 heterogeneity를 중심으로 가져가되, 저장 용량의 heterogeneity가 결과에 미칠 수 있는 부분을 discussion 및 appendix에서 다루었다.}


%%%%%%%%%%%%%%%%%%%%%%%%%%%%%%%%%%%%%%%%%%%%%%%%%%%%%%%%%
\begin{quotation}
{\em
\noindent \textbf{Senior Editor: } (b) Information: R2 raises a question about how an individual provider could learn the information about the expected demand, which requires some clarification. If there is no specific mechanism or channel for providers to learn this information, I wonder if their “rational expectation” works. At any rate, the authors should carefully think about this and clearly explain how providers know the expected demand. 
}
\end{quotation}\vspace{-4mm}

\begin{itemize}
    \item Reviewer 2와 같은 결로 답변
\end{itemize}
% (재웅) 여러 포인트로 대답하면 좋을 것 같음.
%  1. 완전 zero information은 아님(저장된 용량, 전체 capacity 등의 정보 공시 하는 플랫폼 많음, 본인의 operation 역량에 대해서는 확실히 알고있음) 
%  2. Equilibrium의 원론적인 이야기 -> 결국은 수렴할 것

\begin{quotation}
{\em
\noindent \textbf{Senior Editor: }
(c) Probability of Service Failure: R1 correctly notes that is it is not rigorous to claim that “by setting a proper m and k, the chance of service failure is very small,” because price is endogenous and it affects the number of active providers. The authors should at least examine whether the desirable m and k will occur in equilibrium (under the equilibrium price). 
}
\end{quotation}\vspace{-4mm}

\begin{itemize}
    \item R1과 같이 답변
\end{itemize}

%%%%%%%%%%%%%%%%%%%%%%%%%%%%%%%%%%%%%%%%%%%%%%%%%%%%%%%%%%%%%%%%%%%%%%%%
\noindent\textbf{Main Concern: Writing}\\[-11mm]
%%%%%%%%%%%%%%%%%%%%%%%%%%%%%%%%%%%%%%%%%%%%%%%%%%%%%%%%%%%%%%%%%%%%%%%%%

\begin{quotation}
{\em
\noindent \textbf{Senior Editor: } I agree with R2 that the writing could be further improved. The reviewer generously provides detailed suggestions. I would encourage the authors to follow the suggestions and go even beyond to further polish the paper. }
\end{quotation} \vspace{-4mm}

\ju{R2 response 내용 정리}



%%%%%%%%%%%%%%%%%%%%%%%%%%%%%%%%%%%%%%%%%%%%%%%%%%%%%%%%%

\begin{quotation}
{\em
\noindent \textbf{Senior Editor: } I am very thankful to the two experienced reviewers for providing their valuable comments. I hope the authors can take advantage of their comments and further improve the paper. The authors did a good job clarifying many issues in the first round. I would encourage the authors to address the reviewers’ comments with the same standard. Good luck with the revision. }
\end{quotation}\vspace{-4mm}



\newpage
%%%%%%%%%%%%%%%%%%%%%%%%%%%%%%%%%%%%%%%%%%%%%%%%%%%%%%%%%%%%%%%
\noindent \underline{\large \bf Authors' Response to Reviewer 1}
%%%%%%%%%%%%%%%%%%%%%%%%%%%%%%%%%%%%%%%%%%%%%%%%%%%%%%%%%%%%%%%


%%%%%%%%%%%%%%%%%%%%%%%%%%%%%%%%%%%%
\begin{quotation}
{\em
\noindent \textbf{Reviewer 1: }
The revised manuscript addressed part of my concerns, while the following two major concerns are still not addressed: 

1. About the probability of service failure. The revised paper argues that by setting a proper $m$ and $k$, the chance of service failure is very small. Essentially, this is based on that the required $m$ and $k$ can always be achievable, regardless of price. However, it is not the case since price will certainly affect the number of active providers, which will affect the achievable $m$.  
 
The fact that the platform penalizes the providers who do not maintain certain uptime does NOT mean that the redundancy level can be achieved since it also largely depends on price, which is the decision to be made. 
 
All in all, this is still a big missing point and incorrect assumption. 
}
\end{quotation}\vspace{-4mm}

\begin{itemize}
    \item 코멘트와 같이 $m$ and $k$가 실제로 achievable 하지 않을 수 있고 이 컨디션을 찾았음을 보임. 이 컨디션이 발생하는 경우는 항상 모든 potential provider이 들어오는 경우
    \item 우리는 본 논문에서는 potential provider이 모두 들어오는 경우는 unrealistic하다고 판단하고, 이 경우는 제외
    \item 그 대신, extension or appendix(--미정)에서 이 case가 언제 나오는지에 대한 구체적인 threshold 제공 (혹은 그 이상의 solution을 제공 -- 미정--단순히 threshold 만 보여줄 지, 혹은 그 경우들에 대해서 optimal pricing 및 profit / surplus 를 다 비교할지는 아직 결정X -- numerical로는 충분히 보일 수 있음)
    \item forum같은 곳에 보면 실제로 profitable한지에 대한 엄청난 argue가 있음. 블록체인 기반의 기술인만큼 computing이 많이 들어가고, 이로 인하여 provider들이 실제로 profiable한지에 대한 concern이 있음(cost에 concern--computing 관련--이 있음)
    \item $\theta - \xi$ 그래프를 넣어두고 아래의 좌측 그래프와 같이 $\theta$가 뻗어져 나가는 형태가 아니라, 우측과 같이 적절한 $\theta$에서 optimal이 나오는 경우들에 대해 설명? 혹은 listing
\end{itemize}


\begin{figure}[ht!]
\centering
\includegraphics[width=5cm]{figure_3rd/figure_c_28_case_2.png} 
\includegraphics[width=5cm]{figure_3rd/figure_c_28_case_4.png} 
\end{figure}


% i) reviewer 1이 말한 case가 나오는 경우 $\rightarrow$ $\xi$가 얼마 이하로 떨어질때\\
% ii) i)이 나오는 경우는 우리의 optimal $\xi \theta$가 최소화 될 때 였으니까 $\theta$가 커지고 $\xi$가 작은 경우가 optimal이 되는 경우\\
% iii) 엄밀히보면 이 경우에도 사실 $\xi$가 특정 threshold 밑으로만 안떨어지면 상관은 없음 -- 이러한 관점으로 보자면, 저희가 cost를 얼마로 잡는지에 너무 depend하다. 하루에 1분 켜지면 cost가 엄청 크다고도 가정은 가능 -- 이 관점으로 접근하는 것은 조금 설득력이 떨어짐\\
% iv) 설명의 방향을 $\xi^o \theta^o$이라는 optimal이 있을 때, 이 optimal 값이 적절한 범위에서 나오는 cost structure들이 reasonable함을 보이는 것을 목적으로 함\\




% \yj{우리가 푼 모델에서 실제로 missing 된 내용이 있음. 극단적인 경우를 봤을 때, $\theta$가 매우 크고 $t$가 매우 작은 경우, 결과에서 틀린 부분이 발생함. 이 부분이 나오는 원인은 높은 $\theta$라기 보다는 낮은 $t$로부터 나오는 낮은 entry barrier가 더 중요한 역할을 함. 단순히 provider - renter 사이의 수를 비교하는 $\theta$와 관련된 term만 보면 우리는 first-best price(provider > $\theta    $renter), market clearing price (provider = $\theta$ renter)를 모두 반영하였음. 하지만, $\xi$가 낮아지게 되면 모든 provider가 다 들어오는 경우가 발생할 수 있고, 이 이후에는 $\theta$의 증가는 순수히 renter의 이탈을 만들어냄으로써만 발생하게 됨. 따라서 현실적으로 $\xi$의 lower bound가 존재하게 되고, 이는 $\theta$의 upper bound로 이어지게 됌. 이와 같이 기존 모델에서 고려하지 못했던 내용이 있음에도 불구하고 우리의 결과는 consistent함. 하지만 라이팅 관점에서 고민을 해야할 요소는 있음. 1. $\xi$의 lower bound를 본문에서는 임의로 주고(합당한 설명과 함께) 우리의 현재 결과를 그대로 쓴 후에 extension 느낌으로 $\xi$가 더 작아지고 $\theta$가 더 커지면 어떻게 되는 것인가?에 대해서 discussion하는 방법. 2. $\xi$의 lower bound를 개념적으로 주지 않는 상황에서 아예 식으로 모든 case를 풀어버리는 방법. 본문의 서술을 깔끔하게 하는 관점에서는 1.이 더 좋다고 판단은 되나, '합당한 설명'이라는 부분을 제대로 할 수 있는지 여부가 중요함}

% \ju{1을 뒷받침하는 적절한 설명? Operating cost가 너무 낮아서 모든 provider가 들어올 수 있는 상황은 현실에 맞지 않는다. 이는 P2P network가 상당한 network 사용량 및 computational burden을 동반하기 때문임. 이는 실제 플랫폼 운영 상에서 provider가 들어올 때, uptime을 제공할 때 여러 제약조건 및 penalty scheme을 이용하는 근거가 됨. 따라서 우리는 P2P network에 머무르는 데에 유의미한 비용이 발생하고 있고, 이를 감당할 수 없는 provider가 존재할 것이라는 가정이 매우 현실적임을 알 수 있다.}

% $\theta \xi$에 따라 cost structure가 달라질 수 있다.

%%%%%%%%%%%%%%%%%%%%%%%%%%%%%%%%%%%%

\begin{quotation}
{\em
\noindent \textbf{Reviewer 1: } 2.    Same for assumption on k. Similar as the comments above, the combo of m and k are not exogenous. The actual REALISED redundancy level will depend on the price schemes and decisions. 
}
\end{quotation}\vspace{-4mm}

\begin{itemize}
    \item 첫번째 질문에 대한 답이 두번째에 대한 답으로써 어느정도 작용할 수 있음. 본문 내의 순서를 바꿔서, redundancy 및 pricing 간의 endogenous decision을 다루었다는 것 만으로도 충분한 답변이 될 수 있을 것 이라고 판단됨.
    \item 부가적인 설명으로 redundancy level 자체가 자유자재로 바꿀 수 있는 것은 아니다. 따라서 pricing을 고려해서 redundancy level을 고려하는 것은 맞으나, redundancy level 이후에 pricing을 결정하는 순서로 결정되는 것이 자연스러움
    \item redundancy가 어려운 decision이 될 수 있다는 부분을 얘기하기 위하여 Sia 예시도 충분히 이용할 수 있음 (10-29 $\rightarrow 64-96$ 부수적인 목적으로써)
\end{itemize}


% (재웅) 우리가 설명이 충분하지 않았던 것인지, 무언가 고려하지 못한 게 있었던 것인지 확실하게 논의 필요

% + Redundancy m \& k를 임의로 바꾸기에 기술적 한계 등이 제약이 되기도 함\\
% + Sia가 10-29 $\rightarrow$ 64-96로 바꾸려고 시도 중에 있음 \\ 
% + 이와 같은 내용이 price랑 endogeneous하게 결정하는 것이 아니라, technical issue에 가깝다고 할 수 있음 

% provider의 function에서만 penalty를 반영하는 형태로 잡아서, uptime t와 penalty 사이의 관계를 보여준다. numerically하게
% 이 것이 세 가지 pricing scheme에 따라 다를 수도 있다.
% 가로축: penalty
% 세로축: failure probability

% $\xi$에 따른 lowerbound가 pricing scheme에 따라 다르다

% \yj{the actual realised redundancy level 이 명확히 어떤것을 의미하는지에 대해 약간 모호하기는 하지만, 블록체인이라는 기술 내에서의 신뢰도 이슈는 상당히 중요한 부분이라는 점도 같이 강조할 수 있을지 생각해볼 필요는 있을듯}

% + 한 번 정하고 나서 수정하기 쉽지 않으므로, 먼저 정하고 들어간다는 느낌

% \ju{(구체적으로 R1이 원하는 포인트가 무엇인지 자세히 논의해보기) 아주 좋은 포인트임. 이러한 algorithm 체계는 platform이 초기 단계에서 자체적으로 결정할 수 있는 여지가 있으며, 이러한 부분을 고려하여 revised manuscript에서는 redundancy algorithm에 대한 결정을 main model에 포함하였음. 한편 한 번 결정된 알고리듬을 변경하는 것은 매우 큰 작업이 들어가는 요인이므로 상대적으로 변경이 용이한 가격체계에 대한 결정 이전에 미리 알고리즘이 결정되는 형태를 유지하였다. 그리고 m \& k의 경우 일단 알고리듬 상에서 결정된 parameter인 경우 renter와 provider의 contract 단계에서 지정된 숫자만큼을 매칭하기 때문에 오차가 발생하지는 않으나, 중간에 계약을 맺은 provider가 bandwidth를 제공하지 않는 경우 your concern과 같은 현상이 발생할 수 있다. 이러한 현상을 방지하기 위해 플랫폼들은 penalty scheme 및 contracted storage 중 유효하지 않은 provider들을 유효한 provider로 교체하여 계약을 갱신하는 형태로 운영하고 있다.}

\newpage
%%%%%%%%%%%%%%%%%%%%%%%%%%%%%%%%%%%%%%%%%%%%%%%%%%%%%%%%%%%%%%%
\noindent \underline{\large \bf Authors' Response to Reviewer 2}
%%%%%%%%%%%%%%%%%%%%%%%%%%%%%%%%%%%%%%%%%%%%%%%%%%%%%%%%%%%%%%%


%%%%%%%%%%%%%%%%%%%%%%%%%%%%%%%%%%%%
%\begin{quotation}
%{\em
%\noindent \textbf{Reviewer 2: }
%The study examines pricing schemes of a peer-to-peer (P2P) storage sharing platform, which is a two-sided marketplace of renters with various storage and bandwidth needs and providers with available storage capacity. The pricing scheme considered is a two-part tariff: a storage fee that is dependent of the volume of the files and a bandwidth fee that is dependent of the access rate.
%Renters decide whether to adopt the P2P platform, while providers determine the uptime that influences the service level. Based on a model that captures the renters’ and providers’ utilities, the optimal pricing parameters are derived under different objectives of the platform: profit-seeking, consumer-welfare maximizing, and a mixed objective that combines both.
%}
%\end{quotation}\vspace{-4mm}
%%%%%%%%%%%%%%%%%%%%%%%%%%%%%%%%%%%%
\begin{quotation}
{\em
\noindent \textbf{Reviewer 2: }
In this revised manuscript, the most notable changes, as compared to the previous manuscript, are as follows. (1) The contribution of this paper is re-positioned as investigation of the impacts of different pricing schemes (subscription-based, two-part tariff, and hybrid). (2) The objective of the platform is changed to maximize its expected profit, and the impacts of the platform’s optimal pricing decisions on the total social surplus are examined. (3) Heterogeneity (renters’ willingnessto-pay and providers’ operating costs) is incorporated. (4) The motivation examples are enhanced. As can be seen, the authors have put substantial efforts to address the review team’s concerns raised in the last round, and, as a result, this paper has been completely revamped. I would like to thank them for their sincere efforts. On the positive side, I think the introduction section has been tightened with better explanations for the two motivation examples. Moreover, the potential contribution of this paper is clearer with the focus on comparing the impacts of the three common pricing schemes. Last, I think the results about the ranking of the performance of those pricing schemes based on different performance measure (platform’s profit vs total social surplus) and different supply-demand conditions (sufficient vs insufficient supply) are interesting. Since the paper has been improved in those areas mentioned above, it has good potential to get published. Having said that, I feel that there are some significant questions or concerns about the revised model and there is still plenty of room for improvement in the writing. Hence, I think the paper still needs to be strengthened and polished. My recommendation is Major Revision, hoping that the authors would clear those concerns in a satisfactory fashion (as they did in the last round) such that we can see a clearer path for acceptance.
}
\end{quotation}\vspace{-4mm}



%%%%%%%%%%%%%%%%%%%%%%%%%%%%%%%%%%%%%%%%%%%%%%%%%%%%%%%%%%%%%%%%%%%%%%%%%
\noindent\textbf{Concerns on key model features/assumptions:
}\\[-11mm]
%%%%%%%%%%%%%%%%%%%%%%%%%%%%%%%%%%%%%%%%%%%%%%%%%%%%%%%%%%%%%%%%%%%%%%%%%
\begin{quotation}
{\em
\noindent \textbf{Reviewer 2: }
1. The most distinct model features of storage sharing in cloud as compared to other P2P platforms are still not very clear or highlighted enough. First, one may argue that the pricing schemes under consideration as well as the heterogeneity of renters’ (buyers’) willingness to pay and providers’ (suppliers’) operating costs are also common features of other P2P platforms. Second, according to the authors’ response letter, it sounds like the redundancy algorithm is the distinct feature. But on the other hand, the study of the redundancy algorithm is only an extension to check the robustness of the model; in other words, the main results are somewhat independent of the redundancy algorithm, so it can be assumed as exogenously given. The authors should summarize the distinct model features specific to the storage sharing in cloud business model and emphasize them more.}
\end{quotation}\vspace{-4mm}

\begin{itemize}
    \item 두 가지요소를 둘 다 강조
    \item 1. renter의 사용량이 provider의 operating cost에 영향을 미침. 
    \item 2. redundancy level이라는 decentralized cloud storage system의 고유한 성질
    \item heterogeneity 자체는 common한 게 맞음. 따라서 우리는 renter이 provider에게 cost를 미치는 영향 자체를 고려하면서 둘 사이의 interdependence에 더욱 집중. -- 이 부분에 대한 설명은 왜 우리가 bandwidth를 무료로 주는 것만 고려하는지에 대한 설명과 연관이 되므로 이 둘 사이에서의 충돌이 나지 않도록 논리 전개 필요
    \item redundancy의 경우에는 모든 모델에 내재화되어 포함되어 있음 -- reviewer 1한테 설명하는 부분이 $\xi - \theta$ 사이의 역할이니까, redundancy 관련된 요소들을 포함함
\end{itemize}

% \yj{우리가 highlight할 distinct model features는 redundancy algorithm이 되어야 함. 기존에는 사용량에 비례하는 cost를 조금 더 강조하였었는데, 이 부분은 기존의 sharing economy 차원에서도 해석할 수 있음. e.g.) airbnb를 사용한다 하더라도 숙박 후 정리하는 과정들이 많아진다는 형태로 언급할 수 있음. 하지만 redundancy 의 개념은 본 논문에서 다루는 내용의 고유한 특성이라고 확실히 얘기할 수 있음 - 실제로 용량도 관련이 있으므로, 특히 platform을 운영하는 관점에서도 중요하다고 할 수 있음. 따라서 redundancy와 관련된 부분을 앞 chapter로 가져와서 아예 이 부분을 강조하는 식의 chapter 구성 변경이 필요할 것으로 판단. stage of the game을 redundancy algorithm 결정(systemetic한 decision이므로 가장 빠르게) -> provider 및 renter의 수 realize -> optimal price 결정으로 구성하는 것도 괜찮을듯함}

% \ju{다시 읽고 느낀 점은 algorithm 부분이 main model로 가는 것 자체로 이 부분은 사실상 해소될 것 같음.}

\begin{quotation}
{\em
\noindent \textbf {Reviewer 2: }2. The assumptions that a renter needs a storage space for her unit volume of files and that the provider can share his unit volume of storage space need a second thought. 

a. As can be seen, the model focuses on the heterogeneity of renters’ bandwidth usage level only, while the heterogeneity of their storage volume is absent due to the assumption of unit volume per renter. This assumption takes away an important tradeoff from the model. In reality, there are renters with large storage volume and low frequency of download request versus renters with small storage volume and  high frequency download request. The specific pricing scheme must have different impacts on renters with different storage and bandwidth needs, thus affecting the segmentation of renters. In short, I wonder whether the authors could build their model based on heterogeneity of not only bandwidth usage but also storage volume.
}
\end{quotation}\vspace{-4mm}

\begin{itemize}
    \item 본 논문에서는 그대로 volume 1로 homogeneous하게 가정
    \item volume heterogeneous는 모두 extension or appendix에 언급
    \item 세 가지 고려해야 할 부분, 1) writing 적인 측면에서 volume이 포함되어 있는 utility식을 어디에 처음 언급할 것인가 2) renter의 net utility가 $u\lambda$로 시작하는데, 이 부분이 용량과 비례하게 늘어난다는 부분에 대한 서술 3) 기존에는 free bandwidth allowance라는 느낌으로 $q\lambda_0$라는 빈도와 이어지는 개념으로써 무료 다운로드 용량을 정의 -- 이 부분은 저장과 비례하게 다운로드를 제공한다는 형태로 일관되게 서술 필요 
    \item volume heterogeneous가 있는 경우는 크게 두 가지 요소로 구분 1) 무료로 다운로드를 제공하는 용량이 저장용량에 비례하는 경우 2) 무료로 다운로드를 제공하는 용량이 고정되어 있는 경우 
    \item 이 모든 것에 대해서 volume을 high인 그룹 low 인 그룹으로 나누어 numerical study 실시
    \item 1)인 경우에는 volume이 모두 cancel out되어서 어차피 aggregate된 관점에서는 provider 및 platform 입장에서는 완전 동일
    \item 2)인 경우에 대하여 numerical study 를 진행한 것은 아래 그림과 같음 -- profit의 rank와 surplus가 역전되는 부분의 측면에서 기존의 결과와 완벽히 consistent하게 나옴 
    \item 이때의 효과를 요약하면 저장 용량이 큰 사람은 무료로 제공받는 용량이 작아지는 것으로 이어짐 -- 따라서 용량이 큰 사람이 많아지는 경우에는 two-part tariff와 비슷한 결과로 이어지게 됌.
    \item 추가적으로 용량이 큰 고객들의 b가 높다고 가정하고 (용량이 큰 고객 중에는 high frequency인 고객이 적다고) numerical을 돌리는 것도 가능 
\end{itemize}
\begin{figure}[ht!]
\centering
\includegraphics[width=6cm]{figure_3rd/volume_hetero_profit_1.png} 
\includegraphics[width=6cm]{figure_3rd/volume_hetero_surplus_1.png} 
\caption{$b = 2$, $v_h = 10$, $v_l = 1$, $q= 5$ $n_h : n_l = 1:90$}
\end{figure}

\begin{figure}[ht!]
\centering
\includegraphics[width=6cm]{figure_3rd/volume_hetero_profit_2.png} 
\includegraphics[width=6cm]{figure_3rd/volume_hetero_surplus_2.png} 
\caption{$b = 2$, $v_h = 10$, $v_l = 1$, $q= 5$ $n_h : n_l = 9:10$}
\end{figure}


% \yj{volume의 heterogeneous를 고려하기 위해서 free bandwidth allowance에 대한 두 가지 접근 방식이 있음. 1. 사용하는 용량에 비례하여 무료 빈도를 준다. 2. 무료로 주는 용량 자체가 정해져있다. 이 두 가지 경우에 대해서 모두 고려해야 할 것은 1. volume이 포함된 utility function을 어떻게 처리할 것이냐. - net utility를 $V u \lambda$로 할 것이냐? 한다면 이를 명시적으로 쓸 것이냐? 2. 각 results는 어디에 넣을 것이냐? - response letter만 넣을 것인가? 혹은 본문이나 appendix  or extension에 넣을 것이냐? + 추가적으로 결과는 consistent할 것이라고 추측하고 있지만 이 부분에 대해서도 검증은 필요함 (현재 numerical study 진행 중)}

% 1. 다름을 강조할 때는 redundancy에 집중\\
% 2. sharing economy니까 dropbox와 무엇이 다를까? - storage와 bandwidth 를 구분\\
% - dropbox는 infra가 있으므로 storage에 부과하는게 자연스러운 반면, sharing economy는 infra보다는 provider의 cost에 집중하기에 bandwidth에 집중하는게 자연스러움

% \yj{만약 volume에 대해서 numerical study를 진행한다면, high v인 고객들에 대해서는 b가 크고, low v 인 고객들에 대해서는 b가 작은 형태로 study 를 진행.}

% \ju{두 번째 이야기는 comment 4a에서 답하면 될듯?}

\begin{quotation}
{\em
\noindent \textbf{Reviewer 2: }
b. Compared to the concern on the unit volume of each renter, perhaps the assumption that each provider supplies unit volume of unused capacity is fine. But the implication is that the provider’s income and cost are both proportional to the volume of capacity supplied. If this is most likely the case in the storage sharing industry, the authors need to explain this point to justify this assumption; otherwise, I wonder whether the authors could relax this assumption, i.e., whether the impacts of a specific pricing scheme on providers with different capacity are different.}
\end{quotation}\vspace{-4mm}

\begin{itemize}
    \item 실제로 말이 된다 정도로 언급
    \item cost적인 측면에서 전기료 같은 실제 operating cost를 강조할 것인지--$\xi$에 대응, 다운로드 용량 등의 cognitive한 cost를 강조할 것인지 고려 필요--$\hat{\omega}_b$에 대응.
    \item 특히 두 term이 곱으로 되어있어서 설명에 대한 주의 필요 
    \item 이 답변은 초안 써보고 계속해서 수정하는 식으로.... 
    \item 계속 아이디어 붙여가면서 써가면서 수정
\end{itemize}


% \ju{Provider의 저장용량과 revenue가 비례하는 것은 구조 상으로 이야기하면 될 것 같고, operating cost가 비례하는 것은 조금 더 justify할 필요가 있을 것 같음. (논의 필요)}

\begin{quotation}
{\em
\noindent \textbf{Reviewer 2:  }3. A provider’s decision is based on his knowledge of the expected demand for the storage service and the bandwidth service, which, however, are dependent on conditions of the whole market. In practice, how can an individual provider possess such information? Does the platform facilitate information sharing?}
\end{quotation}\vspace{-4mm}

\begin{itemize}
    \item 완전 zero information은 아님(저장된 용량, 전체 capacity 등의 정보 공시 하는 플랫폼 많음, 본인의 operation 역량에 대해서는 확실히 알고있음
    \item 1. online platform이라는 특성상 각 provider과 renter의 behavior을 빠르게 tracking 할 수 있음 2. provider와 renter의 가입절차가 다르므로 각 player들도 구분할 수 있음. $\rightarrow$ 결과적으로 equilibrium으로 빠르게 수렴한다고 할 수 있음
\end{itemize}



\begin{quotation}
{\em
\noindent \textbf{Reviewer 2:  }4. Other model assumptions to justify: a. The assumption about independence of a renter’s utility from the download bandwidth service from her download frequency is questionable. One may argue that the two parameters are positively correlated as the more frequent the renter needs to access the storage, the more valuable the bandwidth service means to her.}
\end{quotation}\vspace{-4mm}

\begin{itemize}
    \item utility를 [0, 0.5] [0.5, 1]로 쪼개서 numerical study
    \item [0, 0.5]에게 high b, [0.5, 1]에게 low b를 주어서 결과 값 보이고 consistent하다는 것을 서술하는 쪽으로 진행
\end{itemize}


% \yj{independence자체에 대해서 직접다루기보다는 pareto distribution에서 b가 변화하더라도 결과가 consistent하다는 것을 numerical로 보임으로써 u와 lambda의 관계가 달라지더라도 결과가 유지된다는 것에 집중하여 argue}

% \ju{+ b의 변화가 correlation과 거의 같음을 설명하기}
% \yj{설명 가능한지에 대한 검토 필요}


\begin{quotation}
{\em
\noindent \textbf{Reviewer 2:  }b. The assumption that the renters’ bandwidth usage level follows a Pareto distribution
needs better justification. Are there other studies (or empirical evidence) than Li and Kumar (2018) that also adopt this assumption? Can the same results be derived based on other distributions (e.g., Poisson)?}
\end{quotation}\vspace{-4mm}

\begin{itemize}
    \item Poisson distribution까지는 굳이 하지 말고, pareto distribution 내에서 b를 변화가면서 (decreasing 하지만 tail의 두께가 변화하는 형태) numerical study 진행
    \item b가 커짐에 따라서 각 pricing scheme에 따른 차이가 줄어듬 $\rightarrow$ 이는 pricing scheme이 영향을 주는 부분들은 결국 bandwidth 사용량이 많은 고객들인데, b가 커짐에 따라 이 비율이 줄어들게 됌
    \item 따라서 surplus적인 측면에서도 $n_p$가 적은 경우에 provider에게 two-part tariff 등을 통해 incentive를 주어서 전체 total surplus가 증가하는 효과도 작아짐(효과 자체는 유지)
\end{itemize}


% \yj{distribution을 바꿔서 numerical 시행하는 것에 대해서는 검토가 필요 - 한다면 heavy tail distribution으로 한정지어서 진행?}

% \ju{다른 논문은 인용할 예정, 혹시 Poisson으로 풀리는지? (expected value는 간단할텐데) - numerical로 하면 될 것이기는 한데, 모든 분포를 고려해야할지(concave한 형태 등)에 대한 고민 필요}

%%%%%%%%%%%%%%%%%%%%%%%%%%%%%%%%%%%%%%%%%%%%%%%%%%%%%%%%%%%%%%%%%%%%%%%%%
\noindent\textbf{Suggestions on writing of some subsections:}\\[-11mm]
%%%%%%%%%%%%%%%%%%%%%%%%%%%%%%%%%%%%%%%%%%%%%%%%%%%%%%%%%%%%%%%%%%%%%%%%%
\begin{quotation}
{\em
\noindent \textbf{Reviewer 2:  }1. In subsection 3.2, the presentation of the provider’s profit function is somewhat unclear. The reader may wonder how to derive the expected storage and bandwidth volume for each provider ($\hat{\omega}_s, \hat{\omega}_b$, and $\hat{\omega}_{bp}$), while it turns out that the derivations will be introduced later in section 4. Moreover, to facilitate the reader’s understanding of the formulation, it would be better to explain why $\hat{\omega}_b = \hat{\omega}_s \cdot \nu_b / (\theta \nu_s)$ holds (i.e., the meaning of the ratio $\nu_b / (\theta \nu_s) $is not immediately clear to the reader). }
\end{quotation}\vspace{-4mm}

\begin{itemize}
    \item $\frac{\nu_b}{\theta \nu_s}$가 가지는 `frequency'라는 해석적인 요소가 unclear하게 받아들여졌다는 느낌
    \item $\hat{\omega}_b = \nu_b \cdot \frac{\hat{\omega}_s}{\theta \nu_s}$ 와 같이 표현을 바꿔서, 다운로드 용량 자체가 저장 용량의 크기에 비례하여 증가/감소 한다.의 의미를 명확히 제공 
\end{itemize}


\begin{quotation}
{\em
\noindent \textbf{Reviewer 2:  }a. Similarly, it would be better to explain, for the platform’s decision problem, why $V_{bp} = V_s \cdot \nu_{bp} / (\theta \nu_s)$(i.e., the meaning of the ratio $\nu_{bp} / (\theta \nu_s) $ is not immediately clear to the reader).}
\end{quotation}\vspace{-4mm}
\begin{itemize}
    \item 1과 마찬가지로 $V_{bp} = \nu_{bp} \cdot \frac{V_s}{\theta \nu_s}$
\end{itemize}


\begin{quotation}
{\em
\noindent \textbf{Reviewer 2:  }b. On a related note, in the formulation of the providers’ total surplus PS (at the end of page 20), it seems that the authors assume $\rho_j$ is uniformly distributed over $[0,1]$, but I didn’t find the statement of this assumption (or perhaps I just missed it).}
\end{quotation}\vspace{-4mm}


\begin{quotation}
{\em
\noindent \textbf{Reviewer 2:  }2. The literature review needs to be streamlined. For example, on top of page 7, the discussion is about “this study also extends the literature on sharing platforms for computing resources,” then, on the same page, the discussion continues to be about “emerging platforms have enabled sharing various computing resources.” But, later on page 8, again the discussion goes back to “this study also contributes to the literature on the pricing for resource sharing services.” The writing of these paragraphs feels like on an ad hoc basis. Maybe a better sequence is as follows: (1) pricing (and capacity management) in centralized public cloud; (2) P2P sharing of computing resources; (3) literature on other forms of sharing economies. Finally, in the last paragraph, summarize the distinct model features and contribution to the existing literature.}
\end{quotation}\vspace{-4mm}


$U_i = \lambda_i u_i - \theta p_s - (\lambda_i - q \lambda_0) p_b$\\
i) $q \lambda_0$: 용량당 무료로 주는 frequency인가?\\
ii) $v = 1$이라는것을 암묵적으로 둔 다음에, $q \lambda_0 v = q \lambda_0$라는 것이 무료로 주는 용량을 의미하는 것인가?\\


$U_i = \lambda_i u_i v - \theta p_s v - \lambda_i p_b v + q \lambda_0 p_b v$\\
$U_i = \lambda_i u_i v - \theta p_s v - \lambda_i p_b v + q v_0 p_b $\\
용량이 고려가 될 경우 net utility:\\
$\lambda_i u_i v $ 이것에 대한 당위성이 부여가능한지? 만약 부여가능하다면 이걸 어디에다가 언급할건지?\\

Main model: $v=1 $ homogeneous하다고 가정하고, 기존 식 $U_i = \lambda_i u_i - \theta p_s - (\lambda_i - q \lambda_0) p_b$ 만 설명, footnote 등으로 appendix 혹은 extension에서 v가 heterogeneous한 경우를 다루었다고 언급 정도만\\
extension or appendix: $v > 1$인 경우를 언급 $\rightarrow$ 이때 우리는 net utility가 용량에 비례한다고 가정 (이 부분에 대해서 당위성을 부여할 수 있는지에 대한 고려 필요) \\

heterogeneity : $v_l = 1 \rightarrow \gamma$ / $v_h >1 \rightarrow 1-\gamma$\\ 

i) 만약에 용량이 큰 사람에 대해서 이에 비례하는 저장용량에 비례하여 무료 용량을 준다고 가정했을 때(기존의 $q \lambda_0$) $\rightarrow $ 이 경우는 우리가 하는 경우와 완전 일치 ($v$가 cancel out)\\
$U_i = \lambda_i u_i v - \theta p_s v - (\lambda_i - q \lambda_0) p_b v $\\
$v$ 가 큰 경우 / 작은 경우를 $b$를 다르게 set해서 numerical 시행\\
용량 작은경우/ 큰 경우 관계없이 : $q \lambda_0$의 frequency만큼 무료로 사용가능\\

ii) 그런데 만약, 비례하지 않는 경우, $v_0$만큼 주는 경우\\
$U_i = \lambda_i u_i v - \theta p_s v - (\lambda_i v - v_0) p_b $\\
$v_l = 1$ / $v_h, v_0$\\
용량이 작은 경우: $v_0$의 frequency만큼\\
용량이 큰 경우: $\frac{v_0}{v_h}$의 frequency만큼\\



수정할 내용:\\
contents:\\
1) redundancy algorithm $\rightarrow$ 각 provider들에게 할당되는 cost가 매우 작아서, 모든 provider들이 in을 하는 경우\\
1-1) 어떠한 region을 reasonable 가정둔 후, 본문에서는 기존 결과 그대로 진행\\
1-2) 그런데, 만약 provider들의 cost가 너무 작아져서 모든 provider들이 들어오는 경우가 된다면 어떤 일이 발생할 수 있는지에 대해서 extension or appendix에 추가 \\

2) Volume heterogeneous: 위에서 언급한 대로 진행\\

3) redundancy algorithm을 더 차별점으로 강조하기 위해서, $\xi \theta$가 어떠한 형태로 구성되는지 numerical analysis을 통하여 설명. 이때 그냥 A일때는 B다 라는 형식으로 설명하기 보다는, 어떠어떠한 cost structure일 때는 $\theta$를 그리 높이지 않는 것이 좋다 라는 것을 위주로, 우리가 $\xi$는 적당히 크고, $\theta$는 적당히 작은 경우를 고려하는 것이 바람직하다 라는 것을 목표로 설명 \\


writing: \\
1. redundancy algorithm 자체를 앞으로 빼서 구성\\
2. 이 부분을 우리 모델의 main diffence로 설명 \\
3. bandwidth만 무료로 제공하는 것을 다룸 $\rightarrow$ 기존에는 model이 가지는 main difference를 bandwidth로 나오는 cost에 조금 비중을 뒀었음 $\rightarrow $ 저장이 아니라 다운로드 자체에만 heterogeneous를 둔 것에 대해 당위성이 어느정도 부여되었다. 그런데 key feature를 redundancy algorithm으로 간다면 `왜' bandwidth만 무료로 주는 것을 고려하냐! 에 대한 어느정도 다른 당위성 부여 방법이 필요할듯하다. 






\end{document} 
